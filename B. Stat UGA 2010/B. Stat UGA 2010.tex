\documentclass[11pt, a4paper]{article}

\usepackage[top=1 in, bottom = 1 in, left = 1 in, right = 1 in ]{geometry}

\usepackage{amsmath, amssymb, amsfonts}
\usepackage{enumerate}

\title{B. Stat UGA 2010}
\author{Ananda Biswas}
\date{}

\begin{document}

\maketitle


\begin{enumerate}


	\item[15.] For any real number $x$, let $ \tan^{-1} (x) $ denote the unique real number $ \theta $ in $ (-\pi / 2, \pi / 2) $ such that $ \tan \theta = x $. Then 
\begin{center}
$ \lim\limits_{n \to \infty } \displaystyle{\sum\limits_{m = 1}^n} \tan^{-1} \dfrac{1}{1+m+m^2}  $
\end{center}

	\begin{enumerate}[(A)]
		\item is equal to $ \pi / 2 $;
		\item is equal to $ \pi / 4 $;
		\item does not exist;
		\item none of the above.
	
	\end{enumerate}

\begin{flushleft}{\textbf{Answer ::}}\end{flushleft}
Now, 
	\begin{align*}
	\tan^{-1} \dfrac{1}{1+m+m^2} &= \tan^{-1} \dfrac{(m+1)-m}{1 + m(m+1)} \\
	&= \tan^{-1} (m+1) - \tan^{-1} m
	\end{align*}	
	\begin{align*}
	\therefore \displaystyle{\sum\limits_{m = 1}^n} \tan^{-1} \dfrac{1}{1+m+m^2} &= \displaystyle{\sum\limits_{m = 1}^n} \big[ \tan^{-1} (m+1) - \tan^{-1} m \big] \\
	&= \tan^{-1} (n+1) - \tan^{-1} 1  
	\end{align*}
	\begin{align*}
	\therefore \lim\limits_{n \to \infty } \displaystyle{\sum\limits_{m = 1}^n} \tan^{-1} \dfrac{1}{1+m+m^2} &= \lim\limits_{n \to \infty } \big[ \tan^{-1} (n+1) - \tan^{-1} 1 \big] \\
	&= \tan^{-1} (\infty) - \tan^{-1}	\\
	&= \dfrac{\pi}{2} - \dfrac{\pi}{4} \\
	&= \dfrac{\pi}{4}
	\end{align*}
\begin{flushright}
\textbf{(B) is equal to $ \dfrac{\pi}{4} $}
\end{flushright}
\pagebreak






	\item[22.] Suppose that $ \alpha $ and $ \beta $ are two distinct numbers in the interval $ (0,\pi) $. If \begin{center}
	$ \sin \alpha + \sin \beta = \sqrt{3} (\cos \alpha - \cos \beta) $
	\end{center} then the value of $ \sin 3\alpha + \sin 3\beta $ is
	\begin{enumerate}[(A)]
		\item 0;
		\item $ 2\sin \dfrac{3(\alpha + \beta)}{2} $;
		\item $ 2\cos \dfrac{3(\alpha - \beta)}{2} $
		\item $ \cos \dfrac{3(\alpha - \beta)}{2} $.
	\end{enumerate}
	
\begin{flushleft}{\textbf{Answer ::}}\end{flushleft}
Given that, 
$\sin \alpha + \sin \beta = \sqrt{3} (\cos \alpha - \cos \beta)$ 

\begin{align*}
\therefore 2 \sin \dfrac{\alpha + \beta}{2} \cos \dfrac{\alpha - \beta}{2} &= \sqrt{3} \cdot 2 \sin \dfrac{\alpha + \beta}{2} \sin \dfrac{\beta - \alpha}{2}\\ \\
or, \cos \dfrac{\beta - \alpha}{2} &= \sqrt{3} \cdot \sin \dfrac{\beta - \alpha}{2} \\ \\
or, \tan \dfrac{\beta - \alpha}{2} &= \dfrac{1}{\sqrt{3}} \\ \\
or, \tan \dfrac{\beta - \alpha}{2} &= \tan \dfrac{\pi}{6}\\ \\
\therefore \dfrac{\beta - \alpha}{2} &= \dfrac{\pi}{6} \Rightarrow \beta - \alpha = \dfrac{\pi}{3}
\end{align*}

Now,
\begin{align*}
\sin 3\alpha + \sin 3\beta &= 2 \sin \dfrac{3(\alpha + \beta)}{2} \cdot \cos \dfrac{3(\alpha - \beta)}{2} \\
&= \quad 2 \sin \dfrac{3(\alpha + \beta)}{2} \cdot \cos \left( \dfrac{3}{2}\cdot\dfrac{\pi}{3}\right) \big[\because \cos a = \cos -a\big] \\
&= 2 \sin \dfrac{3(\alpha + \beta)}{2} \cdot \cos \dfrac{\pi}{2}\\
&=0 \quad \Big[\because \cos \dfrac{\pi}{2}= 0 \Big]
\end{align*}

\begin{flushright}
\textbf{(A) 0}
\end{flushright}
 





	







\end{enumerate}


\end{document}