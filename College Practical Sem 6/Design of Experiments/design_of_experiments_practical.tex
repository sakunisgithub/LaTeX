\documentclass[11pt, a4paper]{article}

\usepackage[top=1 in, bottom = 1 in, left = 1 in, right = 1 in ]{geometry}

\usepackage{amsmath, amssymb, amsfonts}
\usepackage{enumerate}
\usepackage{multirow}
\usepackage{hhline}
\usepackage{array}
\usepackage{longtable}
\usepackage{graphicx}

\title{\textbf{QUESTIONS}}

\author{}
\date{}

\begin{document}

\maketitle

\begin{enumerate}

%01

	\item A set of data involving four ``tropical feed stuffs A, B, C, D" tried on 20 chicks is given below. All the twenty chicks are treated alike in all respects except the feeding treatments and each feeding treatment is given to 5 chicks. Analyse the data.
	
	\begin{table}[!htbp]
	\def\arraystretch{1.5}
	
	\begin{center}
	\begin{tabular}{|>{\centering}m{1cm}|>{\centering}m{1cm}>{\centering}m{1cm}>{\centering}m{1cm}>{\centering}m{1cm}>{\centering}m{1cm}|>{\centering\arraybackslash}m{2cm}|}
	
	\hline
	
	Feed & \multicolumn{5}{c|}{Gain in Weight} & Total $T_i$ \\
	
	\hline
	
	A & 55 & 49 & 42 & 21 & 52 & 219 \\
	
	
	B & 61 & 112 & 30 & 89 & 63 & 355 \\
	
	
	C & 42 & 97 & 81 & 95 & 92 & 407 \\
	
	
	D & 169 & 137 & 169 & 85 & 154 & 714 \\
	
	\hline
	
	
	\end{tabular}
	\end{center}
	\end{table}
	
	
	
	
	
	
	
	
	
	
	
	
	
	
	
	
	
	
%02
	\item Consider the results given in the following table for an experiment involving six treatments in four randomised blocks. The treatments are indicated by numbers within parentheses.
	
	\begin{table}[!htbp]
	\def\arraystretch{1.5}
	
	\begin{center}
	\begin{tabular}{|>{\centering}m{1.2cm}|>{\centering}m{1.5cm}>{\centering}m{1.5cm}>{\centering}m{1.5cm}>{\centering}m{1.5cm}>{\centering}m{1.5cm}>{\centering\arraybackslash}m{1.5cm}|}
	
	\hline
	
	Blocks & \multicolumn{6}{c|}{Yield for a randomised block experiment} \\
	
	\hline
	
	\multirow{2}{*}{1} & (1) & (3) & (2) & (4) & (5) & (6) \\
	
	& 24.7 & 27.7 & 20.6 & 16.2 & 16.2 & 24.9 \\
	
	\hline
	
	\multirow{2}{*}{2} & (3) & (2) & (1) & (4) & (6) & (5) \\
	
	& 22.7 & 28.8 & 27.3 & 15.0 & 22.5 & 17.0 \\
	
	\hline
	
	\multirow{2}{*}{3} & (6) & (4) & (1) & (3) & (2) & (5) \\
	
	& 26.3 & 19.6 & 38.5 & 36.8 & 39.5 & 15.4 \\
	
	\hline
	
	\multirow{2}{*}{4} & (5) & (2) & (1) & (4) & (3) & (6) \\
	
	& 17.7 & 31.0 & 28.5 & 14.1 & 34.9 & 22.6 \\
	
	\hline
	\end{tabular}
	\end{center}
	
	\end{table}
	
	Test whether the treatments differ significantly. Also
	\begin{enumerate}[(i)]
	\item determine the critical difference between the means of any two treatments;
	\item obtain the efficiency of this design relative to its layout as $C.R.D.$
	
	\end{enumerate}
	
	
	
	
	
	
	
\newpage
	
	
	
	
	
	
	

%03
	\item An experiment was carried out to determine the effect of claying the ground on the field of barley grains; amount of clay used were as follows : \\
	$A$ : No clay \\
	$B$ : Clay at 100 per acre \\
	$C$ : Clay at 200 per acre \\
	$D$ : Clay at 300 per acre. \\
	The yields were in plots of 8 meters by 8 meters and are given in the following table.
	
	\begin{table}[!htbp]
	\def\arraystretch{1.5}
	
	\begin{center}
	\begin{tabular}{>{\centering}m{1cm}|>{\centering}m{1.5cm}|>{\centering}m{1.5cm}|>{\centering}m{1.5cm}|>{\centering\arraybackslash}m{1.5cm}|}
	
	\multicolumn{1}{c}{} & \multicolumn{1}{c}{$I$} & \multicolumn{1}{c}{$II$} & \multicolumn{1}{c}{$III$} & \multicolumn{1}{c}{$IV$} \\
	
	\hhline{~----}
	
	\multirow{2}{*}{$I$} & $D$ & $B$ & $C$ & $A$ \\
	
	& 29.1 & 18.9 & 29.4 & 5.7 \\
	
	\cline{2-5}
	
	\multirow{2}{*}{$II$} & $C$ & $A$ & $D$ & $B$ \\
	
	& 16.4 & 10.2 & 21.2 & 19.1 \\
	
	\cline{2-5}
	
	\multirow{2}{*}{$III$} & $A$ & $D$ & $B$ & $C$ \\
	
	& 5.4 & 38.8 & 24.0 & 37.0 \\
	
	\cline{2-5}
	
	\multirow{2}{*}{$IV$} & $B$ & $C$ & $A$ & $D$ \\
	
	& 24.9 & 41.7 & 9.5 & 28.9 \\
	
	\cline{2-5}
	
	\end{tabular}
	\end{center}
	
	\end{table}
	
	\begin{enumerate}[(i)]
	\item Perform the ANOVA and calculate the critical difference for the treatment mean yields.
	\item Calculate the efficiency of the above Latin Square Design over 
		\begin{enumerate}[(a)]
		\item $R.B.D.$ 
		\item $C.R.D.$
		\end{enumerate}		 
	\item Yield under `A' in the first column was missing. Estimate the missing value and carry out the ANOVA.
	
	\end{enumerate}
	
	
	
	
	
	
	


















%04
	\item A varietal trial was conducted at a Research Station. The design adopted for the same was five randomised blocks of 6 plots each. The yields in lb. per plot (of $\frac{1}{20}$th of an acre) obtained from the experiment are given in the following table.
	
	\begin{table}[!htbp]
	\def\arraystretch{1.5}
	
	\begin{center}
	\begin{tabular}{|>{\centering}m{1.5cm}|>{\centering}m{1.5cm}>{\centering}m{1.5cm}>{\centering}m{1.5cm}>{\centering}m{1.5cm}>{\centering}m{1.5cm}>{\centering\arraybackslash}m{1.5cm}|}
	
	\hline
	
	\multirow{2}{*}{Blocks} & \multicolumn{6}{c|}{Varieties} \\
	
	\cline{2-7}
	
	& $V_1$ & $V_2$ & $V_3$ & $V_4$ & $V_5$ & $V_6$ \\
	
	\hline
	
	$I$ & 30 & 23 & 34 & 25 & 20 & 13 \\
	
	$II$ & 39 & 22 & 28 & 25 & 28 & 32 \\
	
	$III$ & 56 & 43 & 43 & 31 & 49 & 17 \\
	
	$IV$ & 38 & 45 & 36 & 35 & 32 & 20 \\
	
	$V$ & 44 & 51 & 23 & 58 & 40 & 30 \\
	
	\hline
	
	\end{tabular}
	\end{center}
	
	\end{table}
	
	Analyse the design and comment on your findings.
	
	
	
	
	
	
	
	
	
	
	
\newpage
	
	
	
	
	
	
	
	
	
	
	
%05
	\item The following data were obtained from an experiment using the treatments : $0.32$ \% of Blitox, $0.16$ \% of Dithane z-78, $0.09$ \% of Brestan-60 and control. After sowing rhizomes of the matgrass \textit{Cyperus tagetum Roxb} in four plots in each of three villages, the above four treatments were applied at random to the plots in each village after 30 days of sowing. The yields in gm. of 30 sq. cm. cutting per plot after 120 days are given below. Analyse the data to find out if there are any significant treatment effects.
	
	\begin{table}[!htbp]
	\def\arraystretch{1.5}
	
	\begin{center}
	\begin{tabular}{|>{\centering}m{3cm}|>{\centering}m{3cm}|>{\centering}m{3cm}|>{\centering\arraybackslash}m{3cm}|}
	
	\hline
	
	\multirow{2}{*}{Treatment} & \multicolumn{3}{c|}{Village} \\
	
	\hhline{~---}
	
	& I & II & III \\
	
	\hline
	
	Blitox & 678.2 & 510.2 & 531.2 \\
	
	Dithane z-78 & 703.2 & 689.5 & 611.2 \\
	
	Brestan-60 & 736.8 & 574.2 & 573.7 \\
	
	Control & 556.4 & 510.2 & 500.0 \\
	
	\hline
	
	\end{tabular}
	\end{center}
	
	\end{table}
	
	
	
	
	
	
	
	
	
	
	
	
	
	
	
	
	
	
	
	
	
	
	
%06
	\item Analyse the following randomised block design after estimating the missing value. Also compare the treatments $T_1$ and $T_2$.
	
	\begin{table}[!htbp]
	\def\arraystretch{1.5}
	
	\begin{center}
	\begin{tabular}{|>{\centering}m{2.5cm}|>{\centering}m{2.5cm}|>{\centering}m{2.5cm}|>{\centering}m{2.5cm}|>{\centering\arraybackslash}m{2.5cm}|}
	
	\hline
	
	\multirow{2}{*}{Treatment} & \multicolumn{4}{c|}{Blocks} \\
	
	\hhline{~----}
	
	& I & II & III & IV \\
	
	\hline
	
	$T_1$ & 19.1 & $-$ & 22.5 & 25.5 \\
	
	\hline
	
	$T_2$ & 26.0 & 28.0 & 27.0 & 33.0 \\
	
	\hline
	
	$T_3$ & 20.5 & 28.5 & 21.5 & 25.5 \\
	
	\hline
	
	\end{tabular}
	\end{center}
	
	\end{table}
	
	
	
	
	
	
	
	
	
	
	
	
	
	
	
	
	
	
	
	
	
	
	
	
%07
	\item The following table gives the yield of wheat (kgs./plot) as observed in an experiment carried out in a 5 $\times$ 5 Latin Square. The five manurial treatments are indicated by $A$, $B$, $C$, $D$ and $E$. Obtain an estimate of the missing value and analyse the design.
	
	\begin{table}[!htbp]
	\def\arraystretch{1.44}
	
	\begin{center}
	\begin{tabular}{>{\centering}m{1.5cm}|>{\centering}m{1.5cm}|>{\centering}m{1.5cm}|>{\centering}m{1.5cm}|>{\centering}m{1.5cm}|>{\centering\arraybackslash}m{1.5cm}|}
	
	\multicolumn{1}{c}{} & \multicolumn{1}{c}{1} & \multicolumn{1}{c}{2} & \multicolumn{1}{c}{3} & \multicolumn{1}{c}{4} & \multicolumn{1}{c}{5} \\
	
	\hhline{~-----}
	
	\multirow{2}{*}{1} & $B$ & $C$ & $A$ & $D$ & $E$ \\
	
	& 57.8 & 48.6 & 33.4 & 53.5 & 41.8 \\
	
	\hhline{~-----}
	
	\multirow{2}{*}{2} & $D$ & $E$ & $C$ & $B$ & $A$ \\
	
	& 50.5 & 45.5 & 51.8 & 52.6 & 31.9 \\
	
	\hhline{~-----}
	
	\multirow{2}{*}{3} & $A$ & $D$ & $B$ & $E$ & $C$ \\
	
	& 46.1 & 47.9 & 55.6 & $-$ & 53.3 \\
	
	\hhline{~-----}
	
	\multirow{2}{*}{4} & $C$ & $B$ & $E$ & $A$ & $D$ \\
	
	& 58.2 & 55.1 & 43.2 & 38.8 & 53.3 \\
	
	\hhline{~-----}
	
	\multirow{2}{*}{5} & $E$ & $A$ & $D$ & $C$ & $B$ \\
	
	& 53.0 & 41.0 & 48.7 & 54.6 & 55.7 \\
	
	\hhline{~-----}
	
	
	
	\end{tabular}
	\end{center}
	\end{table}		
	
	




















%08
	\item Consider the following $2^2$ factorial experiment involving 2 factors N and S each at two levels - 0 and 1.
	
	\begin{table}[!htbp]
	\def\arraystretch{2}
	
	\begin{center}
	\begin{tabular}{|>{\centering}m{2cm}||>{\centering}m{1.5cm}|>{\centering}m{1.5cm}|>{\centering}m{1.5cm}|>{\centering\arraybackslash}m{1.5cm}|}

	\hline
	
	\multirow{2}{*}{Block} & \multicolumn{4}{c|}{Treatment Combination} \\
	
	\hhline{~----}
	
	& $(1)$ & $n$ & $s$ & $ns$ \\
	
	\hline	
	
	$I$ & 117 & 124 & 106 & 125 \\
	
	\hline
	
	$II$ & 120 & 124 & 117 & 124 \\
	
	\hline
	
	$III$ & 111 & 127 & 114 & 126 \\
	
	\hline
	
	$IV$ & 108 & 131 & 112 & 125 \\
	
	\hline
	
	$V$ & 73 & 138 & 97 & 95 \\
	
	\hline
	
	$VI$ & 81 & 158 & 117 & 125 \\
	
	\hline 
	\end{tabular}
	\end{center}
	\end{table}
	
	Analyse the design. Does treatment effect $N$ differ from treatment effect $S$ significantly ?
	
	
	
	
	
	
	
	
	
	
	
	
	
	
	
	
	
	
	
%09
	\item Analyse the following $2^3$ factorial experiment in blocks of 4 plots, involving three fertilisers $N$, $P$ and $K$ each at two levels.
	
	\begin{table}[!htbp]
	\def\arraystretch{1.35}
	
	\begin{center}
	\begin{tabular}{>{\centering}m{2cm}|>{\centering}m{1cm}|>{\centering}m{1cm}|>{\centering}m{1cm}|>{\centering\arraybackslash}m{1cm}|}
	
	\multicolumn{1}{c}{} & \multicolumn{4}{c}{Replicate $I$} \\
	
	\hhline{~----}
	
	\multirow{2}{*}{Block 1} & $np$ & $npk$ & $(1)$ & $k$ \\
	
	& 101 & 111 & 75 & 55 \\
	
	\hhline{~----}
	
	\multirow{2}{*}{Block 2} & $p$ & $n$ & $pk$ & $nk$ \\
	
	& 88 & 90 & 115 & 75 \\
	
	\hhline{~----}
	
	\end{tabular}
	\end{center}
	
	\end{table}


	
	
	
	\begin{table}[!htbp]
	\def\arraystretch{1.35}
	
	\begin{center}
	\begin{tabular}{>{\centering}m{2cm}|>{\centering}m{1cm}|>{\centering}m{1cm}|>{\centering}m{1cm}|>{\centering\arraybackslash}m{1cm}|}
	
	\multicolumn{1}{c}{} & \multicolumn{4}{c}{Replicate $II$} \\
	
	\hhline{~----}
	
	\multirow{2}{*}{Block 3} & $(1)$ & $npk$ & $nk$ & $p$ \\
	
	& 125 & 95 & 80 & 100 \\
	
	\hhline{~----}
	
	\multirow{2}{*}{Block 4} & $np$ & $k$ & $pk$ & $n$ \\
	
	& 115 & 95 & 90 & 80 \\
	
	\hhline{~----}
	
	\end{tabular}
	\end{center}
	
	\end{table}
	
	
	
	
	
	
	\begin{table}[!htbp]
	\def\arraystretch{1.35}
	
	\begin{center}
	\begin{tabular}{>{\centering}m{2cm}|>{\centering}m{1cm}|>{\centering}m{1cm}|>{\centering}m{1cm}|>{\centering\arraybackslash}m{1cm}|}
	
	\multicolumn{1}{c}{} & \multicolumn{4}{c}{Replicate $III$} \\
	
	\hhline{~----}
	
	\multirow{2}{*}{Block 5} & $pk$ & $nk$ & $(1)$ & $np$ \\
	
	& 75 & 100 & 55 & 92 \\
	
	\hhline{~----}
	
	\multirow{2}{*}{Block 6} & $n$ & $npk$ & $p$ & $k$ \\
	
	& 53 & 76 & 65 & 82 \\
	
	\hhline{~----}
	
	\end{tabular}
	\end{center}
	
	\end{table}
	
	
	
	
	
	
	
	
	
	
	
	
	
	
	
	
	
	
	
	
	
	
	
%10
	\item An experiment was performed by Gretchen Krueger at Arizona State University to determine how the pan material, the brand of brownie mix and the stirring method affect the scrumptiousness of brownies. \\
	
	The factor levels were:
	
	\begin{table}[!htbp]
	\def\arraystretch{1.5}
	
	\begin{center}
	\begin{tabular}{|>{\centering}m{5cm}||>{\centering}m{2cm}|>{\centering\arraybackslash}m{2cm}|}
	
	\hline
	
	Factor & Low $(-)$ & High $(+)$ \\
	
	\hline
	\hline
	
	A = pan material & Glass & Aluminium \\
	
	\hline
	
	B = stirring method & Spoon & Mixer \\
	
	\hline
	
	C = brand of mix & Expensive & Cheap \\
	
	\hline
	
	\end{tabular}
	\end{center}
	
	\end{table}

	
	The response variable was scrumptiousness, a subjective measure derived from a questionnaire given to the subjects who tasted each batch of brownies. The test panel results for eight persons corresponding to each batch are given below.
	
	\begin{table}[!htbp]
	\def\arraystretch{2}
	
	\begin{center}
	\begin{tabular}{|>{\centering}m{2.5cm}||>{\centering}m{1.2cm}|>{\centering}m{1.2cm}|>{\centering}m{1.2cm}||>{\centering}m{0.65cm}|>{\centering}m{0.65cm}|>{\centering}m{0.65cm}|>{\centering}m{0.65cm}|>{\centering}m{0.65cm}|>{\centering}m{0.65cm}|>{\centering}m{0.65cm}|>{\centering\arraybackslash}m{0.65cm}|}
	
	\hline
	
	\multirow{2}{*}{Brownie Batch} & \multicolumn{3}{c||}{Treatment Combination} & \multicolumn{8}{c|}{Test Panel Result} \\
	
	\cline{2-12}
	
	& A & B & C & \textbf{1} & \textbf{2} & \textbf{3} & \textbf{4} & \textbf{5} & \textbf{6} & \textbf{7} & \textbf{8} \\
	
	\hline
	\hline
	
	1 & $-$ & $-$ & $-$ & 11 & 9 & 10 & 10 & 11 & 10 & 8 & 9 \\
	
	\hline
	
	2 & $+$ & $-$ & $-$ & 15 & 10 & 16 & 14 & 12 & 9 & 6 & 15 \\
	
	\hline
	
	3 & $-$ & $+$ & $-$ & 9 & 12 & 11 & 11 & 11 & 11 & 11 & 12	\\
	
	\hline
	
	4 & $+$ & $+$ & $-$ & 16 & 17 & 15 & 12 & 13 & 13 & 11 & 11 \\
	
	\hline
	
	5 & $-$ & $-$ & $+$ & 10 & 11 & 15 & 8 & 6 & 8 & 9 & 14 \\
	
	\hline
	
	6 & $+$ & $-$ & $+$ & 12 & 13 & 14 & 13 & 9 & 13 & 14 & 9 \\
	
	\hline
	
	7 & $-$ & $+$ & $+$ & 10 & 12 & 13 & 10 & 7 & 7 & 17 & 13 \\
	
	\hline
	
	8 & $+$ & $+$ & $+$ & 15 & 12 & 15 & 13 & 12 & 12 & 9 & 14 \\
	
	\hline

	
	\end{tabular}
	\end{center}
	
	\end{table}
	
	Analyse the data and comment on the results.
	








\newpage






%11
	\item The yield data (in kg. per plant) obtained in a factorial experiment to compare the effect of three fertilizers $N$, $P$ and $K$ each at two levels applied on a variety are given below.
	
	\begin{table}[!htbp]
	\def\arraystretch{1.35}
	
	\begin{center}
	\begin{tabular}{>{\centering}m{2cm}|>{\centering}m{1cm}|>{\centering}m{1cm}|>{\centering}m{1cm}|>{\centering\arraybackslash}m{1cm}|}
	
	\multicolumn{1}{c}{} & \multicolumn{4}{c}{Replicate $I$} \\
	
	\hhline{~----}
	
	\multirow{2}{*}{Block 1} & $npk$ & $k$ & $p$ & $n$ \\
	
	& 12.0 & 17.7 & 14.6 & 12.7 \\
	
	\hhline{~----}
	
	\multirow{2}{*}{Block 2} & $nk$ & $np$ & $pk$ & $(1)$ \\
	
	& 11.7 & 12.8 & 13.8 & 10.9 \\
	
	\hhline{~----}
	
	\end{tabular}
	\end{center}
	
	\end{table}


	
	
	
	\begin{table}[!htbp]
	\def\arraystretch{1.35}
	
	\begin{center}
	\begin{tabular}{>{\centering}m{2cm}|>{\centering}m{1cm}|>{\centering}m{1cm}|>{\centering}m{1cm}|>{\centering\arraybackslash}m{1cm}|}
	
	\multicolumn{1}{c}{} & \multicolumn{4}{c}{Replicate $II$} \\
	
	\hhline{~----}
	
	\multirow{2}{*}{Block 3} & $npk$ & $p$ & $k$ & $n$ \\
	
	& 10.3 & 8.9 & 9.3 & 10.8 \\
	
	\hhline{~----}
	
	\multirow{2}{*}{Block 4} & $np$ & $nk$ & $pk$ & $(1)$ \\
	
	& 9.3 & 9.8 & 10.0 & 12.7 \\
	
	\hhline{~----}
	
	\end{tabular}
	\end{center}
	
	\end{table}
	
	
	
	
	
	
	\begin{table}[!htbp]
	\def\arraystretch{1.35}
	
	\begin{center}
	\begin{tabular}{>{\centering}m{2cm}|>{\centering}m{1cm}|>{\centering}m{1cm}|>{\centering}m{1cm}|>{\centering\arraybackslash}m{1cm}|}
	
	\multicolumn{1}{c}{} & \multicolumn{4}{c}{Replicate $III$} \\
	
	\hhline{~----}
	
	\multirow{2}{*}{Block 5} & $nk$ & $np$ & $p$ & $k$ \\
	
	& 12.3 & 10.3 & 15.2 & 14.3 \\
	
	\hhline{~----}
	
	\multirow{2}{*}{Block 6} & $n$ & $pk$ & $(1)$ & $npk$ \\
	
	& 11.3 & 13.0 & 10.7 & 11.5 \\
	
	\hhline{~----}
	
	\end{tabular}
	\end{center}
	
	\end{table}
	
	Each replicate consists of two blocks of 4 plots each. Find the confounded effects, analyse the data and draw conclusion.

	
	
	
\end{enumerate}
\end{document}