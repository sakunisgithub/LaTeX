\documentclass[11pt, a4paper]{article}

\usepackage[top=1 in, bottom = 1 in, left = 1 in, right = 1 in ]{geometry}

\usepackage{amsmath, amssymb, amsfonts}
\usepackage{enumerate}
\usepackage{multirow}
\usepackage{hhline}
\usepackage{array}
\usepackage{longtable}
\usepackage{graphicx}

\title{\textbf{QUESTIONS}}

\author{}
\date{}

\begin{document}

\maketitle

\begin{enumerate}

%01

	\item Consider the survival data given in the following table. Compute and plot the estimated survivorship function, the probability density function, and the hazard function.
	
	\vspace{1cm}
	
	\begin{table}[!htbp]
	\def\arraystretch{3}
	
	\begin{center}
	\begin{tabular}{|>{\centering}m{4cm}|>{\centering}m{5cm}|>{\centering\arraybackslash}m{4cm}|}
	
	\hline
	
	Year of Follow-up & Number Alive at Beginning of the Interval & Number Dying in the Interval \\
	
	\hline
	
	$0-1$ & 1100 & 240 \\
	
	$1-2$ & 860 & 180 \\
	
	$2-3$ & 680 & 184 \\
	
	$3-4$ & 496 & 138 \\
	
	$4-5$ & 358 & 118 \\
	
	$5-6$ & 240 & 60 \\
	
	$6-7$ & 180 & 52 \\
	
	$7-8$ & 128 & 44 \\
	
	$8-9$ & 84 & 32 \\
	
	$\geq$ 9 & 52 & 28 \\
	
	\hline
	
	\end{tabular}
	\end{center}
	\end{table}
	
	
	
	
	
\newpage






%02
	\item The following is a life table for the total population (of 100000 live births) in the United States, 1959$-$1961. Compute the estimated survivorship function, the probability density function, and the hazard
function.

\vspace{1cm}

	\begin{table}[!htbp]
	\def\arraystretch{2}
	
	\begin{center}
	\begin{tabular}{|>{\centering}m{3cm}|>{\centering}m{5cm}|>{\centering\arraybackslash}m{4cm}|}
	
	\hline
	
	Age Interval & Number living at beginning of age interval & Number dying in age interval \\
	
	\hline
	
	$0-1$ & 100000 & 2593 \\
	
	$1-5$ & 97407 & 409 \\
	
	$5-10$ & 96998 & 233 \\
	
	$10-15$ & 96765 & 214 \\
	
	$15-20$ & 96551 & 440 \\
	
	$20-25$ & 96111 & 594 \\
	
	$25-30$ & 95517 & 612 \\
	
	$30-35$ & 94905 & 761 \\
	
	$35-40$ & 94144 & 1080 \\
	
	$40-45$ & 93064 & 1686 \\
	
	$45-50$ & 91378 & 2622 \\
	
	$50-55$ & 88756 & 4045 \\
	
	$55-60$ & 84711 & 5644 \\
	
	$60-65$ & 79067 & 7920 \\
	
	$65-70$ & 71147 & 10290 \\
	
	$70-75$ & 60857 & 12687 \\
	
	$75-80$ & 48170 & 14594 \\
	
	$80-85$ & 33576 & 15034 \\
	
	$85$ and over & 18542 & 18542 \\
	
	\hline
		
	\end{tabular}
	\end{center}
	
	\end{table}
	
	
	
	
	
	
	
	


\newpage





%03
	\item Consider a clinical trial in which 10 lung cancer patients are followed to death. The following table lists the survival time $t$ in months and number of deaths $i$. Find the estimate of survivorship function, draw its graph and comment.
	
	\begin{table}[!htbp]
	\def\arraystretch{1.5}
	
	\begin{center}
	\begin{tabular}{|>{\centering}m{0.5cm}|>{\centering}m{0.5cm}>{\centering}m{0.5cm}>{\centering}m{0.5cm}>{\centering}m{0.5cm}>{\centering}m{0.5cm}>{\centering}m{0.5cm}>{\centering}m{0.5cm}>{\centering}m{0.5cm}>{\centering\arraybackslash}m{0.5cm}|}
	
	\hline
	
	$t$ & 4 & 5 & 6 & 8 & 8 & 10 & 10 & 11 & 12 \\
	
	\hline
	
	$i$ & 1 & 2 & 3 & 4 & 5 & 6 & 7 & 8 & 9 \\
	
	\hline
	
	\end{tabular}
	\end{center}
	
	\end{table}
	
	
	

\vspace{5pt}	
	
	
	
%04
	\item Suppose 10 patients join a clinical study at the beginning of 2000. During that year, 6 patients die and 4 patients survive. At the end of the year, 20 additional patients join the study. In 2001, 3 patients who entered in the beginning of 2000 and 15 patients who entered later die, leaving one and five survivors, respectively. Suppose that the study terminates at the end of 2001. Find the estimate of survivorship function and draw its graph.
	
	
	
\vspace{5pt}
	
	
	
	
	
	
	
%05
	\item Suppose that the following remission durations are observed from 10 patients ($n$ = 10) with solid tumors. Six patients relapse at 3.0, 6.5, 6.5, 10, 12, and 15 months; 1 patient is lost to follow-up at 8.4 months; and 3 patients are still in remission at the end of the study after 4.0, 5.7, and 10 months.
	\begin{enumerate}[(i)]
		\item  Find the product limit estimate of the survival function.
		\item Find $var(\widehat{\mathrm{S}(10)})$ and estimate the standard error.
\end{enumerate}	 




\vspace{5pt}




%06
	\item Remission times for two groups of leukemia patients, one given drug 6-MP and other a placebo are as follows. \\
	
	\underline{6-MP} : 6, 6, 6, 6+, 7, 9+, 10, 10+, 12+, 13, 16, 17+, 19+, 20+, 22, 23, 25+, 32+, 32+, 34+, 35+ \\
	
	\underline{placebo} : 1, 1, 2, 2, 3, 4, 4, 5, 5, 8, 8, 8, 8, 11, 11, 12, 12, 15, 17, 22, 22 \\
	
	\begin{enumerate}[(i)]
	\item Find Kaplan-Meier product limit estimate for two groups.
	
	\item Find $var(\widehat{\mathrm{S}(16)})$ for the group given the drug 6-MP.
	
	\item Draw the survival function for the two groups.
	
	\end{enumerate}
	
	
	
	
	
	
\vspace{5pt}	
	
	
	
	
	
	
	
	
%07
	\item In order to estimate the mean burning time of a particular brand of bulb, 30 bulbs were left burning. The bulbs that failed are not replaced upon failure. The following burning time (in hours) were recorded: \\
	
	20, 27, 52, 61, 110, 122, 214, 232, 238, 371, 393, 426, 445, 472, 503, 526, 581, 627, 798, 805, 909, 976, 1001, 1016, 1033, 1086, 1192, 1322, 1681, 1723. \\
	
	
Calculate the maximum likelihood estimate and minimum variance unbiased estimate of $R(t)$ at $t = 900$ hrs.







\newpage







%08
	\item Consider an experiment with 10 electric bulbs of a particular brand. The experimenter decides to continue the study until the failure of the first 5 bulbs and then the experiment is terminated. The survival times of the first 5 bulbs are 4, 5, 8, 9 and 10 weeks. \\
	
	Assuming that the failure of these bulbs follow exponential distribution with mean $\lambda$,
		\begin{enumerate}[(i)]
		\item Identify the type of censoring and justify.
		
		\item Estimate the survival rate and mean survival time.
		
		\item Estimate the probability that a bulb will survive longer than 8 weeks.
		
		\item If the survival times for 10 bulbs are 4, 5, 8, 9, 10, 10, 10, 10, 10, 10; estimate the survival rate for this case.
		
		\end{enumerate}





\vspace{1cm}





%09
	\item The time in days of development of a tumor for rats exposed to a carcinogen follows Weibull distribution with $p = 2$ and $\lambda = 0.001$. 
	
	\begin{enumerate}[(a)]
		\item What is the probability that a rat will be tumor free at 30 days ? 45 days ? 60 days ?
		
		\item Find the hazard rate of the time to tumor appearance at 30 days, 45 days and 60 days.
		
		\item Find the median time to tumor.
	
	\end{enumerate}




\vspace{1cm}





%10
	\item Consider a small study with 8 objects. The event times were recorded as follows: 
	
	\begin{center}
	1, 3, 2+, 3+, 2, 5+, 6+, 3. \\
	\end{center}
	
	
	Here “1” means the exact event time is 1. “2+” means the subject is censored at time 2 and its event time is greater than 2. 
	
	\begin{enumerate}[(a)]
		\item Calculate the Kaplan-Meier estimates for the survival function $S(t)$.
		
		\item Make a plot of the survival function based on the estimates obtained in part (a).
		
		\item Find out the estimate of the variance of estimate of the survival function at $t = 3$.
		
		
	\end{enumerate}
		
		
	
	
	
	
	
\end{enumerate}
\end{document}