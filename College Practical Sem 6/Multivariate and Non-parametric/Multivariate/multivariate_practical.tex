\documentclass[11pt, a4paper]{article}

\usepackage[top=1 in, bottom = 1 in, left = 1 in, right = 1 in ]{geometry}

\usepackage{amsmath, amssymb, amsfonts}
\usepackage{enumerate}
\usepackage{multirow}
\usepackage{hhline}
\usepackage{array}
\usepackage{longtable}
\usepackage{graphicx}

\title{\textbf{MULTIVARIATE ANALYSIS}}

\author{}
\date{}

\begin{document}

\maketitle

\begin{enumerate}

%01

	\item The following table shows for each of 18 cinchona plants the yield of dry bark (in oz.), the height (in inches) and the girth (in inches) at a height of $6^{\prime \prime}$ from the ground. Fit a multiple linear regression model to the data.
	
	\begin{table}[!htbp]
	\def\arraystretch{2}
	
	\begin{center}
	\begin{tabular}{|>{\centering}m{2cm}|>{\centering}m{4cm}|>{\centering}m{2cm}|>{\centering\arraybackslash}m{4cm}|}
	
	\hline
	
	Plant no. & Yield of dry bark (oz.) & Height (in.) & Girth at a height of $6^{\prime \prime}$ \\
	
	\hline
	
	1 & 19 & 8 & 4 \\
	
	2 & 51 & 15 & 5 \\
	
	3 & 30 & 11 & 3 \\
	
	4 & 42 & 21 & 3 \\
	
	5 & 25 & 7 & 2 \\
	
	6 & 18 & 5 & 1 \\
	
	7 & 44 & 10 & 4 \\
	
	8 & 56 & 13 & 6 \\
	
	9 & 38 & 12 & 3 \\
	
	10 & 32 & 13 & 4 \\
	
	11 & 25 & 5 & 2 \\
	
	12 & 10 & 6 & 3 \\
	
	13 & 20 & 4 & 4 \\
	
	14 & 27 & 8 & 4 \\
	
	15 & 13 & 7 & 3 \\
	
	16 & 49 & 12 & 5 \\
	
	17 & 27 & 6 & 3 \\
	
	18 & 55 & 16 & 7 \\
	
	\hline
	
	\end{tabular}
	\end{center}
	
	\end{table}
	
	
	
	
	
	
	
	
	
	
\end{enumerate}
\end{document}