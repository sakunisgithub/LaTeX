\documentclass[11pt, a4paper]{article}

\usepackage[top=1 in, bottom = 1 in, left = 1 in, right = 1 in ]{geometry}

\usepackage{amsmath, amssymb, amsfonts}
\usepackage{enumerate}
\usepackage{multirow}
\usepackage{hhline}
\usepackage{array}
\usepackage{longtable}
\usepackage{graphicx}

\title{\textbf{NON-PARAMETRIC INFERENCE}}

\author{}
\date{}

\begin{document}

\maketitle

\begin{enumerate}

%01

	\item A manufacturer of electric bulbs claims that he has developed a new production process which will increase the mean efficiency (in suitable units) from the present value of $9.03$. The results obtained from an experiment with 15 bulbs from the new process are given as follows : 
	
	\begin{table}[!htbp]
	\def\arraystretch{1.5}
	
	\begin{center}
	\begin{tabular}{>{\centering}m{2cm}>{\centering}m{2cm}>{\centering\arraybackslash}m{2cm}}
	
	9.29 & 10.15 & 8.69 \\
	
	11.25 & 11.47 & 9.76 \\
	
	12.05 & 12.38 & 9.08 \\
	
	10.25 & 8.93 & 9.02 \\
	
	10.87 & 10.00 & 11.56 \\
	
	\end{tabular}
	\end{center}
	
	\end{table}
	
	Do we have reasons to believe that the efficiency has increased ?
	
	
	
	
	
	
	
%02
	\item 20 ear-head measurements of a variety of wheat are given as follows : 
	
	\begin{table}[!htbp]
	\def\arraystretch{1.5}
	
	\begin{center}
	\begin{tabular}{>{\centering}m{2cm}>{\centering}m{2cm}>{\centering}m{2cm}>{\centering\arraybackslash}m{2cm}}
	
	9.3 & 8.8 & 10.7 & 11.5 \\
	
	8.2 & 9.7 & 10.3 & 8.6 \\
	
	11.3 & 10.7 & 11.2 & 9.0 \\
	
	9.8 & 9.3 & 9.9 & 10.3 \\
	
	10.0 & 10.1 & 9.6 & 10.4 \\
	
	\end{tabular}
	\end{center}
	
	\end{table}
	
	
	Test at $5\%$ level of significance whether the population median length of ear-head is 9.9 cm. by using Wilcoxon signed-rank test.
	
	
	
	
	
	
	
	
	
	
	
	
	
%03
	\item The following are the marks secured by two batches of salesmen in the final test taken after completion of training. Use an appropriate non-parametric test with $\alpha = 0.02$ for the null hypothesis that the samples are drawn from identical distributions against the alternative that the distributions differ in location only. \\
	
	Batch $A$ : 26, 27, 31, 26, 19, 21, 20, 25, 30; \\
	
	Batch $B$ :  23, 28, 26, 24, 22, 19. \\
	
	
\end{enumerate}
\end{document}