\documentclass[11pt, a4paper]{article}

\usepackage[top=1 in, bottom = 1 in, left = 1 in, right = 1 in ]{geometry}

\usepackage{amsmath, amssymb, amsfonts}
\usepackage{enumerate}
\usepackage{multirow}
\usepackage{hhline}
\usepackage{array}
\usepackage{longtable}
\usepackage{graphicx}

\title{\textbf{NON-PARAMETRIC INFERENCE}}

\author{}
\date{}

\begin{document}

\maketitle

\begin{enumerate}

%01

	\item A manufacturer of electric bulbs claims that he has developed a new production process which will increase the average efficiency (in suitable units) from the present value of $9.03$. The results obtained from an experiment with 15 bulbs from the new process are given as follows : 
	
	\begin{table}[!htbp]
	\def\arraystretch{1.5}
	
	\begin{center}
	\begin{tabular}{>{\centering}m{2cm}>{\centering}m{2cm}>{\centering\arraybackslash}m{2cm}}
	
	9.29 & 10.15 & 8.69 \\
	
	11.25 & 11.47 & 9.76 \\
	
	12.05 & 12.38 & 9.08 \\
	
	10.25 & 8.93 & 9.02 \\
	
	10.87 & 10.00 & 11.56 \\
	
	\end{tabular}
	\end{center}
	
	\end{table}
	
	Do we have reasons to believe that the efficiency has increased ?
	
	
	
	
	
	
	
%02
	\item 20 ear-head measurements of a variety of wheat are given as follows : 
	
	\begin{table}[!htbp]
	\def\arraystretch{1.5}
	
	\begin{center}
	\begin{tabular}{>{\centering}m{2cm}>{\centering}m{2cm}>{\centering}m{2cm}>{\centering\arraybackslash}m{2cm}}
	
	9.3 & 8.8 & 10.7 & 11.5 \\
	
	8.2 & 9.7 & 10.3 & 8.6 \\
	
	11.3 & 10.7 & 11.2 & 9.0 \\
	
	9.8 & 9.3 & 9.9 & 10.3 \\
	
	10.0 & 10.1 & 9.6 & 10.4 \\
	
	\end{tabular}
	\end{center}
	
	\end{table}
	
	
	Test at $5\%$ level of significance whether the population median length of ear-head is 9.9 cm. by using Wilcoxon signed-rank test.
	
	
	
	
	
	
	
	
	
	
	
	
	
%03
	\item The following are the marks secured by two batches of salesmen in the final test taken after completion of training. Use an appropriate non-parametric test with $\alpha = 0.02$ for the null hypothesis that the samples are drawn from identical distributions against the alternative that the distributions differ in location only. \\
	
	Batch $A$ : 26, 27, 31, 26, 19, 21, 20, 25, 30; \\
	
	Batch $B$ :  23, 28, 26, 24, 22, 19. \\
	
	
	
	
	
	
	
	
	
	
	
	
	
\newpage














%04
	\item Given below are the marks obtained by a group of 20 students in a subject in a college test and in the subsequent public examination. Test at 1\% level whether the group has improved its average performance from the college test to the public examination, by using
	\begin{enumerate}[(i)]
	\item the sign test
	\item the Wilcoxon signed-rank test	
	\end{enumerate}
	
	\begin{table}[!htbp]
	\def\arraystretch{1.95}
	
	\begin{center}
	\begin{tabular}{|>{\centering}m{2cm}||>{\centering}m{4.5cm}|>{\centering\arraybackslash}m{4.5cm}|}
	
	\hline
	
	\multirow{2}{*}{Serial No.} & \multicolumn{2}{c|}{Marks Obtained in} \\
	
	\hhline{~--}
	
	& College Test & Public Examination \\
	
	\hline
	
	1 & 183 & 133 \\
	
	2 & 175 & 193 \\
	
	3 & 134 & 170 \\
	
	4 & 170 & 164 \\
	
	5 & 183 & 199 \\
	
	6 & 167 & 160 \\
	
	7 & 120 & 168 \\
	
	8 & 175 & 158 \\
	
	9 & 126 & 162 \\
	
	10 & 187 & 176 \\
	
	11 & 123 & 126 \\
	
	12 & 121 & 141 \\
	
	13 & 175 & 103 \\
	
	14 & 133 & 126 \\
	
	15 & 144 & 146 \\
	
	16 & 109 & 155 \\
	
	17 & 165 & 162 \\
	
	18 & 144 & 161 \\
	
	19 & 164 & 182 \\
	
	20 & 125 & 119 \\
	
	\hline
	
	\end{tabular}
	\end{center}
	
	\end{table}
	
	
	
	
	
	
	
	
	
	
	
\newpage










%05
	\item Scores on a clerical aptitude test administered to a batch of 6 Secretariat and 7 Directorate clerks are given below. Test whether the two groups of clerks have the same score distribution in the population.
	
	\begin{table}[!htbp]
	\def\arraystretch{1.5}
	
	\begin{center}
	\begin{tabular}{|>{\centering}m{5cm}|>{\centering}m{1cm}>{\centering}m{1cm}>{\centering}m{1cm}>{\centering}m{1cm}>{\centering}m{1cm}>{\centering}m{1cm}>{\centering\arraybackslash}m{1cm}|}
	
	\hline
	
	Scores of Secretariat clerks & 40 & 35 & 52 & 60 & 46 & 55 & \\
	
	\hline
	
	Scores of Directorate clerks & 47 & 56 & 42 & 57 & 50 & 57 & 62 \\
	
	\hline
	
	
	
	\end{tabular}
	\end{center}
	
	\end{table}
	
	
	
	
	
	
	
	
	
	
%06
	\item Consider two samples as follows : \\
	
	$\textbf{X} = (1, 5, 7, 9, 15, 17, 21, 23)$ \\
	
	$\textbf{Y} = (2, 6, 10, 12, 18, 20, 26, 28, 32)$. \\
	
	Test whether they have homogenous population distribution.
	
	
	
	
	





%07
	\item Ten points are taken in an interval of length one meter. The distance of each point from the start of the interval is (in meters) as follows :
	
	\begin{table}[!htbp]
	\def\arraystretch{1.5}
	
	\begin{center}
	
	\begin{tabular}{ccccc}
	0.414 & 0.523 & 0.229 & 0.942 & 0.097 \\
	
	0.394 & 0.572 & 0.486 & 0.273 & 0.358 
	\end{tabular}
	\end{center}
	\end{table}
	
	Test whether the sample can be considered as a sample from $U(0, 1)$ distribution.
	
	
	
	
	
	
	
	
	
	
	
	
	
%08
	\item The following is a random sample of size 20. Test whether the sample can be considered as a sample from $N(0, 1)$ distribution.
	
	\begin{table}[!htbp]
	\def\arraystretch{1.5}
	
	\begin{center}
	\begin{tabular}{ccccc}
	
	2.240 & $-0.400$ & $-1.152$ & 0.980 & 0.361 \\
	
	$-0.123$ & $-0.625$ & 0.682 & 2.323 & $-1.053$ \\
	
	$-0.870$ & $-0.164$ & $-0.340$ & $-0.041$ & 1.405 \\
	
	1.187 & 0.323 & 0.270 & $-0.128$ & 0.101
	
	\end{tabular}
	\end{center}
	
	\end{table}
	
	
	
	
	
	
	
	
	
	



%09	
	\item Thirty observations as given below are obtained : 
	
	\begin{center}
	
	24, 35, 12, 50, 60, 70, 68, 49, 80, 25, 69, 28, 28, 11, 83, \\ 31, 37, 34, 54, 75, 45, 95, 75, 26, 43, 57, 94, 48, 63, 45 
	\end{center}
	
	Test their randomness by considering the sequence of positive and negative signs.
	
	
	
	
	
	
	
	
	
	
	
	
\newpage








%10
	\item Fifteen 3-year-old boys and fifteen 3-year-old girls were observed during two sessions of recess in a nursery school. Each child’s play was scored for incidence and degree of aggression as follows:
	
	\begin{table}[!htbp]
	\def\arraystretch{1.5}
	
	\begin{center}
	\begin{tabular}{>{\centering}m{2cm}ccccccccccccccc}
	
	Boys: & 96 & 65 & 74 & 78 & 82 & 121 & 68 & 79 & 111 & 48 & 53 & 92 & 81 & 31 & 40 \\
	
	Girls: & 12 & 47 & 32 & 59 & 83 & 14 & 32 & 15 & 17 & 82 & 21 & 34 & 9 & 15 & 51 \\
	
	\end{tabular}
	\end{center}
	
	\end{table}
	
	
	Is there evidence to suggest that there are sex differences in the incidence and amount of aggression? Use both Mann–Whitney and run tests.
	
	
	
	
	
	
	
	
	
	
	
	
	
%11
	\item Some depressed people were found, and it was checked that initially they were all equivalently depressed. Then each person was allocated randomly to one of three groups: no exercise; 20 minutes of jogging per day; or 60 minutes of jogging per day. At the end of a month, each participant was asked to rate how depressed they now feel on a Likert scale that runs from 1 (totally miserable) through to 100 (ecstatically happy). \\
	
	\textbf{Rating on depression scale:} \\
	
	\begin{table}[!htbp]
	\def\arraystretch{2}
	
	\begin{center}
	\begin{tabular}{>{\centering}m{3.5cm}|>{\centering}m{3.5cm}|>{\centering}m{3.5cm}|>{\centering\arraybackslash}m{3.5cm}|}
	
	\cline{2-4}
	
	& \textbf{No Exercise} & \textbf{Jogging for 20 minutes} & \textbf{Jogging for 60 minutes} \\

	\cline{2-4}
		
	& 23 & 22 & 59 \\
	
	\cline{2-4}
		
	& 26 & 27 & 66 \\
	
	\cline{2-4}
	
	& 51 & 39 & 38 \\
	
	\cline{2-4}
	
	& 49 & 29 & 49 \\
	
	\cline{2-4}

	& 58 & 46 & 56 \\
	
	\cline{2-4}

	& 37 & 48 & 60 \\
	
	\cline{2-4}

	& 29 & 49 & 56 \\
	
	\cline{2-4}

	& 44 & 65 & 62 \\
	
	\cline{2-4}
	
	\textbf{Mean} $\rightarrow$ & 39.63 & 40.63 & 55.75 \\
	
	\textbf{SD} $\rightarrow$ & 12.85 & 14.23 & 8.73 \\
	
	\cline{2-4}
	
	\end{tabular}
	\end{center}
	
	\end{table}
	
	
	\begin{enumerate}[(a)]
		\item Does physical exercise alleviate depression?
		
		\item Is there a difference in the ratings of the participants allocated to no exercise group as compared to those allocated to 20 minutes of jogging group? 
	
	\end{enumerate}
	
	
	
	
	
	
	
	
	
	
	
	
	
\newpage











%12
	\item A medical researcher wishes to determine if a pill has the undesireable side effect of reducing the blood pressure of the user. The study involves recording the blood pressures of 15 women. After they use the pill regularly for six months, their blood pressures are again recorded. The observations are given in the following table.
	
	\begin{table}[!htbp]
	\def\arraystretch{2}
	
	\begin{center}
	\begin{tabular}{|>{\centering}m{2cm}|>{\centering}m{2cm}|>{\centering\arraybackslash}m{2cm}|}
	
	\hline
	
	\textbf{Subject} & \textbf{Before} & \textbf{After} \\
	
	\hline	
	\hline
	
	1 & 70 & 68 \\
	
	\hline
	
	2 & 80 & 72 \\
	
	\hline
	
	3 & 72 & 62 \\
	
	\hline
	
	4 & 76 & 70 \\
	
	\hline
	
	5 & 76 & 58 \\
	
	\hline
	
	6 & 76 & 66 \\
	
	\hline
	
	7 & 72 & 68 \\
	
	\hline
	
	8 & 78 & 52 \\
	
	\hline
	
	9 & 82 & 64 \\
	
	\hline
	
	10 & 64 & 72 \\
	
	\hline
	
	11 & 74 & 74 \\
	
	\hline
	
	12 & 92 & 60 \\
	
	\hline
	
	13 & 74 & 74 \\
	
	\hline
	
	14 & 68 & 72 \\
	
	\hline
	
	15 & 84 & 74 \\
	
	\hline
	
	\end{tabular}
	\end{center}
	
	\end{table}
	
	
	Carry out a suitable non-parametric test to determine if blood pressure has decreased after taking the pill.
	
	
\end{enumerate}
\end{document}