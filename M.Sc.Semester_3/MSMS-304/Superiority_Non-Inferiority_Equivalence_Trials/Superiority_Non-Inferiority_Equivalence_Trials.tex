\documentclass[11pt, a4paper]{article}

\usepackage[top = 1 in, bottom = 1 in, left = 1 in, right = 1 in]{geometry}

\usepackage{amsmath, amssymb, amsfonts}
\usepackage{enumerate}
\usepackage{multirow}
\usepackage{hhline}
\usepackage{array}
\usepackage{longtable}
\usepackage{graphicx}
\usepackage{tabularray}
\usepackage{undertilde}
\usepackage{dingbat}
\usepackage{fontawesome5}
\usepackage[colorlinks=true, linkcolor=blue, urlcolor=red]{hyperref}
\usepackage{tasks}
\usepackage{bbding}
\usepackage{twemojis}
% how to use bull's eye ----- \scalebox{2.0}{\twemoji{bullseye}}
\usepackage{customdice}
% how to put dice face ------ \dice{2}

\title{MSMS 304 - Biostatistics}
\author{Ananda Biswas}
\date{31.07.2025}


\begin{document}

\maketitle

$\bullet$ \textbf{Superiority, Equivalence, Non-Inferiority Trials} : Suppose the control has a cure rate of $x\%$. The new intervention is said to be superior than the current control if the cure rate of the new intervention is greater than or equal to $(x + \delta) \%$ where $\delta > 0$ is a predefined margin called \underline{superiority margin}. Many clinical trials are designed to demonstrate that a new intervention is superior to some established intervention. Such trials are called Superiority trials.

However, not all trials share this goal. New interventions may have no superiority to existing therapy, but as long as they are not materially worse, maybe of interest because they are less toxic, bear less cost, require fewer doses, or have some other value to patients. In trials where we test whether the new intervention is no worse than or at least as good as, some established intervention, are called Non-inferiority trials.

Moreover, in some trials the goal is not to determine whether the new intervention is better or worse, rather we want to test whether it is similar enough in effectiveness to be used interchangeably with the current intervention. Such trials are called Equivalence trials.


\end{document}