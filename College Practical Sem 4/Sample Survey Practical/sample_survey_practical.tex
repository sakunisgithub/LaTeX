\documentclass[11pt, a4paper]{article}

\usepackage[top=1 in, bottom = 1 in, left = 1 in, right = 1 in ]{geometry}

\usepackage{amsmath, amssymb, amsfonts}
\usepackage{enumerate}
\usepackage{multirow}
\usepackage{hhline}
\usepackage{array}
\usepackage{longtable}

\title{\textbf{QUESTIONS}}

\author{}
\date{}

\begin{document}

\maketitle


\begin{enumerate}




	\item Signatures to a petition were collected on 700 sheets. Each sheet was provided with space for 50 signatures, but the signatories put their signatures in erratic ways and the number of signatures per sheet was not definite. 12 sheets were spoiled in transit. Of the available sheets, a random sample of 50 was drawn and the numbers of signatures per sheet counted, which are shown as below :
	
	
	\begin{table}[h]
	\def\arraystretch{1.5}
	
	\begin{center}
	\begin{tabular}{|c||c|c|c|c|c|c|c|c|c|c|c|c|c|}
	
	\hline
	
	Number of signatures $(y_i)$ & 52 & 51 & 46 & 42 & 40 & 37 & 32 & 29 & 27 & 15 & 14 & 10 & 8 \\
	
	\hline
	
	Number of sheets $(n_i)$ & 1 & 2 & 21 & 8 & 7 & 2 & 2 & 1 & 1 & 2 & 1 & 1 & 1 \\
	
	\hline
	
	\end{tabular}
	\end{center}
	
	\end{table}
	
	
	Estimate the total number of signatures to the petition and calculate $95\%$ confidence limits.
	
	
	
	
	
	
	
	
	
	
	
	
	
	
	
	
	
	
	
	\item The data given below to pertain to one complete lactation of milk yield (in 10 kg) of 250 cows in an organised dairy farm.
	\begin{enumerate}[(i)]
		\item Select a simple random sample of size 25.
		\item Estimate the mean with its standard error.
		\item Construct a $95\%$ confidence limit for the population mean.
	
	\end{enumerate}
	
	

	
	\begin{center}
	
	\def\arraystretch{1.5}
	\setlength{\tabcolsep}{12pt}
	
	\begin{longtable}{cccccccccc}
	
	\hline
	
	230 & 293 & 163 & 290 & 200& 173 & 194 & 322 & 169 & 230 \\
	
	297 & 151 & 248 & 271 & 259 & 214 & 167 & 207 & 240 & 286 \\
	
	184 & 248 & 327 & 338 & 165 & 177 & 270 & 177 & 202 & 
155 \\

	155 & 293 & 190 & 172 & 150 & 319 & 151 & 118 & 213 &  114 \\
	
	186 & 167 & 129 & 185 & 231 & 199 & 265 & 306 & 173 & 
276 \\

	291 & 231 & 205 & 220 & 246 & 239 & 186 & 299 & 233 &
208 \\

	265 & 204 & 300 & 195 & 239 & 173 & 237 & 282 & 221 & 218 \\

	197 & 215 & 213 & 290 & 146 & 232 & 305 & 184 & 149 & 
267 \\

	188 & 219 & 171 & 99 & 329 & 199 & 180 & 225 & 257 & 202 \\
	
	189 & 207 & 792 & 327 & 201 & 300 & 206 & 199 & 299 & 
153 \\

	175 & 287 & 277 & 230 & 258 & 137 & 174 & 301 & 260 & 282 \\

	211 & 212 & 284 & 214 & 283 & 139 & 223 & 212 & 207 & 
224 \\

	207 & 111 & 272 & 192 & 127 & 303 & 221 & 187 & 309 & 
263 \\

	203 & 176 & 233 & 239 & 176 & 218 & 193 & 243 & 236 &
275 \\

	288 & 198 & 241 & 219 & 167 & 193 & 234 & 179 & 126 &  173 \\
	
	279 & 178 & 275 & 260 & 191 & 174 & 235 & 338 & 242 & 238 \\
	
	211 & 187 & 184 & 189 & 305 & 221 & 253 & 225 & 327 & 
203 \\

	195 &	158 & 156 & 185 & 170 & 271 & 160 & 188 & 165 & 
218 \\

	312 & 143 & 267 & 298 & 196 & 139 & 205 & 298 & 238 & 
217 \\

	145 &	201 & 313 & 230 & 185 & 166 & 147 & 223 & 271 & 133 \\
	
	155 & 230 & 287 & 329 & 265 & 150 & 286 & 271 & 268 & 
198 \\

	214 & 231 & 163 & 335 & 198 & 270 & 187 & 174 & 163 &
201 \\

	192 & 247 & 247 & 297 & 178 & 240 & 290 & 234 & 170 & 
227 \\

	230 & 353 & 170 & 159 & 236 & 181 & 230 & 240 & 212 & 242 \\
	
	151 & 158 & 253 & 179 & 263 & 158 & 250 & 226 & 246 & 301 \\
	
	\hline
	
	\end{longtable}
	\end{center}
	
	
	
	
	
	
	
	
	
	
	
	
	
	
	
	
	
	
	
	
	
	
	
	
	
	
	\item 2000 cultivators' holdings in Uttar Pradesh (India) were stratified according to their sizes. The number of holdings $(N_i)$, mean area under wheat per holdings $(\overline{Y_i})$ and s.d. of area under wheat per holdings $(S_i)$ are given below for each stratum :
	
	\begin{table}[h]
	\def\arraystretch{1.5}
	
	\begin{center}
	\begin{tabular}{|>{\centering}m{2cm}|>{\centering}m{2cm}|>{\centering}m{4cm}|>{\centering\arraybackslash}m{4cm}|}
	
		\hline
		
		Stratum Number & Number of holdings & Mean area under wheat per-holding & s.d. of area under wheat per-holding \\
		
		& $(N_i)$ & $(\overline{Y_i})$ & $(S_i)$ \\
		
		\hline
		
		1 & 394 & 5.4 & 8.3 \\
		
		2 & 461 & 16.3 & 13.3 \\
		
		3 & 381 & 24.3 & 15.1 \\
		
		4 & 334 & 34.5 & 19.8 \\
		
		5 & 169 & 42.1 & 24.5 \\
		
		6 & 113 & 50.1 & 26.0 \\
		
		7 & 148 & 63.8 & 35.2 \\
		
		\hline
		
	\end{tabular}
	\end{center}
	\end{table}
	
	For a sample of 200 farms, compute the sample size in each stratum under proportional and optimum allocations. \\ 		Calculate the sampling variance of the estimated area under wheat from the sample 
		\begin{enumerate}[(i)]
			\item if the farms are selected under proportional allocation by with and without replacement methods
			\item if the farms are selected under Neyman's allocation by with and without replacement methods.
		\end{enumerate}
		
	Also compute the gain in efficiency from these procedures as compared to simple random sampling.
	
	
	
	
	
	
	
	
	
	
	
	
	
	
	
	
	
	
	
	
	
	
	
	
	
	
	
	
	
	\item The number of pepper standards for selected villages in each of the three strata of Trivandrum zone is as follows : \\
	
	
	
	
	\begin{table}[h]
	\def\arraystretch{1.5}
	
	\begin{center}
	\begin{tabular}{|>{\centering}m{2cm}|>{\centering}m{4cm}|>{\centering}m{4cm}|>{\centering\arraybackslash}m{4cm}|}
	
	\hline
	
	Stratum & Total number of villages in the stratum & Number of villages selected from the stratum & Number of pepper standars in each of the selected villages \\
	
	\hline
	
	1 & 441 & 11 & 41, 116, 19, 15, 144, 159, 212, 57, 28, 119, 76 \\
	
	2 & 405 & 12 & 39, 70, 38, 37, 161, 38, 27, 119, 36, 128, 30, 208 \\
	
	3 & 103 & 7 & 252, 385, 192, 296, 115, 159, 120 \\
	
	\hline
	
	\end{tabular}
	\end{center}
	\end{table}
	
	
	Estimate the total number of pepper standards along with its standard error in Trivandrum zone. Also, estimate the gain in precision due to stratification.
	
	
	
	
	
	
	
	
	
	
	
	
	
	
	
	
	
	
	
	
	
	
	
	
	
	
	
	
	
	
	
	
	
	
	
	
	\item A random sample of $ n = 2 $ households was drawn from a small colony of 5 households (hypothetical population) having monthly income (in rupees) as follows :
	
	\begin{table}[h]
	\def\arraystretch{1.5}
	
	\begin{center}
	\begin{tabular}{|c||c|c|c|c|c|}
	
	\hline
	
	Household & 1 & 2 & 3 & 4 & 5 \\
	
	\hline
	
	Income (in rupees) & 156 & 149 & 166 & 164 & 155 \\
	
	\hline
	
	\end{tabular}
	\end{center}
	
	\end{table}
	
	\begin{enumerate}[(i)]
	
	\item Calculate population mean $(\overline{Y})$, variance $(\sigma^2)$ and mean square $(S^2)$.
	
	\item Enumerate all possible samples of size 2 by replacement method and show that
	
		\begin{enumerate}[(a)]
		\item the sample mean gives an unbiased estimate of the population mean and find its sampling variance;
		
		\item the sample variance $(s^2)$ is an unbiased estimate of the population variance $(\sigma^2)$; and 
		
		\item $v(\overline{y}) = \dfrac{(y_1 - y_2)^2}{4}$ is an unbiased estimator of $V(\overline{Y})$, $i.e.$ $E(v(\overline{y})) = V(\overline{Y}) = \dfrac{\sigma^2}{2}.$
		
		\end{enumerate}
	
	
	\item Enumerate all possible samples of size 2 by without replacement method and show that
	
		\begin{enumerate}[(a)]
		\item the sample mean gives an unbiased estimate of the population mean and find its sampling variance;
		
		\item the sample variance $(s^2)$ is an unbiased estimate of the population mean square $(S^2)$; and 
		
		\item $v(\overline{y}) = \dfrac{3(y_1 - y_2)^2}{20}$ is an unbiased estimator of $V(\overline{Y})$, $i.e.$ $$E(v(\overline{y})) = V(\overline{Y}) = \left( \dfrac{1}{2} - \dfrac{1}{5} \right)S^2 = \dfrac{3}{10}S^2.$$
		
		\end{enumerate}
	
	\end{enumerate}
	
	
	
	
	
	
	
	
	
	
	
	
	
	
	
	
	\pagebreak
	
	
	
	
	
	
	
	
	
	
	
	
	
	
	
	
	
	
	\item An investigator desires to take a stratified random sample with the following assumptions:
	
	\begin{table}[h]
	\def\arraystretch{1.5}
	
	\begin{center}
	\begin{tabular}{|c|c|c|c|}
	
	\hline
	
	Stratum & $S_i$ & $N_i$ & $C_i$ (in Rs.) \\
	
	\hline
	
	1 & 400 & 10 & 4 \\
	
	\hline
	
	2 & 600 & 20 & 9 \\
	
	\hline
	
	\end{tabular}
	\end{center}
	
	\end{table}
	
	\begin{enumerate}[(i)]
	\item Estimate the values of $\dfrac{n_1}{n}$ and $\dfrac{n_2}{n}$ which minimize the total field cost $C = c_1 n_1 + c_2 n_2$  for a given value of $V(\overline{y}_{st})$.
	
	\item Estimate the total sample size required, under the scheme of optimum allocation, to make $V(\overline{y}_{st}) = 1$, when fpc is ignored.
	
	
	\item Also estimate the cost of the survey.
	\end{enumerate}
	
	
	
	
	
	
	
	
	
	
	
	
	
	
	
	
	
	
	
	
	
	
	
	
	
	
	
	
	
	
	
	
	
	
	
	
	
	
	\item Given below are the daily milk yield (in liters) records of the first lactation of a specified cow belonging to the Tharparkar herd maintained at the Government Cattle Farm, Patna. The milk yields of the first five days were not recorded, being the colostrum period.
	
	\begin{center}
	
	\def\arraystretch{1.5}
	\setlength{\tabcolsep}{12pt}
	
	\begin{longtable}{|c|cccccccccc|}
	
	\hline
	
	Day & 1 & 2 & 3 & 4 & 5 & 6 & 7 & 8 & 9 & 10 \\
	
	Milk Yield & 10 & 11 & 14 & 10 & 14 & 9 & 10 & 8 & 11 & 10 \\
	
	\hline
	
	Day & 11 & 12 & 13 & 14 & 15 & 16 & 17 & 18 & 19 & 20 \\
	
	Milk Yield & 6 & 9 & 8 & 7 & 9 & 10 & 11 & 11 & 13 & 12 \\
	
	\hline
	
	Day & 21 & 22 & 23 & 24 & 25 & 26 & 27 & 28 & 29 & 30 \\
	
	Milk Yield & 12 & 10 & 11 & 11 & 14 & 15 & 12 & 17 & 18 & 16 \\
	
	\hline
	
	Day & 31 & 32 & 33 & 34 & 35 & 36 & 37 & 38 & 39 & 40 \\
	
	Milk Yield & 13 & 14 & 14 & 15 & 16 & 16 & 16 & 13 & 16 & 17 \\
	
	\hline
	
	Day & 41 & 42 & 43 & 44 & 45 & 46 & 47 & 48 & 49 & 50 \\
	
	Milk Yield & 14 & 16 & 15 & 14 & 14 & 15 & 17 & 15 & 16 & 17 \\
	
	\hline
	
	Day & 51 & 52 & 53 & 54 & 55 & 56 & 57 & 58 & 59 & 60 \\
	
	Milk Yield & 25 & 22 & 23 & 19 & 18 & 16 & 22 & 21 & 21 & 23 \\
	
	\hline
	
	Day & 61 & 62 & 63 & 64 & 65 & 66 & 67 & 68 & 69 & 70 \\
	
	Milk Yield & 21 & 19 & 19 & 19 & 19 & 19 & 19 & 19 & 19 & 19 \\
	
	\hline
	
	Day & 71 & 72 & 73 & 74 & 75 & 76 & 77 & 78 & 79 & 80 \\
	
	Milk Yield & 18 & 19 & 21 & 20 & 17 & 16 & 18 & 18 & 18 & 22 \\
	
	\hline
	
	Day & 81 & 82 & 83 & 84 & 85 & 86 & 87 & 88 & 89 & 90 \\
	
	Milk Yield & 22 & 22 & 20 & 20 & 20 & 18 & 20 & 21 & 21 & 20 \\
	
	\hline
	
	\pagebreak
	
	\hline
	
	Day & 91 & 92 & 93 & 94 & 95 & 96 & 97 & 98 & 99 & 100 \\
	
	Milk Yield & 18 & 21 & 22 & 22 & 20 & 21 & 21 & 21 & 21 & 21 \\
	
	\hline
	
	Day & 101 & 102 & 103 & 104 & 105 & 106 & 107 & 108 & 109 & 110 \\
	
	Milk Yield & 19 & 20 & 21 & 20 & 21 & 20 & 21 & 20 & 21 & 20 \\
	
	\hline
	
	Day & 111 & 112 & 113 & 114 & 115 & 116 & 117 & 118 & 119 & 120 \\
	
	Milk Yield & 19 & 21 & 18 & 21 & 20 & 22 & 21 & 21 & 21 & 16 \\
	
	\hline
	
	Day & 121 & 122 & 123 & 124 & 125 & 126 & 127 & 128 & 129 & 130 \\
	
	Milk Yield & 19 & 15 & 15 & 16 & 19 & 12 & 16 & 14 & 15 & 17 \\
	
	\hline
	
	Day & 131 & 132 & 133 & 134 & 135 & 136 & 137 & 138 & 139 & 140 \\
	
	Milk Yield & 16 & 20 & 15 & 19 & 16 & 16 & 20 & 20 & 18 & 21 \\
	
	\hline
	
	Day & 141 & 142 & 143 & 144 & 145 & 146 & 147 & 148 & 149 & 150 \\
	
	Milk Yield & 22 & 22 & 21 & 22 & 21 & 21 & 21 & 18 & 20 & 17 \\
	
	\hline
	
	Day & 151 & 152 & 153 & 154 & 155 & 156 & 157 & 158 & 159 & 160 \\
	
	Milk Yield & 20 & 20 & 21 & 21 & 21 & 20 & 20 & 16 & 16 & 15 \\
	
	\hline
	 
	Day & 161 & 162 & 163 & 164 & 165 & 166 & 167 & 168 & 169 & 170 \\
	
	Milk Yield & 18 & 19 & 18 & 20 & 19 & 18 & 16 & 14 & 14 & 13 \\
	
	\hline
	
	Day & 171 & 172 & 173 & 174 & 175 & 176 & 177 & 178 & 179 & 180 \\
	
	Milk Yield & 16 & 16 & 16 & 18 & 16 & 15 & 16 & 18 & 18 & 15 \\
	
	\hline
	
	Day & 181 & 182 & 183 & 184 & 185 & 186 & 187 & 188 & 189 & 190 \\
	
	Milk Yield & 18 & 16 & 17 & 18 & 16 & 17 & 13 & 14 & 13 & 12 \\
	
	\hline
	
	Day & 191 & 192 & 193 & 194 & 195 & 196 & 197 & 198 & 199 & 200 \\
	
	Milk Yield & 16 & 10 & 13 & 8 & 8 & 6 & 8 & 9 & 4 & 5 \\
	
	\hline
	
	Day & 201 & 202 & 203 & & & & & & &\\
	
	Milk Yield & 6 & 6 & 4 & & & & & & & \\
	
	\hline
	
	
	
	\end{longtable}
	\end{center}
	
	
	
	Find the efficiency of systematic sampling at 7 and 14 days' interval of recording, with respect to corresponding simple random sampling, in estimating the lactation yield of the cow.
	
	
	
	
	
	
	
	
	
	
	
	
	
	
	
	
	
	
	
	
	
	
	
	
	
	
	
	
	
	
	\item In an experimental agricultural census carried out by I.A.S.R.I., New Delhi, in the Loni Block of Meerut District of U.P. (India) during 1967-68, two villages of the block were selected randomly. Out of 225 holding in two villages, namely Panch-lok and Agrola, 45 holdings were selected for systematic sampling (with 5 as the sampling interval). The total arable land (in kacha bigha)* for the 45 selected holdings are given below :
	
	\pagebreak
	
	\begin{table}[h]
	\def\arraystretch{1.5}
	
	\begin{center}
	\begin{tabular}{|c|cccccccccc|}
	
	\hline
	
	S.N. & 1 & 2 & 3 & 4 & 5 & 6 & 7 & 8 & 9 & 10 \\
	
	Total arable land & 60 & 50 & 14 & 10 & 1 & 0 & 0 & 0 & 0 & 0 \\
	
	\hline
	
	S.N. & 11 & 12 & 13 & 14 & 15 & 16 & 17 & 18 & 19 & 20 \\
	
	Total arable land & 150 & 150 & 100 & 20 & 0 & 25 & 192 & 25 & 0 & 13 \\
	
	\hline
	
	S.N. & 21 & 22 & 23 & 24 & 25 & 26 & 27 & 28 & 29 & 30 \\
	
	Total arable land & 0 & 0 & 50 & 0 & 10 & 0 & 0 & 0 & 85 & 30 \\
	
	\hline
	
	S.N. & 31 & 32 & 33 & 34 & 35 & 36 & 37 & 38 & 39 & 40 \\
	
	Total arable land & 30 & 70 & 30 & 35 & 0 & 30 & 0 & 0 & 10 & 0 \\
	
	\hline
	
	S.N. & 41 & 42 & 43 & 44 & 45 & & & & & \\
	
	Total arable land & 20 & 70 & 16 & 15 & 35 & & & & & \\
	\hline
	
	\end{tabular}
	\end{center}
	\end{table}
	
	
	Estimate the total arable land in the two villages and also the approximate standarad error of the estimate.
	
	
	
	
	
	
	
	
	
	
	
	
	
	
	
	
	
	
	
	
	
	
	
	
	
	
	
	
	
	
	
	
	
	\item For studying milk yield, feeding and management practices of milch animals in the year 1977-78, the whole Haryana state was divided into 4 zones according to agro-climatic conditions. The total number of milch animals in 17 randomly selected villages (in 1977-78) of zone A, along with their livestock census data in 1976, are as shown below :
	
	
	\begin{center}
	\def\arraystretch{1.5}
	\setlength{\tabcolsep}{10pt}
	
	\begin{longtable}{|c|>{\centering}m{4cm}|>{\centering\arraybackslash}m{4cm}|}
	
	\hline
	
	Serial No. of Village & No. of Milch Animals in Survey (y) & No. of Milch Animals in Census (x) \\
	
	\hline
	
	1 & 1129 & 1141 \\
	
	\hline
	
	2 & 1144 & 1144 \\
	
	\hline
	
	3 & 1125 & 1127 \\
	
	\hline
	
	4 & 1138 & 1153 \\
	
	\hline
	
	5 & 1137 & 1117 \\
	
	\hline
	
	6 & 1127 & 1140 \\
	
	\hline
	
	7 & 1163 & 1153 \\
	
	\hline
	
	8 & 1153 & 1146 \\
	
	\hline
	
	9 & 1164 & 1189 \\
	
	\hline
	
	10 & 1130 & 1137 \\
	
	\hline
	
	11 & 1153 & 1170 \\
	
	\hline
	
	12 & 1125 & 1115 \\
	
	\hline
	
	13 & 1116 & 1130 \\
	
	\hline
	
	\pagebreak
	
	\hline
	
	14 & 1115 & 1118 \\
	
	\hline
	
	15 & 1112 & 1122 \\
	
	\hline
	
	16 & 1112 & 1113 \\
	
	\hline
	
	17 & 1123 & 1166 \\
	
	\hline
	
	
	
	\end{longtable}
	\end{center}
	

	Estimate the total number of milch animals in 117 villages of zone A
	\begin{enumerate}[(i)]
		\item by ratio method and
		\item by simple mean per unit method
	
	\end{enumerate}
	
	\setlength{\parindent}{45pt} Also compare its precision, given the total number of milch animals in the \\ census = 143968. 
	
	
	
	
	
	
	
	
	
	
	
	
	
	
	
	
	
	
	
	
	
	
	
	
	
	
	
	
	\item The number of labourers $x$ (in thousands) and the quantity of raw materials $y$ (in lakhs of bales) are given below for 20 jute mills. Draw a sample of 5 units by SRSWOR. Estimate the total amount of raw materials consumed by 20 mills by 
	
	\begin{enumerate}[(i)]
	\item sample mean estimator, and
	
	\item ratio estimator along with their variance estimators.
	
	\end{enumerate}
	
	\setlength{\parindent}{0pt}Also, compare the variance of these estimators.
	
	\begin{table}[h]
	\def\arraystretch{1.5}
	
	\begin{center}
	\begin{tabular}{|>{\centering}m{2cm}|>{\centering}m{1.5cm}|>{\centering}m{1.5cm}||>{\centering}m{2cm}|>{\centering}m{1.5cm}|>{\centering\arraybackslash}m{1.5cm}|}
	
	\hline
	
	Jute Mill & $x$ & $y$ & Jute Mill & $x$ & $y$ \\
	
	\hline
	\hline
	
	1 & 368 & 31 & 11 & 512 & 31 \\
	
	\hline
	
	2 & 384 & 33 & 12 & 503 & 29 \\
	
	\hline
	
	3 & 361 & 37 & 13 & 472 & 38 \\
	
	\hline
	
	4 & 347 & 39 & 14 & 429 & 41 \\
	
	\hline
	
	5 & 403 & 43 & 15 & 387 & 40 \\
	
	\hline
	
	6 & 529 & 61 & 16 & 376 & 38 \\
	
	\hline
	
	7 & 703 & 68 & 17 & 412 & 42 \\
	
	\hline
	
	8 & 396 & 42 & 18 & 345 & 45 \\
	
	\hline
	
	9 & 473 & 41 & 19 & 297 & 32 \\
	
	\hline
	
	10 & 509 & 49 & 20 & 633 & 50 \\
	
	\hline
	
	
	
	\end{tabular}
	\end{center}
	
	\end{table}
	
	
	
	
	
	
	
	
\pagebreak




















\item The population in 1975 $(x)$ and the population in 1985 $(y)$ for 22 municipal town areas divided into two zones in Sweden are given in the following table. 

	\setlength{\parindent}{25pt} Draw a simple random sample of size 4 from each of the two zones and obtain a seperate ratio estimator and a combined ratio estimator of the total population in 1985 for the whole area taking $(x)$ as an auxiliary variable. Obtain the variance estimators.
	
	\begin{table}[h]
	\def\arraystretch{1.5}
	
	\begin{center}
	\begin{tabular}{|c|>{\centering}m{1.5cm}|>{\centering}m{1.5cm}||c|>{\centering}m{1.5cm}|>{\centering\arraybackslash}m{1.5cm}|}
	
	\hline
	
	\multicolumn{3}{|c||}{Zone A} & \multicolumn{3}{c|}{Zone B} \\
	
	\hline
	
	Municipal Area & $x$ & $y$ & Municipal Area & $x$ & $y$ \\
	
	\hline
	
	1 & 27 & 33 & 1 & 29 & 32 \\
	
	\hline
	
	2 & 15 & 19 & 2 & 14 & 20 \\
	
	\hline
	
	3 & 20 & 26 & 3 & 40 & 53 \\
	
	\hline
	
	4 & 15 & 19 & 4 & 27 & 28 \\
	
	\hline
	
	5 & 52 & 56 & 5 & 43 & 48 \\
	
	\hline
	
	6 & 15 & 16 & 6 & 671 & 653 \\
	
	\hline
	
	7 & 62 & 70 & 7 & 78 & 79 \\
	
	\hline
	
	8 & 54 & 66 & 8 & 54 & 59 \\
	
	\hline
	
	9 & 12 & 12 & 9 & 28 & 27 \\
	
	\hline
	
	10 & 50 & 60 & 10 & 35 & 49 \\
	
	\hline
	
	 &  &  & 11 & 36 & 38 \\
	 
	 \hhline{~~~---}
	 
	  &  &  & 12 & 6 & 6 \\
	  
	  \hline
	
	
	\hline
	
	\end{tabular}
	\end{center}
	
	\end{table}
	
	
	
	
	
	
	
	
	
	
	
	
	
	
	
	
	
	
	
	
	
	
	
	
	
	
	
	
	
	
	
	\item The following data give the household sizes for 32 households in a village. Draw a simple random sample with replacement of 6 draws and hence obtain an estimate of the average household size along with its standard error. Obtain a 95\% confidence interval for the average household size in the population. 
	
	
5, 3, 7, 11, 4, 6, 10, 9, 8, 12, 11, 10, 10, 11, 8, 7, 6, 8, 9, 4, 1, 5, 7, 7, 12, 8, 9, 10, 9, 7, 6, 8.





















	\item The following data give the geographical area (in acres) under paddy for 58 villages. Draw an SRSWOR of 8 villages, find an estimate of average area per village under paddy, an estimate of its variance and its 95\% confidence interval.
	
	
	98, 270, 79, 273, 130, 158, 116, 194, 41, 33, 78, 56, 58, 19, 64, 81, 141, 58, 29, 46, 93, 127, 114, 88, 108, 58, 47, 69, 44, 56, 102, 102, 187, 161, 179, 76, 137, 179, 76, 137, 127, 104, 117, 170, 210, 101, 222, 223, 96, 114, 318, 272, 155, 292, 240, 201, 261, 189.
	
	
	
	
	
	
	
	
	
	
	
	
	
	
	
	
	
	
	
	
	
	
	

	
	
	
	
	
	
	
	
	\item Using the following data, estimate the total number of milch animals in 117 villages of Zone A by the method of regression estimation. Also, compare its precision with the ratio estimate and mean per unit estimate.
	
	\pagebreak
	
	
	\begin{center}
	\def\arraystretch{1.21}
	\setlength{\tabcolsep}{10pt}
	
	\begin{longtable}{|c|>{\centering}m{4cm}|>{\centering\arraybackslash}m{4cm}|}
	
	\hline
	
	Serial No. of Village & No. of Milch Animals in Survey (y) & No. of Milch Animals in Census (x) \\
	
	\hline
	
	1 & 1129 & 1141 \\
	
	\hline
	
	2 & 1144 & 1144 \\
	
	\hline
	
	3 & 1125 & 1127 \\
	
	\hline
	
	4 & 1138 & 1153 \\
	
	\hline
	
	5 & 1137 & 1117 \\
	
	\hline
	
	6 & 1127 & 1140 \\
	
	\hline
	
	7 & 1163 & 1153 \\
	
	\hline
	
	8 & 1153 & 1146 \\
	
	\hline
	
	9 & 1164 & 1189 \\
	
	\hline
	
	10 & 1130 & 1137 \\
	
	\hline
	
	11 & 1153 & 1170 \\
	
	\hline
	
	12 & 1125 & 1115 \\
	
	\hline
	
	13 & 1116 & 1130 \\
	
	\hline
	
	14 & 1115 & 1118 \\
	
	\hline
	
	15 & 1112 & 1122 \\
	
	\hline
	
	16 & 1112 & 1113 \\
	
	\hline
	
	17 & 1123 & 1166 \\
	
	\hline
	
	
	
	\end{longtable}
	\end{center}
	
	
	
	
	
	
	
	
	
	
	
	

	
	
	
	
	
	
	
	
	
	
	
	
	\item Using the following data, estimate the total number of trees in the districts by the regression method of estimation and compare its precision.
	
	
	
	\begin{table}[h]
	\def\arraystretch{1.5}
	
	\begin{center}
	\begin{tabular}{|>{\centering}m{2cm}|>{\centering}m{2cm}|>{\centering}m{2cm}|>{\centering}m{2cm}|>{\centering}m{2cm}|>{\centering\arraybackslash}m{2cm}|}
	
	\hline
	
	Stratum Number & Total no. of villages $(N_m)$ & Total area (in Hect.) under orchard $(X_m)$ & No. of villages in sample $(n_m)$ & Area under orchards in Hect. $(x_m)$ & Total number of trees $(y_m)$ \\
	
	\hline
	
	1 & 985 & 11253 & 6 & 10.63, 9.90, 1.45, 3.38, 5.17, 10.35 & 747, 719, 78, 201, 311, 448 \\
	
	\hline
	
	2 & 2196 & 25115 & 8 & 14.66, 2.61, 4.35, 9.87, 2.42, 5.60, 4.70, 36.75 & 580, 103, 316, 739, 196, 235, 212, 1646 \\
	
	\hline
	
	3 & 1020 & 18870 & 11 & 11.60, 5.29, 7.94, 7.29, 8.00, 1.20, 11.50, 7.96, 23.15, 1.70, 2.01 & 488, 227, 374, 491, 499, 50, 455, 47, 879, 115, 115 \\
	
	\hline
	
	\end{tabular}
	\end{center}
	
	\end{table}
	
	
	
	
	
	
	
	
	
	
	
	\pagebreak
	
	
	
	
	
	
	
	
	
	
	
	
	
	
	
	
	
	
	
	
	
	
	
	\item A pilot sample survey for study of cultivation practices and yield of guava was conducted by IASRI in Prayagraj district of Uttar Pradesh (India). From Umerpur-Neerna village, out of a total of 412 bearing trees, 15 clusters of size 4 trees each were selected and yields (in kg.) were recorded as given below :
	
	
	\begin{table}[h]
	\def\arraystretch{1.5}
	
	\begin{center}
	\begin{tabular}{|>{\centering}m{2cm}||>{\centering}m{2cm}|>{\centering}m{2cm}|>{\centering}m{2cm}|>{\centering\arraybackslash}m{2cm}|}
	
	\hline
	
	Cluster & 1st Tree & 2nd Tree & 3rd Tree & 4th Tree \\
	
	\hline
	
	1 & 5.53 & 4.84 & 0.69 & 15.79 \\
	
	\hline
	
	2 & 26.11 & 10.93 & 19.08 & 11.18 \\
	
	\hline
	
	3 & 11.08 & 0.65 & 4.21 & 7.56 \\
	
	\hline
	
	4 & 12.66 & 32.52 & 16.92 & 37.02 \\
	
	\hline
	
	5 & 0.87 & 3.56 & 4.81 & 57.54 \\
	
	\hline
	
	6 & 6.40 & 11.68 & 40.05 & 5.15 \\
	
	\hline
	
	7 & 54.21 & 34.63 & 52.55 & 37.96 \\
	
	\hline
	
	8 & 1.94 & 35.97 & 29.54 & 25.98 \\
	
	\hline
	
	9 & 37.94 & 47.07 & 16.94 & 28.11 \\
	
	\hline
	
	10 & 56.92 & 17.69 & 26.24 & 6.77 \\
	
	\hline
	
	11 & 27.59 & 38.10 & 24.76 & 6.53 \\
	
	\hline
	
	12 & 45.98 & 5.17 & 1.17 & 6.53 \\
	
	\hline
	
	13 & 7.13 & 34.35 & 12.18 & 9.86 \\
	
	\hline
	
	14 & 14.23 & 16.89 & 28.93 & 21.70 \\
	
	\hline
	
	15 & 3.53 & 40.76 & 5.15 & 1.25 \\
	
	\hline
	
	
	\end{tabular}
	\end{center}
	
	\end{table}
	
	\begin{enumerate}[(i)]
	\item Estimate the average yield (in kg.) per tree of guava in the Umerpur-Neerna village of Prayagraj along with its standard error.
	
	\item Estimate the intracluster correlation coefficient between trees within clusters and efficiency of cluster sampling as compared to simple random sampling.
	
	\end{enumerate}
\end{enumerate}


\end{document}