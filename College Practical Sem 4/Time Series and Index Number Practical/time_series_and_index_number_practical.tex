\documentclass[11pt, a4paper]{article}

\usepackage[top=1 in, bottom = 1 in, left = 1 in, right = 1 in ]{geometry}

\usepackage{amsmath, amssymb, amsfonts}
\usepackage{enumerate}
\usepackage{multirow}
\usepackage{hhline}
\usepackage{array}

\title{\textbf{QUESTIONS}}

\author{}
\date{}

\begin{document}

\maketitle



\begin{center}


\section*{A. TIME SERIES}


\end{center}



\vspace{1cm}




\begin{enumerate}
	
	\item Below are the production of a fertiliser factory :
		
	\begin{table}[h]
	\def\arraystretch{1.5}
	\begin{center}
	
	\begin{tabular}{|c||c|c|c|c|c|c|c|}
	
	\hline
	
	Year & 1995 & 1997 & 1998 & 1999 & 2000 & 2001 & 2004 \\
	
	\hline
	
	Production (1000 tonnes) & 77 & 88 & 94 & 85 & 91 & 98 & 90 \\
	
	\hline
	\end{tabular}
	\end{center}
	
	\end{table}
	
	
	\begin{enumerate}[(i)]
		\item Fit a linear trend line and obtain the trend values.
		\item What is the monthly increase in the production ?
	\end{enumerate}
	
	
	
	
	
	
	
	
	
	
\vspace{2cm}
	
	
	
	
	
	
	
	
	
	
	
	
	\item Below are the production of a certain factory manufacturing air-conditioners :
	
	\begin{table}[h]
	\def\arraystretch{1.5}
	
	\begin{center}
	
	\begin{tabular}{|c||c|c|c|c|c|c|c|c|c|c|c|}
	
	\hline
	
	Year & 1990 & 1991 & 1992 & 1993 & 1994 & 1995 & 1996 & 1997 & 1998 & 1999 & 2000 \\
	
	\hline
	
	Production (1000 units) & 17 & 20 & 19 & 26 & 24 & 40 & 35 & 55 & 51 & 74 & 79 \\
	
	\hline
	
	\end{tabular}
	\end{center}
	
	\end{table}
	
	Fit a second degree equation and obtain the trend values.
	
	
	
	
	
	
	
	
	
\vspace{2cm}










	\item Below are the population of India :
	
	\begin{table}[h]
	\def\arraystretch{1.5}
	
	\begin{center}
	\begin{tabular}{|c||c|c|c|c|c|c|c|}
	
	\hline
	
	Year & 1911 & 1921 & 1931 & 1941 & 1951 & 1961 & 1971 \\
	
	\hline
	
	Population (in Crores) & 25 & 25.1 & 27.9 & 31.9 & 36.1 & 43.9 & 54.7 \\
	
	\hline
	
	
	\end{tabular}
	\end{center}
	
	\end{table}
	
	\begin{enumerate}[(i)]
		\item Fit an exponential trend equation and obtain the trend values.
		
		\item Estimate the population in 1981, 2001, 2011.	
	\end{enumerate}
	
	
	
	
	
	
	
	
	
	
	
	
	
\newpage
	
	
	
	
	
	
	
	
	
	\item Using Ratio to Trend method, determine the quarterly seasonal indices for the following data :
	
	\begin{table}[h]
	\def\arraystretch{1.5}
	
	\begin{center}
	\begin{tabular}{|>{\centering}m{2cm}||>{\centering}m{2cm}|>{\centering}m{2cm}|>{\centering}m{2cm}|>{\centering\arraybackslash}m{2cm}|}
	
	\hline
	
	Year & Q1 & Q2 & Q3 & Q4 \\
	
	\hline
	\hline
	
	1995 & 30 & 40 & 36 & 34 \\
	
	\hline
	
	1996 & 34 & 52 & 50 & 44 \\
	
	\hline
	
	1997 & 40 & 58 & 54 & 48 \\
	
	\hline
	
	1998 & 54 & 76 & 68 & 62 \\
	
	\hline
	
	1999 & 80 & 92 & 86 & 82 \\
	
	\hline
	\end{tabular}
	\end{center}
	
	\end{table}
	



























	\item The following data show the annual rainfall (in millimeters) in the District of Midnapore, West Bengal :
	
	\begin{table}[h]
	\def\arraystretch{1.5}
	
	
	\begin{center}
	\begin{tabular}{|>{\centering}m{3cm}|>{\centering}m{3cm}||>{\centering}m{3cm}|>{\centering\arraybackslash}m{3cm}|}
	
	\hline
		
	Year & Rainfall & Year & Rainfall \\
	
	\hline
	\hline
	
	1980 & 1391 & 1986 & 1822 \\
	
	1981 & 1913 & 1987 & 1236 \\
	
	1982 & 1254 & 1988 & 1390 \\
	
	1983 & 1292 & 1989 & 1558 \\
	
	1984 & 1665 & 1990 & 2152 \\
	
	1985 & 1351 & 1991 & 1584 \\
	
	
	
	

	\hline
	

	
	\end{tabular}
	\end{center}
	
		
	\end{table}

	
	Determine trend by the method of moving averages and also by fitting a polynomial of appropriate degree.









\vspace{2cm}











	\item The following table gives the production of steel in India during 1972 to 1975 (in 000 tons) over different quarters :
	
	\begin{table}[h]
	\def\arraystretch{1.5}
	
	\begin{center}
	\begin{tabular}{|>{\centering}m{2cm}||>{\centering}m{2cm}|>{\centering}m{2.1cm}|>{\centering}m{2cm}|>{\centering\arraybackslash}m{2cm}|}
	
	\hline
	
	Year & 1st Quarter & 2nd Quarter & 3rd Quarter & 4th Quarter \\
	
	\hline
	
	1972 & 1336 & 1065 & 1215 & 1335 \\
	
	1973 & 1463 & 1039 & 1183 & 1161 \\
	
	1974 & 1306 & 1041 & 1290 & 1321 \\
	
	1975 & 1525 & 1251 & 1456 & 1408 \\
	
	\hline
		
	\end{tabular}
	\end{center}
	
	\end{table}

	Obtain sesonal indices by the method of trend ratios, assuming a linear trend.












	\pagebreak




















	\item The following data represent the monthly averages of tourist arrival in India for the years 1970 to 1975 :
	
	\begin{table}[h]
	\def\arraystretch{1.5}
	
	\begin{center}
	\begin{tabular}{|>{\centering}m{3cm}|>{\centering\arraybackslash}m{3cm}|}
	
	\hline
	
	Year & Monthly average tourist arrival \\
	
	\hline
	
	1970 & 23401 \\
	
	1971 & 25083 \\
	
	1972 & 28579 \\
	
	1973 & 34157 \\
	
	1974 & 35263 \\
	
	1975 & 38773 \\
	
	\hline
	
	\end{tabular}
	\end{center}
	
	\end{table}

	Fit an exponential trend to the data. Represent the original values and the trend values on a graph paper.
	
	
	
	
	
	
	
	
	
	
	
	
	
	
	
	
	
	
	
	
	
	
	
	
	
	
	
	
	
	\item The following data represent the production of finished steel in India for the years 1972 to 1975 :
	
	\begin{center}
		PRODUCTION OF FINISHED STEEL (000 tons)
	\end{center}
	
	\begin{table}[h]
	\def\arraystretch{1.5}
	
	\begin{center}
	\begin{tabular}{|c||c|c|c|c|c|c|c|c|c|c|c|c|}
	
	
	\hline
	
	Year & Jan & Feb & Mar & Apr & May & Jun & Jul & Aug & Sep & Oct & Nov & Dec \\
	
	\hline
	
	1972 & 420 & 414 & 502 & 365 & 368 & 332 & 390 & 396 & 429 & 417 & 422 & 496 \\
	
	1973 & 491 & 456 & 516 & 337 & 342 & 360 & 409 & 402 & 372 & 391 & 394 & 376 \\
	
	1974 & 463 & 365 & 478 & 310 & 325 & 406 & 415 & 437 & 438 & 445 & 430 & 446 \\
	
	1975 & 502 & 487 & 536 & 404 & 418 & 429 & 489 & 492 & 475 & 456 & 476 & 476 \\
	
	\hline
	
	\end{tabular}
	\end{center}
	
	\end{table}
	
	
	Compute the seasonal indices by ratio to moving average method. 









































	\item The following table gives the production of Iron ore (Lakh Tonnes) in India from 1976 to 1979 for different quarters :
	
	\begin{table}[h]
	\def\arraystertch{1.5}
	
	\begin{center}
	\begin{tabular}{|c||c|c|c|c|}
	
	\hline
	
	Year & 1st Quarter & 2nd Quarter & 3rd Quarter & 4th Quarter \\
	
	\hline
	
	1976 & 126 & 108 & 79 & 113 \\
	
	1977 & 131 & 110 & 73 & 110 \\
	
	1978 & 116 & 90 & 72 & 108 \\
	
	1979 & 124 & 97 & 69 & 101 \\
	
	\hline
	
	\end{tabular}
	\end{center}
	
	\end{table}
	
	Obtain the sesonal indices by the method of trend ratios, assuming a linear trend.
	
	
	
	
	
	
	
	
	
	
	
	
	\pagebreak
	
	
	
	
	
	
	
	
	
	
	
	
	
	
	\item Obtain the trend values for the following series by fitting a second-degree polynomial. Represent the trend values and the original data in a suitable diagram.
	
	\begin{table}[h]
	\def\arraystretch{1.5}
	
	\begin{center}
	\begin{tabular}{|>{\centering}m{2cm}|c|}
	
	\hline
	
	Year & Gross Earnings (Rs. crores) \\
	
	\hline
	
	1964-65 & 666 \\
	
	1965-66 & 748 \\
	
	1966-67 & 777 \\
	
	1967-68 & 823 \\
	
	1968-69 & 903 \\
	
	1969-70 & 957 \\
	
	1970-71 & 1010 \\
	
	\hline
	
	\end{tabular}
	\end{center}
	
	\end{table}
	
	
	
	
	
	
	
	
	
	
	
	
	
	
	
	
	
	
	
	
	
	
	
	
	
	
	
	
	\item Fit an \textit{Autoregressive Model of order one} for the following problem : \\
	
	Consider a time series dataset representing the monthly average temperature in a specific city over a period of 24 months. The dataset is as follows: \\
	
	15.2, 14.8, 16.5, 18.2, 19.7, 21.3, 23.1, 24.8, 22.5, 20.6, 18.9, 17.3, 16.1, 15.8, 16.6, 18.3, 20.1, 21.7, 23.2, 24.4, 23.3, 21.7, 19.5, 17.7.
	
	
	
	
	
	
	
	
	
	
	
	
	\item Fit an \textit{Autoregressive Model of order two} for the following problem : \\
	
	You have a dataset for the monthly sales (in thousands of units): \\
	
	12, 15, 18, 14, 17, 20, 19, 21, 23, 25, 22, 24.
	
	
	
	
	
	
	
	
	
	
	
	
	
	
	
	\item Forecast by \textit{exponential smoothing} technique for the following problem : \\
	
	Suppose you are a demand planner for a retail company, and you are tasked with forecasting the sales of a particular product for the next four months. You have historical sales data for the past 12 months, which you will use to estimate the optimal smoothing factor (alpha) and make the forecast. \\
	
	The sales data for the past 12 months is as follows: \\
	
	100, 110, 120, 115, 125, 135, 130, 140, 145, 150, 155, 160.
	
	
	
	
	
	
	
	
	
	
	
	
	
	
	
	
	
	
	
	
	
	\item Sixteen successive observations on a stationary time series are as follows :- \\
	
	1.6, 0.8, 1.2, 0.5, 0.9, 1.1, 1.1, 0.6, 1.5, 0.8, 0.9, 1.2, 0.5, 1.3, 0.8, 1.2 \\
	
	Evaluate the values of the \textit{Autocorrelation Function (A.C.F.)} for lag 1, 2, 3 and plot them.
	
	
	
\end{enumerate}




\pagebreak








\begin{center}

\section*{B. INDEX NUMBER}

\end{center}



\vspace{1cm}




\begin{enumerate}


	\item Compute price index and quantity index numbers for the year 2005 with 2000 as base year, using
	
	\begin{enumerate}[(i)]
	
		\item Laspeyre's Method,
		
		\item Paasche's Method and
		
		\item Fisher's Method.
	
	\end{enumerate}
	
	
	
	\begin{table}[h]
	\def\arraystretch{1.5}
	
	\begin{center}
	\begin{tabular}{|c||>{\centering}m{2cm}|>{\centering}m{2cm}||>{\centering}m{2cm}|>{\centering\arraybackslash}m{2cm}|}
	
	\hline
	
	\multirow{2}{*}{Commodity} & \multicolumn{2}{c||}{Quantity(Units)} & \multicolumn{2}{c|}{Expenditure(Rs.)} \\
	
	\hhline{~----}
	
	& 2000 & 2005 & 2000 & 2005 \\
	
	\hline
	
	A & 100 & 150 & 500 & 900 \\
	
	B & 80 & 100 & 320 & 500 \\
	
	C & 60 & 72 & 150 & 360 \\
	
	D & 30 & 33 & 360 & 297 \\
	
	\hline
	
	\end{tabular}
	\end{center}
	
	\end{table}
	
	
	
	
	
	
	
	
	
	
	
	
	
	
	
	
	
	
	
	
	
	
	\item Construct the wholesale price index number for 2004 and 2005 from the following data, using 2003 as the base year.
	
	\begin{table}[h]
	\def\arraystretch{1.45}
	
	\begin{center}
	\begin{tabular}{|c||>{\centering}m{2cm}>{\centering}m{2cm}>{\centering\arraybackslash}m{2cm}|}
	
	\hline
	
	\multirow{2}{*}{Commodity} & \multicolumn{3}{c|}{Wholesale price (in '00 Rs.) per quintal} \\
	
	& 2003 & 2004 & 2005 \\
	
	\hline
	
	A & 140 & 160 & 190 \\
	
	B  & 120 & 130 & 140 \\
	
	C & 100 & 105 & 108 \\
	
	D & 75 & 80 & 90 \\
	
	E & 250 & 270 & 300 \\
	
	F & 400 & 420 & 450 \\
	
	\hline
	
	\end{tabular}
	\end{center}
	
	\end{table}
	
	
	
	
	
	
	
	
	
	
	
	
	
	
	
	

	
	
	
	
	
	
	
	
	
	\item An inquiry into the budgets of middle class families in a city gave the following \\ information :
	
	\begin{table}[h]
	\def\arraystretch{1.5}
	
	\begin{center}
	\begin{tabular}{|c||c|c|c|c|c|}
	
	\hline
	
	\textit{Expenses on :} & Food 30\% & Rent 15\% & Clothing 20\% & Fuel 10\% & Others 25\% \\
	
	\hline
	
	\textit{Prices (in Rs.) in 2002 :} & 100 & 20 & 70 & 20 & 40 \\
	
	\textit{Prices (in Rs.) in 2005 :} & 90 & 20 & 60 & 15 & 55 \\
	
	\hline
	
	\end{tabular}
	\end{center}
	
	\end{table}
	
	
	Compute the price index number using :
	
	\begin{enumerate}[(i)]
		\item Weighted A.M. of price relatives, and
		
		\item Weighted G.M. of price relatives.
	
	\end{enumerate}
\end{enumerate}

\end{document}