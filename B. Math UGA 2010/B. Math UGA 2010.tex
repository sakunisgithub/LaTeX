\documentclass[11pt, a4paper]{article}

\usepackage[top=1 in, bottom = 1 in, left = 1 in, right = 1 in ]{geometry}

\usepackage{amsmath, amssymb, amsfonts}
\usepackage{enumerate}

\title{B. Math UGA 2010}
\author{Ananda Biswas}
\date{}

\begin{document}

\maketitle

\begin{enumerate}

	\item[8.] The roots of the equation $ x^4 + x^2 = 1 $ are
	\begin{enumerate}[(A)]
		\item all real and positive;
		\item never real;
		\item 2 positive and 2 negative;
		\item 1 positive, 1 negative and 2 non-real.
	
	\end{enumerate}
\begin{flushleft}
\textbf{Answer :}
\end{flushleft}
Clearly, for $ x \leq -1 $ and $ x \geq 1 $ the given equation will not have any solution. 

The equation will have solutions only when $ -1 < x < 1 $.

Let $ x^2 = t $ where $ t \in (0,1) $

Then, $ t^2 + t - 1 = 0 $ $ \quad $ $ \therefore t = \dfrac{-1 \pm \sqrt{5}}{2} $

When $ x^2 = \dfrac{-1 - \sqrt{5}}{2} < 0 $, we shall have 2 non-real roots of the given equation.

Now, $ \sqrt{5} > 2 $.  So, $ -1 + \sqrt{5} > 0 $ $ \quad $ $ \therefore \dfrac{-1 + \sqrt{5}}{2} > 0  $

When $ x^2 = \dfrac{-1 + \sqrt{5}}{2} $, we shall have $ x = \pm \sqrt{\dfrac{-1 + \sqrt{5}}{2}} $ 

\begin{flushright}
\textbf{(D) 1 positive, 1 negative and 2 non-real}
\end{flushright}


\end{enumerate}

\end{document}