\documentclass[11pt, a4paper]{article}

\usepackage[top=1 in, bottom = 1 in, left = 1 in, right = 1 in ]{geometry}

\usepackage{amsmath, amssymb, amsfonts}
\usepackage{enumerate}
\usepackage{multirow}
\usepackage{hhline}
\usepackage{array}
\usepackage{longtable}

\title{\textbf{QUESTIONS}}

\author{}
\date{}

\begin{document}

\maketitle

\begin{enumerate}

%01

	\item A random sample of size 9 is drawn from the distribution with pdf
	\begin{align*}
	f_{\theta}(x) &\propto \dfrac{x^2}{{\theta}^3}; \,  -3\theta < x < \theta; \, \theta > 0 \,\, \text{and} \\
	f_{\theta}(x) &= 0; \, \text{otherwise}
	\end{align*}
	and the observations are found to be 10, –30, 14, –45, –34, 7, 12, 11, –13.

Find the maximum likelihood estimate of $\theta$. Also find (with justification) the maximum likelihood estimate of the variance for the above distribution.





%02
	\item The time a client waits to be served by the mortgage specialist at a bank has probability density function 
	\begin{align*}
	f(x) &= \dfrac{1}{2{\theta}^3} x^2 e^{-x/\theta}; \, x > 0; \, \theta > 0.
	\end{align*}
	The waiting times of 15 clients are found to be 6, 12, 15, 14, 12, 10, 8, 9, 10, 9, 8, 7, 10, 7 and 3 minutes. 
Calculate the values of the maximum likelihood estimate and the method of moments estimate of $\theta$. 










%03
	\item Consider the life-time of an electric bulb which is exponentially distributed with mean $3\theta$. The life-time of 20 bulbs are found to be 0.11, 2.28, 6.33, 0.67, 3.68, 1.46, 4.17, 2.96, 4.93, 9.49, 1.02, 0.60, 3.85, 1.32, 0.24, 2.38, 0.41, 4.98, 6.91, 2.14 years. Obtain 
	\begin{enumerate}[(i)]
		\item MLE of $\theta$.
		\item unbiased estimate of $\theta$.
		\item MLE of $P\left[ X_5 > 6 \right]$.	
	\end{enumerate}
	
	
	
	
	
	
	
%04
	\item In order to test whether a coin is perfect or not a coin is tossed 6 times. The null hypothesis is rejected if and only if three heads are not obtained. What is the probability of type I error ? Find the power of the test when the corresponding probability of head is 0.4.
	
	
	
	
	
	
%05
	\item The life (in hours) of an electrical component is exponentially distributed with mean $\theta$, where $\theta (> 0)$ is an unknown parameter. For testing the null hypothesis $H_0 : \theta = 6$ against the alternative $H_1 : \theta < 6$, four such components are drawn independently. Under a test rule which rejects the null hypothesis when three or more of these four survive for less than six hours, what is the probability of type I error? Also, find the probability of type II error when $\theta = 4.2$.
	
	Draw a sketch of the power curve of the above test and comment.
\end{enumerate}
\end{document}