\documentclass[11pt, a4paper]{article}

\usepackage[top=1 in, bottom = 1 in, left = 1 in, right = 1 in ]{geometry}

\usepackage{amsmath, amssymb, amsfonts}
\usepackage{enumerate}
\usepackage{tasks}
\usepackage{multirow}
\usepackage{hhline}
\usepackage{array}
\usepackage{longtable}



\author{}
\date{}

\begin{document}


\begin{center}

\textbf{\huge Methods of Estimation}

\end{center}


\vspace{20pt}

\begin{enumerate}


%01

	\item A random sample of size 9 is drawn from the distribution with PDF
	\begin{align*}
	f_{\theta}(x) &\propto \dfrac{x^2}{{\theta}^3}; \,  -3\theta < x < \theta; \, \theta > 0 \,\, \text{and} \\
	f_{\theta}(x) &= 0; \, \text{otherwise}
	\end{align*}
	and the observations are found to be 10, –30, 14, –45, –34, 7, 12, 11, –13.

Find the maximum likelihood estimate of $\theta$. Also find (with justification) the maximum likelihood estimate of the variance for the above distribution.



\vspace{10pt}


%02
	\item The time a client waits to be served by the mortgage specialist at a bank has probability density function 
	\begin{align*}
	f(x) &= \dfrac{1}{2{\theta}^3} x^2 e^{-x/\theta}; \, x > 0; \, \theta > 0.
	\end{align*}
	The waiting times of 15 clients are found to be 6, 12, 15, 14, 12, 10, 8, 9, 10, 9, 8, 7, 10, 7 and 3 minutes. 
Calculate the values of the maximum likelihood estimate and the method of moments estimate of $\theta$. 





\vspace{10pt}




%03
	\item Consider the life-time of an electric bulb which is exponentially distributed with mean $3\theta$. The life-time of 20 bulbs are found to be 0.11, 2.28, 6.33, 0.67, 3.68, 1.46, 4.17, 2.96, 4.93, 9.49, 1.02, 0.60, 3.85, 1.32, 0.24, 2.38, 0.41, 4.98, 6.91, 2.14 years. Obtain 
	\begin{enumerate}[(i)]
		\item MLE of $\theta$.
		\item unbiased estimate of $\theta$.
		\item MLE of $P\left[ X_5 > 6 \right]$.	
	\end{enumerate}
	
	
	
	
\vspace{10pt}	
	
	
%04
	\item A random sample of size 20 is drawn from the distribution with PDF
	\begin{equation*}
	 f(x) =
		\begin{cases}
		 \theta e^{-\theta x}, & x > 0  \\
		 \hspace{10pt} 0, & \text{otherwise}.
		\end{cases}
	\end{equation*}
	
and the observations are found to be 0.10, 0.34, 0.77, 0.57, 1.63, 0.06, 1.14, 0.40, 0.07, 1.17, 0.17, 0.17, 0.16, 0.01, 0.79, 0.06, 0.38, 0.28, 0.17, 1.30.

Obtain method of moments estimate of $\theta$.
	
	
\vspace{10pt}	
	
	
%05
	\item A random sample of size 20 is drawn from the distribution with PDF
	\begin{equation*}
	 f(x) =
		\begin{cases}
		 \dfrac{\alpha^{\beta}}{\Gamma(\beta)} e^{-\alpha x}x^{\beta - 1}, & x > 0  \\
		 \hspace{30pt} 0, & \text{otherwise}.
		\end{cases}
	\end{equation*}
	
and the observations are found to be 2.13, 3.30, 3.78, 1.99, 2.91, 1.68, 1.18, 2.97, 3.89, 4.71, 2.05, 1.56, 0.48, 2.26, 3.27, 2.16, 2.71, 4.46, 1.78, 4.47.

Obtain method of moments estimates of $\alpha$ and $\beta$.










\newpage


\begin{center}

\textbf{\huge Type-I, Type-II Errors \& Power Curve}

\end{center}


\vspace{50pt}



%06
	\item An urn contains 6 marbels of which $\theta$ are white and the others black. In order to test the null hypothesis $H_0 : \theta = 3$ against the alternative $H_1 : \theta = 4$, two marbels are drawn at random (without replacement) and $H_0$ is rejected if both the marbels are white; otherwise $H_0$ is accepted. Find the probabilities of committing type $I$ and type $II$ errors.
	
	
	
	
	
\vspace{50pt}	
	
	
	
%07
	\item A random sample of size 10 is drawn from the distribution with PDF
	\begin{equation*}
	 f(x) =
		\begin{cases}
		 \dfrac{1}{2\sqrt{2\pi}x} \, \cdot \text{exp}\{-\frac{1}{8}(\ln x - \mu )^2\}, & x > 0, \,\, -\infty < \mu < \infty  \\
		 \hspace{60pt} 0, & \text{otherwise}.
		\end{cases}
	\end{equation*}
	
	Let the random sample be $X_1$, $X_2$, \ldots, $X_{10}$. To test $H_0 : \mu = 1$ against $H_1 : \mu > 1$, we reject $H_0$ if $\dfrac{1}{10} \sum\limits_{i = 1}^{10} \ln X_i > 1$. \\ \\
	Find P(type $I$ error). Also find P(type $II$ error) and power of the test when $\mu = 2$. Also draw a sketch of the power curve for $\mu = -3, -2, -1, 0, 1, 2, 3$ and comment on your findings.
	
	
	
	
	




\newpage

\begin{center}

\textbf{\huge Most Powerful Critical Region}

\end{center}


\vspace{50pt}
	


%08
	\item Let $X$ be a random variable with PMF under $H_0$ and $H_1$ given by
	
	\begin{table}[!htbp]
	\def\arraystretch{1.5}
	
	\begin{center}
	\begin{tabular}{|c||cccccc|}
	
	\hline
	
	$x$ & 1 & 2 & 3 & 4 & 5 & 6 \\
	
	\hline
	
	$f_0(x)$ & 0.01 & 0.01 & 0.01 & 0.01 & 0.01 & 0.95 \\
	
	\hline
	
	$f_1(x)$ & 0.05 & 0.04 & 0.03 & 0.02 & 0.01 & 0.85 \\
	
	\hline
	
	\end{tabular}
	\end{center}
	
	\end{table}
	
	Find the most powerful test of level $\alpha = 0.03$ . Also find the power of the test.
	
	
	
	
	
	
	
\vspace{50pt}	
	
	
	
	
	
%09
	\item Let $X$ be a random variable with PMF under $H_0$ and $H_1$ given by
	
	\begin{table}[!htbp]
	\def\arraystretch{2.5}
	
	\begin{center}
	\begin{tabular}{|>{\centering}m{1cm}||>{\centering}m{1cm}>{\centering}m{1cm}>{\centering}m{1cm}>{\centering}m{1cm}>{\centering\arraybackslash}m{1cm}|}
	
	\hline
	
	\multirow{1}{*}{$x$} & \multirow{1}{*}{1} & \multirow{1}{*}{2} & \multirow{1}{*}{3} & \multirow{1}{*}{4} & \multirow{1}{*}{5} \\
	
	\hline
	
	\multirow{1}{*}{$f_0(x)$} & \multirow{1}{*}{$\dfrac{1}{5}$} & \multirow{1}{*}{$\dfrac{1}{5}$} & \multirow{1}{*}{$\dfrac{1}{5}$} & \multirow{1}{*}{$\dfrac{1}{5}$} & \multirow{1}{*}{$\dfrac{1}{5}$} \\ 
	
	\hline
	
	\multirow{1}{*}{$f_1(x)$} & \multirow{1}{*}{$\dfrac{1}{6}$} & \multirow{1}{*}{$\dfrac{1}{4}$} & \multirow{1}{*}{$\dfrac{1}{6}$} & \multirow{1}{*}{$\dfrac{1}{4}$} & \multirow{1}{*}{$\dfrac{1}{6}$} \\
	
	\hline
	
	\end{tabular}
	\end{center}
	\end{table}
	
	Find the most powerful test of level $\alpha = 0.5$. Also find the power of the test.
	
	
	
	
	
	




\newpage

\begin{center}

\textbf{\huge Uniformly Most Powerful Critical Region}

\end{center}


\vspace{50pt}







	
	
	
	
%10
	\item A random sample of size 10 is drawn from the distribution with PDF
	\begin{equation*}
	 f(x) =
		\begin{cases}
		 \dfrac{1}{2\sqrt{2\pi}x} \, \cdot \text{exp}\{-\frac{1}{8}(\ln x - \mu )^2\}, & x > 0, \,\, -\infty < \mu < \infty  \\
		 \hspace{60pt} 0, & \text{otherwise}.
		\end{cases}
	\end{equation*}
	
	and the observations are found to be 1.71, 1.18, 3.50, 0.82, 2.01, 0.60, 0.60, 1.68, 0.31, 1.10. Find the uniformly most powerful test of level $\alpha = 0.05$ for testing $H_0 : \mu = 1$ against $H_1 : \mu > 1$ and draw your conclusions.
	
	
	
	
	
	
	

\vspace{50pt}






%11
	\item A random sample of size 10 is drawn from the distribution with PDF 
	\begin{equation*}
	f(x) = 
		\begin{cases}
		\theta \left( 1 - x \right)^{\theta - 1}, & 0 < x < 1 \\
		\hspace{25pt} 0, & \text{otherwise.}
		\end{cases}	
	\end{equation*}
	and the observations be $X_1$, $X_2$, \ldots, $X_{10}$.
	\begin{enumerate}[(i)]
	\item Find the uniformly most powerful test of level $\alpha = 0.05$ for testing $H_0 : \theta = 1$ against $H_1 : \theta < 1$.
	\item Find the uniformly most powerful test of level $\alpha = 0.05$ for testing $H_0 : \theta = 1$ against $H_1 : \theta > 1$.
	\item Obtain the power functions for both the cases. Hence, explain why UMP test for tesing $H_0 : \theta = 1$ against $H_1 : \theta \neq 1$ does not exist.
	\end{enumerate}
	
	
	
	
	
	

\newpage

\begin{center}

\textbf{\huge Unbiased Critical Region}

\end{center}


\vspace{50pt}	
	
	
	
	
	
	
%12
	\item Let $X$ be a random variable with PMF
	\begin{equation*}
	f(x) = 
		\begin{cases}
		\dfrac{2+4\alpha_1+\alpha_2}{6}, & \text{if } x = 1, \\ \\
		\dfrac{2-2\alpha_1+\alpha_2}{6}, & \text{if } x = 2, \\ \\
		\dfrac{1-\alpha_1-\alpha_2}{3}, & \text{if } x = 3.
		\end{cases}
	\end{equation*}
	where $\alpha_1 \geq 0$ and $\alpha_2 \geq 0$ are unknown parameters such that $\alpha_1 + \alpha_2 \leq 1$. For testing the null hypothesis $H_0 : \alpha_1 + \alpha_2 = 1$ against the alternative hypothesis $H_1 : \alpha_1 = \alpha_2 = 0$, suppose the critical region is $C = \{2, 3\}$. Check whether the said critical region is unbiased.
	
	
	
	
	
	
	
\newpage

\begin{center}

\textbf{\huge Likelihood Ratio Test}

\end{center}


\vspace{50pt}	
	
	
	
	
	
%13
	\item A random sample of size 10 is drawn from the distribution with PDF 
	\begin{equation*}
	f(x) = 
		\begin{cases}
		\theta x^{\theta - 1}, & 0 < x < 1 \\
		\hspace{8pt} 0, & \text{otherwise.}
		\end{cases}	
	\end{equation*}
	and the observations be $X_1$, $X_2$, \ldots, $X_{10}$. Apply the method of likelihood ratio testing to develop a critical region with size $\alpha = 0.05$ for testing the null hypothesis $H_0 : \theta = 1$ against the alternative hypothesis $H_1 : \theta = 2$.
	
	
	
	
	
	
	
	
\vspace{50pt}	
	
	
	
	
%14
	\item Let $X_1$, $X_2$, \ldots, $X_{10}$ be a random sample of size 10 drawn from a $N(\mu, 4)$ population. Use the method of likelihood ratio testing to develop a critical region with size $\alpha = 0.05$ for testing the null hypothesis $H_0 : \mu = 1$ against the alternative hypothesis $H_1 : \mu \neq 1$.
	
	
	
	
	

\newpage

\begin{center}

\textbf{\huge Confidence Interval}

\end{center}


\vspace{50pt}	
	
	
	
%15
	\item A random sample of size 10 is drawn from $U(0, \theta)$ distribution and the observations are found to be 1.39, 1.69, 1.48, 0.99, 0.15, 1.12, 1.76, 1.79, 1.94, 0.18. Obtain a 95\% confidence interval for $\theta$.
	
	
	

\vspace{50pt}
	
	
	
	
%16
	\item A random sample of size 10 is drawn from a distribution with PDF
	\begin{equation*}
	f(x) = 
		\begin{cases}
		e^{-\left(x-\theta\right)}, & x \geq \theta \\
		\hspace{15pt} 0, & \text{otherwise}
		\end{cases}	
	\end{equation*}
	and the observations are found to be 3.29, 2.78, 3.11, 2.69, 4.44, 2.43, 3.64, 2.21, 2.43, 2.69. Obtain a 95\% confidence interval for $\theta$.
	


\vspace{50pt}	
	
	
%17
	\item Suppose the body weights of 100 fathers and first-born sons are measured and the sample correlation 
coefficient is found to be $r = 0.38$. Obtain a 95\% confidence interval for the population correlation 
coefficient $\rho$. 





\newpage

\begin{center}

\textbf{\LARGE Large Sample Test based on Pearsonian $\chi^2$}

\end{center}


\vspace{30pt}



	
%18
	\item A six faced die is thrown 300 times and the results obtained are as follows:
	\begin{table}[!htbp]
	\def\arraystretch{1.5}
	
	\begin{center}
	\begin{tabular}{|>{\centering}m{2.5cm}||>{\centering}m{1cm}|>{\centering}m{1cm}|>{\centering}m{1cm}|>{\centering}m{1cm}|>{\centering}m{1cm}|>{\centering\arraybackslash}m{1cm}|}
	
	\hline
	
	Face & 1 & 2 & 3 & 4 & 5 & 6 \\
	
	\hline
	
	Frequency & 31 & 52 & 46 & 40 & 54 & 77 \\
	
	\hline
	
	\end{tabular}
	\end{center}
	
	
	\end{table}
	
	 Use the data to test whether the die is unbiased.
	 
	 
	 
	 
	 
	 
\vspace{30pt}	 
	 
	 
	 
	 
%19
	\item 1072 school boys are classified according to intelligence and at the same time their economic conditions are recorded. The results are shown in the following table.
	
	\begin{table}[!htbp]
	\def\arraystretch{1.5}
	
	\begin{center}
	\begin{tabular}{|>{\centering}m{2.5cm}|>{\centering}m{2cm}|>{\centering}m{2cm}|>{\centering}m{2cm}|>{\centering\arraybackslash}m{2cm}|}
	
	\hline
	
	Economic & \multicolumn{4}{c|}{Intelligence} \\
	
	\hhline{~----}
	
	Condition & Excellent & Good & Mediocore & Dull \\
	
	\hline
	
	Good & 48 & 199 & 181 & 82 \\
	
	\hline
	
	Not good & 81 & 185 & 190 & 106 \\
	
	\hline
	
	\end{tabular}
	\end{center}
	\end{table}	
	
	Judge whether there is any association between intelligence and economic conditions.
	
	
	
	
	
\vspace{30pt}	
	
	
	
	
	
%20
	\item There are two sections in a class having 120 and 100 pupils respectively. The following table gives their results in the half-yearly and annual examinations.
	
	\begin{table}[!htbp]
	\def\arraystretch{1.5}
	
	\begin{center}
	\begin{tabular}{|>{\centering}m{1.5cm}|>{\centering}m{1.5cm}|>{\centering}m{1.5cm}|>{\centering}m{2.5cm}|>{\centering}m{1.5cm}|>{\centering}m{1.5cm}|>{\centering\arraybackslash}m{1.5cm}|}
	
	
	\multicolumn{2}{c}{Section 1} & \multicolumn{1}{c}{} & \multicolumn{1}{c}{} & \multicolumn{1}{c}{} & \multicolumn{2}{c}{Section 2} \\
	
	\multicolumn{2}{c}{Half-yearly exam} & \multicolumn{1}{c}{} & \multicolumn{1}{c}{} & \multicolumn{1}{c}{} &  \multicolumn{2}{c}{Half-yearly exam} \\
	
	\hline
	
	Failed & Passed & & & & Passed & Failed \\
	
	\hline
	
	12 & 48 & Passed & \multirow{2}{*}{Annual Exam} & Passed & 21 & 8 \\
	
	\hhline{---~---}
	
	52 & 8 & Failed & & Failed & 6 & 65 \\
	
	\hline
	
	\end{tabular}
	\end{center}
	
	\end{table}

	\begin{enumerate}[(i)]
	\item For each section, test if the annual exam results have any association with the results of the half-yearly exam. 
	
	\item Test whether the two sections may be regarded as random samples from the same population.
	
	\end{enumerate}
\end{enumerate}
\end{document}