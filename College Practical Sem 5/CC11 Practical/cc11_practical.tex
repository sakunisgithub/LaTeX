\documentclass[11pt, a4paper]{article}

\usepackage[top=1 in, bottom = 1 in, left = 1 in, right = 1 in ]{geometry}

\usepackage{amsmath, amssymb, amsfonts}
\usepackage{enumerate}
\usepackage{multirow}
\usepackage{hhline}
\usepackage{array}
\usepackage{longtable}

\title{\textbf{QUESTIONS}}

\author{}
\date{}

\begin{document}

\maketitle

\begin{enumerate}

%01

	\item A random sample of size 9 is drawn from the distribution with pdf
	\begin{align*}
	f_{\theta}(x) &\propto \dfrac{x^2}{{\theta}^3}; \,  -3\theta < x < \theta; \, \theta > 0 \,\, \text{and} \\
	f_{\theta}(x) &= 0; \, \text{otherwise}
	\end{align*}
	and the observations are found to be 10, –30, 14, –45, –34, 7, 12, 11, –13.

Find the maximum likelihood estimate of $\theta$. Also find (with justification) the maximum likelihood estimate of the variance for the above distribution.





%02
	\item The time a client waits to be served by the mortgage specialist at a bank has probability density function 
	\begin{align*}
	f(x) &= \dfrac{1}{2{\theta}^3} x^2 e^{-x/\theta}; \, x > 0; \, \theta > 0.
	\end{align*}
	The waiting times of 15 clients are found to be 6, 12, 15, 14, 12, 10, 8, 9, 10, 9, 8, 7, 10, 7 and 3 minutes. 
Calculate the values of the maximum likelihood estimate and the method of moments estimate of $\theta$. 
\end{enumerate}
\end{document}