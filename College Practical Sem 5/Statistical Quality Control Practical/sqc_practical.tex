\documentclass[11pt, a4paper]{article}

\usepackage[top=1 in, bottom = 1 in, left = 1 in, right = 1 in ]{geometry}

\usepackage{amsmath, amssymb, amsfonts}
\usepackage{enumerate}
\usepackage{multirow}
\usepackage{hhline}
\usepackage{array}
\usepackage{longtable}

\title{\textbf{QUESTIONS}}

\author{}
\date{}

\begin{document}

\maketitle

\begin{enumerate}

%01

	\item The following data relate to the life(in hours) of 15 samples of 6 electric bulbs each, drawn at intervals of one hour from a production process. Draw the $\overline{x}$ and $R$ charts and comment.
	
	\begin{table}[h]
	\def\arraystretch{1.5}
	
	\begin{center}

	\begin{tabular}{|c||>{\centering}m{1.5cm}>{\centering}m{1.5cm}>{\centering}m{1.5cm}>{\centering}m{1.5cm}>{\centering}m{1.5cm}>{\centering\arraybackslash}m{1.5cm}|}
	\hline
	Sample No. & \multicolumn{6}{c|}{Life-time (in hours)}	\\
	\hline
	1 & 620 & 687 & 666 & 769 & 839 & 686 \\
	
	2 & 501 & 585 & 524 & 585 & 655 & 668 \\
	
	3 & 673 & 701 & 636 & 567 & 622 & 660 \\
	
	4 & 646 & 626 & 572 & 628 & 632 & 743 \\
	
	5 & 494 & 984 & 659 & 643 & 660 & 640 \\
	
	6 & 634 & 755 & 625 & 582 & 685 & 555 \\
	
	7 & 619 & 710 & 664 & 693 & 773 & 534 \\
	
	8 & 631 & 723 & 614 & 535 & 551 & 570 \\
	
	9 & 482 & 791 & 533 & 612 & 497 & 499 \\
	
	10 & 706 & 524 & 626 & 503 & 662 & 754 \\
	
	11 & 530 & 432 & 379 & 690 & 724 & 536 \\
	
	12 & 485 & 497 & 608 & 393 & 648 & 729 \\
	
	13 & 585 & 535 & 762 & 588 & 625 & 737 \\
	
	14 & 462 & 490 & 635 & 587 & 554 & 673 \\
	
	15 & 722 & 608 & 665 & 587 & 531 & 705 \\
	
	\hline
	
	\end{tabular}
	\end{center}
	
	\end{table}
	
	
	
	
	
	
	
	
	
	
	
%02

	\item A machine is set to deliver the packets of a given weight. Ten samples of size five each were examined and the following results were obtained :
	
	\begin{table}[h]
	\def\arraystretch{1.5}
	
	\begin{center}
	\begin{tabular}{|c|cccccccccc|}
	
	\hline
	
	Sample no. & 1 & 2 & 3 & 4 & 5 & 6 & 7 & 8 & 9 & 10 \\
	
	
	Mean & 43 & 49 & 37 & 44 & 45 & 37 & 51 & 46 & 43 & 47 \\
	
	Range & 5 & 6 & 5 & 7 & 7 & 4 & 8 & 6 & 4 & 6 \\
	
	\hline
	
	\end{tabular}
	\end{center}
	
	\end{table}
	
	Calculate the values for the central line and the control limits for the mean chart and range chart. Comment on the state of control.
	
	
	
	
\newpage





%03
	\item Construct a control chart for mean and the range for the following data on the basis of fuses, samples of 5 being taken every hour(each set of 5 has been arranged in ascending order of magnitude). Comment on whether the production seems to be under contorl, assuming that these are the first data :
	
	\begin{table}[h]
	\def\arraystretch{1.5}
	
	\begin{center}
	\begin{tabular}{|cccccccccccc|}
	
	\hline
	
	42 & 42 & 19 & 36 & 42 & 51 & 60 & 18 & 15 & 69 & 64 & 61 \\
	
	65 & 45 & 24 & 54 & 51 & 74 & 60 & 20 & 30 & 109 & 90 & 78 \\
	
	75 & 68 & 80 & 69 & 57 & 75 & 72 & 27 & 39 & 113 & 93 & 94 \\
	
	78 & 72 & 81 & 77 & 59 & 78 & 95 & 42 & 62 & 118 & 109 & 109 \\
	
	87 & 90 & 81 & 84 & 78 & 132 & 138 & 60 & 84 & 153 & 112 & 136 \\
	
	\hline
	
	\end{tabular}
	\end{center}
	
	\end{table}
\end{enumerate}
\end{document}