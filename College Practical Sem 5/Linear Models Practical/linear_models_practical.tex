\documentclass[11pt, a4paper]{article}

\usepackage[top=1 in, bottom = 1 in, left = 1 in, right = 1 in ]{geometry}

\usepackage{amsmath, amssymb, amsfonts}
\usepackage{enumerate}
\usepackage{multirow}
\usepackage{hhline}
\usepackage{array}
\usepackage{longtable}

\title{\textbf{QUESTIONS}}

\author{}
\date{}

\begin{document}

\maketitle

\begin{enumerate}

%01

	\item The average number of days survived by mice inoculated with 5 strains of typhoid organisms along with their standard deviation and number of mice involved in each experiment is given below. On the basis of these data, what would be your conclusions regarding the strains of typhoid organisms ?
	
	\begin{table}[h]
	\def\arraystretch{1.5}
	
	\begin{center}
	\begin{tabular}{|>{\centering}m{4 cm}|>{\centering}m{1.5 cm}>{\centering}m{1.5 cm}>{\centering}m{1.5 cm}>{\centering}m{1.5 cm}>{\centering\arraybackslash}m{1.5 cm}|}
	
	\hline
	
	Strains of typhoid & A & B & C & D & E \\
	
	\hline
	
	No. of mice, $n_i$ & 10 & 6 & 8 & 11 & 5 \\
	
	Average, $\overline{y_i}$ & 10.9 & 13.5 & 11.5 & 11.2 & 15.4 \\
	
	Standard deviation, $s_i$ & 12.72 & 5.96 & 3.24 & 5.65 & 3.64 \\
	
	\hline
	\end{tabular}
	\end{center}
	\end{table}
	
	
	
	
	
	
	





%02
	\item 
	\begin{enumerate}[(a)]
		\item A manufacturing company has purchased three new machines of different makes and wishes to determine whether one of them is faster than the others in producing a certain output. Five-hourly production figures are observed at random from each machine and the results are as follows.
		
		\begin{table}[h]
		\def\arraystretch{1.5}
		
		\begin{center}
		\begin{tabular}{|>{\centering}m{3 cm}|>{\centering}m{1.5 cm}|>{\centering}m{1.5 cm}|>{\centering\arraybackslash}m{1.5 cm}|}
		
		\hline
		
		& Machine $A_1$ & Machine $A_2$ & Machine $A_3$ \\
		
		\hline
		
		& 25 & 31 & 24 \\
		
		& 30 & 39 & 30 \\
		
		Observations & 36 & 38 & 28 \\
		
		& 38 & 42 & 25 \\
		
		& 31 & 35 & 28 \\
		
		\hline
		
		\end{tabular}
		\end{center}
		
		\end{table}
		
		Use analysis of variance technique and determine whether the machines are signifacntly different in their mean speeds. Use $\alpha = 5 \%$.
		
		\item Analyse the above data after shifting the origin to 30. How are the results in Part (a) affected ? Explain.
	
	\end{enumerate}
	
	
	
	
	
	
\newpage
	
	
	
	
%03
	\item The following table gives quality rating of ten service stations by five professional raters.
	
	\begin{table}[h]
	\def\arraystretch{1.5}
	
	\begin{center}
	\begin{tabular}{|c|>{\centering}m{1 cm}>{\centering}m{1 cm}>{\centering}m{1 cm}>{\centering}m{1 cm}>{\centering}m{1 cm}>{\centering}m{1 cm}>{\centering}m{1 cm}>{\centering}m{1 cm}>{\centering}m{1 cm}>{\centering\arraybackslash}m{1 cm}|}
	
	\hline
	
	\multirow{2}{*}{RATER} & \multicolumn{10}{c|}{SERVICE STATION} \\
	
	\hhline{~----------}
	
	& 1 & 2 & 3 & 4 & 5 & 6 & 7 & 8 & 9 & 10 \\
	
	\hline
	
	A & 99 & 70 & 90 & 99 & 65 & 85 & 75 & 70 & 85 & 92 \\
	
	B & 96 & 65 & 80 & 95 & 70 & 88 & 70 & 51 & 84 & 91 \\
	
	C & 95 & 60 & 48 & 87 & 48 & 75 & 71 & 93 & 80 & 93 \\
	
	D & 98 & 65 & 70 & 95 & 67 & 82 & 73 & 94 & 86 & 80 \\
	
	E & 97 & 65 & 62 & 99 & 60 & 80 & 76 & 92 & 90 & 89 \\
	
	\hline
	
	\end{tabular}
	\end{center}
	
	\end{table}
	
	Analyse the data and discuss whether there is any significant difference between raters or between service stations.
	
	
	
	
	
	
	
	
	
%04
	\item The following data shows the birth-weights of babies born, classified according to the age of mother and order of gravida, there being three observations per cell :
	
	\begin{table}[h]
	\def\arraystretch{1.5}
	
	\begin{center}
	\begin{tabular}{|c|c|c|c|c|c|}
	
	\hline
	
	\multirow{2}{*}{Order of gravida} & \multicolumn{5}{c|}{Age-group of mother} \\
	
	\hhline{~-----}
	
	& 15-20 & 20-25 & 25-30 & 30-35 & 35 and over \\
	
	\hline
	
	1 & 5.1, 5.0, 4.8 & 5.0, 5.1, 5.3 & 5.1, 5.1, 4.9 & 4.9, 4.9, 5.0 & 5.0, 5.0, 5.0 \\
	
	2 & 5.2, 5.2, 5.4 & 5.3, 5.3, 5.5 & 5.3, 5.2, 5.2 & 5.2, 5.0, 5.5 & 5.1, 5.3, 5.9 \\
	
	3 & 5.8, 5.7, 5.9 & 6.0, 5.9, 6.2 & 5.8, 5.9, 5.9 & 5.8, 5.5, 5.5 & 5.9, 5.4, 5.5 \\
	
	4 & 6.0, 6.0, 5.9 & 6.2, 6.5, 6.0 & 6.0, 6.1, 6.0 & 6.0, 5.8, 5.5 & 5.8, 5.6, 5.5 \\
	
	5 and over & 6.0, 6.0, 6.0 & 6.0, 6.1, 6.3 & 5.9, 6.0, 5.8 & 5.9, 6.0, 5.5 & 5.5, 6.0, 6.2 \\
	
	\hline
	
	\end{tabular}
	\end{center}
	
	\end{table}
	
	Test whether the age of mother and order of gravida significantly affect the birth-weight.
	
	
	
	
	
	
	
%05
	\item A test was given to give students taken at random from the fith class of three schools of a town. The individual scores are :
	
	\begin{table}[h]
	\def\arraystretch{1.5}
	
	\begin{center}
	\begin{tabular}{|>{\centering}m{2cm}|>{\centering}m{1cm}>{\centering}m{1cm}>{\centering}m{1cm}>{\centering}m{1cm}>{\centering\arraybackslash}m{1cm}|}
	
	\hline
	
	School I : & 9 & 7 & 6 & 5 & 8 \\
	
	School II : & 7 & 4 & 5 & 4 & 5 \\
	
	School III : & 6 & 5 & 6 & 7 & 6 \\
	
	\hline
	
	\end{tabular}
	\end{center}
	
	\end{table}
	
	Carry out the analysis of variance and state your conclusions.
	
	
	
	
	
	
\newpage
	
	
	
%06
	\item An experiment was conducted to determine the effects of different dates of planting and different methods of planting on the yield of sugar-cane. The data below show the yields of suar-cane (in kg.) for four different dates and three methods of planting.
	
	\begin{table}[h]
	\def\arraystretch{1.5}
	
	\begin{center}
	\begin{tabular}{|c|c|c|c|c|}
	
	\hline
	
	\multirow{2}{*}{Method of Planting} & \multicolumn{4}{c|}{Date of planting} \\
	
	\hhline{~----}
	
	& October & November & February & March \\
	
	\hline
	
	I & 7.10 & 3.69 & 4.70 & 1.90 \\
	
	II & 10.29 & 4.79 & 4.58 & 2.64 \\
	
	III & 8.30 & 3.58 & 4.90 & 1.80 \\
	
	\hline
	
	\end{tabular}
	\end{center}
	
	\end{table}
	
	Carry out and analysis of the above data.
\end{enumerate}
\end{document}