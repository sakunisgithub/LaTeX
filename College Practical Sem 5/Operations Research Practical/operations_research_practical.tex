\documentclass[11pt, a4paper]{article}

\usepackage[top=1 in, bottom = 1 in, left = 1 in, right = 1 in ]{geometry}

\usepackage{amsmath, amssymb, amsfonts}
\usepackage{enumerate}
\usepackage{multirow}
\usepackage{hhline}
\usepackage{array}
\usepackage{longtable}

\title{\textbf{QUESTIONS}}

\author{}
\date{}

\begin{document}

\maketitle

\begin{enumerate}

%01

\item Solve the following L.P.P. graphically :
	\begin{align*}
\textit{Minimize \hspace{6pt}} z = 20x_1 + 10x_2 \\
\textit{subject to \hspace{25pt}} x_1 + 2x_2 &\leq 40, \\
3x_1 + x_2 &\geq 30, \\
4x_1 + 3x_2 &\geq 60, \\
x_1, x_2 &\geq 0.
	\end{align*}
	

%02

\item Solve the following L.P.P. graphically :
	\begin{align*}
\textit{Maximize \hspace{6pt}} z = 6x_1 + 4x_2 \\
\textit{subject to \hspace{25pt}} 7x_1 + 5x_2 &\leq 35, \\
5x_1 + 7x_2 &\leq 35, \\
4x_1 + 3x_2 &\geq 12, \\
3x_1 + x_2 &\geq 3, \\
x_1, x_2 &\geq 0.
	\end{align*}
	

%03

\item Solve the following L.P.P. graphically :
	\begin{align*}
\textit{Maximize \hspace{6pt}} z = 3x_1 + 4x_2 \\
\textit{subject to \hspace{25pt}} x_1 - x_2 &\geq 0, \\
-x_1 + 3x_2 &\leq 3, \\
x_1, x_2 &\geq 0.
	\end{align*}
	

%04

\item Solve the following L.P.P. graphically :
	\begin{align*}
\textit{Maximize \hspace{6pt}} z = 6x_1 + 10x_2 \\
\textit{subject to \hspace{25pt}} 3x_1 + 5x_2 &\leq 10, \\
5x_1 + 3x_2 &\leq 15, \\
x_1, x_2 &\geq 0.
	\end{align*}
	
	
%05

\item Find the basic feasible solutions of the following system of equations.
	\begin{align*}
2x_1 + 3x_2 - x_3 + 4x_4 &= 8; \\
x_1 - 2x_2 + 6x_3 - 7x_4 &= -3; \\
x_1, x_2, x_3, x_4 &\geq 0.	
	\end{align*}
	
	
%06
	
\item Find all the basic solutions of the following equations and identify the basic vectors and the basic variables in each case.
	\begin{align*}
x_1 + x_2 + x_3 &= 4; \\
2x_1 + 5x_2 - 2x_3 &= 3.	
	\end{align*}
	

%07

\item Solve the following $L.P.P.$ using \textit{simplex method}.
	\begin{align*}
\textit{Maximize \hspace{6pt}} z = 4x_1 + 7x_2 \\
\textit{subject to \hspace{25pt}} 2x_1 + x_2 &\leq 1000, \\
10x_1 + 10x_2 &\leq 6000, \\
2x_1 + 4x_2 &\leq 2000, \\
x_1, x_2 &\geq 0.
	\end{align*}
	
%08
	
\item Solve the following $L.P.P.$ using \textit{simplex method}.
	\begin{align*}
\textit{Maximize \hspace{6pt}} z = 3x_1 + x_2 + 3x_3 \\
\textit{subject to \hspace{25pt}} 2x_1 + x_2 + x_3 &\leq 2, \\
x_1 + 2x_2 + 3x_3 &\leq 5, \\
2x_1 + 2x_2 + x_3 &\leq 6, \\
x_1, x_2, x_3 &\geq 0.
	\end{align*}
	

%09

\item Solve the following $L.P.P.$ using \textit{penalty method}.
	\begin{align*}
\textit{Maximize \hspace{6pt}} z = 2x_1 + 3x_2 + x_3 \\
\textit{subject to \hspace{10pt}} -3x_1 + 2x_2 + 3x_3 &= 8, \\
-3x_1 + 4x_2 + 2x_3 &= 7, \\
x_1, x_2, x_3 &\geq 0.
	\end{align*}
	

%10

\item Solve the following $L.P.P.$ using \textit{penalty method}.
	\begin{align*}
\textit{Maximize \hspace{6pt}} z = -2x_1 + x_2 + 3x_3 \\
\textit{subject to \hspace{25pt}} x_1 - 2x_2 + 3x_3 &= 2, \\
3x_1 + 2x_2 + 4x_3 &= 1, \\
x_1, x_2, x_3 &\geq 0.
	\end{align*}
	
	
	
%11

\item Use duality to find the optimal solution, if any, of the following $L.P.P.$
	\begin{align*}
\textit{Maximize \hspace{6pt}} z = 3x_1 + 2x_2 \\
\textit{subject to \hspace{20pt}} 2x_1 + x_2 &\leq 5, \\
x_1 + x_2 &\leq 3, \\
x_1, x_2 &\geq 0.
	\end{align*}
	
	


%12

\item Use duality to find the optimal solution, if any, of the following $L.P.P.$
	\begin{align*}
\textit{Maximize \hspace{6pt}} z = 15x_1 + 10x_2 \\
\textit{subject to \hspace{20pt}} 3x_1 + 5x_2 &\geq 5, \\
5x_1 + 2x_2 &\geq 3, \\
x_1, x_2 &\geq 0.
	\end{align*}
	
	
\newpage



%13

\item At a cattle breeding firm it is prescribed that the food ration for one animal must contain at least 14, 22 and 1 units of nutrients $A$, $B$, and $C$ respectively. Two different kinds of fodder are available. Each unit weight of these two contains the following amounts of the three nutrients :

	\begin{table}[h]
	\def\arraystretch{1.5}
	
	\begin{center}
	\begin{tabular}{|c|c|c|}
	
	\hline
	
	& Fodder 1 & Fodder 2 \\
	
	\hline
	
	Nutrient $A$ & 2 & 1 \\
	
	\hline
	
	Nutrient $B$ & 2 & 3 \\
	
	\hline
	
	Nutrient $C$ & 1 & 1 \\
	
	\hline
	
	\end{tabular}
	\end{center}
	
	\end{table}
	
	It is given that the costs of unit quantity of fodder 1 and 2 are 3 and 2 monetary units respectively. Pose a linear programming problem in terms of minimizing the cost of purchasing the fodders for the above cattle breeding firm.
	
	
	
	
	
	
	
%14

\item Three products are processed through three different operations. The times (in minutes) required per unit of each product, the daily capacity of the operations (in minutes per day) and the profit per unit sold for each product (in rupees) are as follows :

	\begin{table}[h]
	\def\arraystretch{1.5}
	
	\begin{center}
	\begin{tabular}{|c|c|c|c|c|}
	
	\hline
	
	\multirow{2}{*}{Operation} & \multicolumn{3}{c|}{Time per unit} & \multirow{2}{*}{Operation Capacity} \\
	
	\hhline{~---~}
	
	& Product 1 & Product 2 & Product 3 & \\
	
	\hline
	
	1 & 3 & 4 & 3 & 42 \\
	
	2 & 5 & 0 & 3 & 45 \\
	
	3 & 3 & 6 & 2 & 41 \\
	
	\hline
	
	Profit & 3 & 2 & 1 & \\
	
	\hline
	
	\end{tabular}
	\end{center}
	
	\end{table}
	
	The zero time indicates that the product does not require the given operation. The problem is to determine the optimum daily production for three products that maximizes the profit.
	
	
	Formulate the above production planning problem as a linear programming problem assumig that all units produced are sold.
	
	
	
	
	
	
	
%15

\item Use \textit{North-West Corner Rule} to find the basic feasible solution of the following transportation problem :

\begin{table}[h]
\def\arraystretch{1.5}

\begin{center}
\begin{tabular}{c|>{\centering}m{1cm}>{\centering}m{1cm}>{\centering}m{1cm}>{\centering\arraybackslash}m{1cm}|c}

\multicolumn{1}{c}{} & \multicolumn{1}{c}{$D_1$} & \multicolumn{1}{c}{$D_2$} & \multicolumn{1}{c}{$D_3$} & \multicolumn{1}{c}{$D_4$} & \multicolumn{1}{c}{$a_i$} \\

\cline{2-5}


$O_1$ & 19 & 20 & 50 & 10 & 7 \\

$O_2$ & 70 & 30 & 40 & 60 & 9 \\

$O_3$ & 40 & 8 & 70 & 20 & 18 \\

\cline{2-5}

\multicolumn{1}{c}{$b_j$} & \multicolumn{1}{c}{5} & \multicolumn{1}{c}{8} & \multicolumn{1}{c}{7} & \multicolumn{1}{c}{14} & \multicolumn{1}{c}{} \\


\end{tabular}
\end{center}

\end{table}





\newpage





%16

\item Use \textit{Row-minima method} to find the basic feasible solution of the following transportation problem :

\begin{table}[h]
\def\arraystretch{1.5}

\begin{center}
\begin{tabular}{c|>{\centering}m{1cm}>{\centering}m{1cm}>{\centering}m{1cm}>{\centering\arraybackslash}m{1cm}|c}

\multicolumn{1}{c}{} & \multicolumn{1}{c}{$D_1$} & \multicolumn{1}{c}{$D_2$} & \multicolumn{1}{c}{$D_3$} & \multicolumn{1}{c}{$D_4$} & \multicolumn{1}{c}{$a_i$} \\

\cline{2-5}


$O_1$ & 21 & 16 & 25 & 13 & 11 \\

$O_2$ & 17 & 18 & 14 & 23 & 13 \\

$O_3$ & 32 & 27 & 18 & 41 & 19 \\

\cline{2-5}

\multicolumn{1}{c}{$b_j$} & \multicolumn{1}{c}{6} & \multicolumn{1}{c}{10} & \multicolumn{1}{c}{12} & \multicolumn{1}{c}{15} & \multicolumn{1}{c}{} \\


\end{tabular}
\end{center}

\end{table}



\vspace{15pt}




%17

\item Use \textit{Column-minima method} to find the basic feasible solution of the following transportation problem :

\begin{table}[h]
\def\arraystretch{1.5}

\begin{center}
\begin{tabular}{c|>{\centering}m{1cm}>{\centering}m{1cm}>{\centering\arraybackslash}m{1cm}|c}

\multicolumn{1}{c}{} & \multicolumn{1}{c}{$D_1$} & \multicolumn{1}{c}{$D_2$} & \multicolumn{1}{c}{$D_3$} & \multicolumn{1}{c}{$a_i$} \\

\cline{2-4}


$O_1$ & 2 & 7 & 4 & 5 \\

$O_2$ & 3 & 3 & 1 & 8 \\

$O_3$ & 5 & 4 & 7 & 7 \\

$O_4$ & 1 & 6 & 2 & 14 \\

\cline{2-4}

\multicolumn{1}{c}{$b_j$} & \multicolumn{1}{c}{7} & \multicolumn{1}{c}{9} & \multicolumn{1}{c}{18} &  \multicolumn{1}{c}{} \\


\end{tabular}
\end{center}

\end{table}






\vspace{15pt}





%18

\item Use \textit{Matrix-minima method} to find the basic feasible solution of the following transportation problem :

\begin{table}[h]
\def\arraystretch{1.5}

\begin{center}
\begin{tabular}{c|>{\centering}m{1cm}|>{\centering}m{1cm}|>{\centering\arraybackslash}m{1cm}|c}

\multicolumn{1}{c}{} & \multicolumn{1}{c}{$A$} & \multicolumn{1}{c}{$B$} & \multicolumn{1}{c}{$C$} & \multicolumn{1}{c}{$a_i$} \\

\cline{2-4}

$F_1$ & 10 & 9 & 8 & 8 \\

\cline{2-4}

$F_2$ & 10 & 7 & 10 & 7 \\

\cline{2-4}

$F_3$ & 11 & 9 & 7 & 9 \\

\cline{2-4}

$F_4$ & 12 & 14 & 10 & 4 \\

\cline{2-4}

\multicolumn{1}{c}{$b_j$} & \multicolumn{1}{c}{10} & \multicolumn{1}{c}{10} & \multicolumn{1}{c}{8} &  \multicolumn{1}{c}{} \\


\end{tabular}
\end{center}

\end{table}






\newpage




%19

\item Obtain an optimal basic feasible solution of the following transportation problem :

\begin{table}[h]
\def\arraystretch{1.5}

\begin{center}
\begin{tabular}{c|>{\centering}m{1cm}|>{\centering}m{1cm}|>{\centering}m{1cm}|>{\centering\arraybackslash}m{1cm}|c}

\multicolumn{1}{c}{} & \multicolumn{1}{c}{$D_1$} & \multicolumn{1}{c}{$D_2$} & \multicolumn{1}{c}{$D_3$} & \multicolumn{1}{c}{$D_4$} & \multicolumn{1}{c}{$a_i$} \\

\cline{2-5}

$O_1$ & 1 & 2 & 1 & 4 & 30 \\

\cline{2-5}

$O_2$ & 3 & 3 & 2 & 1 & 50 \\

\cline{2-5}

$O_3$ & 4 & 2 & 5 & 9 & 20 \\

\cline{2-5}

\multicolumn{1}{c}{$b_j$} & \multicolumn{1}{c}{20} & \multicolumn{1}{c}{40} & \multicolumn{1}{c}{30} & \multicolumn{1}{c}{10} & \multicolumn{1}{c}{} \\


\end{tabular}
\end{center}

\end{table}




\vspace{15pt}




%20

\item Obtain an optimal basic feasible solution of the following transportation problem :

\begin{table}[h]
\def\arraystretch{1.5}

\begin{center}
\begin{tabular}{c|>{\centering}m{1cm}|>{\centering}m{1cm}|>{\centering}m{1cm}|>{\centering\arraybackslash}m{1cm}|c}

\multicolumn{1}{c}{} & \multicolumn{1}{c}{$W_1$} & \multicolumn{1}{c}{$W_2$} & \multicolumn{1}{c}{$W_3$} & \multicolumn{1}{c}{$W_4$} & \multicolumn{1}{c}{$a_i$} \\

\cline{2-5}

$F_1$ & 19 & 30 & 50 & 10 & 7 \\

\cline{2-5}

$F_2$ & 70 & 30 & 40 & 60 & 9 \\

\cline{2-5}

$F_3$ & 40 & 8 & 70 & 20 & 18 \\

\cline{2-5}

\multicolumn{1}{c}{$b_j$} & \multicolumn{1}{c}{5} & \multicolumn{1}{c}{8} & \multicolumn{1}{c}{7} & \multicolumn{1}{c}{14} & \multicolumn{1}{c}{} \\


\end{tabular}
\end{center}

\end{table}




\vspace{15pt}




%21


\item A steel company has three open hearth furnaces and five rolling mills. Transportation costs (rupees per quintal) for transporting steel from furnaces to rolling mills are shown in the following table :

\begin{table}[h]
\def\arraystretch{1.5}

\begin{center}
\begin{tabular}{c|>{\centering}m{1cm}|>{\centering}m{1cm}|>{\centering}m{1cm}|>{\centering}m{1cm}|>{\centering\arraybackslash}m{1cm}|c}

\multicolumn{1}{c}{} & \multicolumn{1}{c}{$M_1$} & \multicolumn{1}{c}{$M_2$} & \multicolumn{1}{c}{$M_3$} & \multicolumn{1}{c}{$M_4$} & \multicolumn{1}{c}{$M_5$} & \multicolumn{1}{c}{$a_i$} \\

\cline{2-6}

$F_1$ & 4 & 2 & 3 & 2 & 6 & 8 \\

\cline{2-6}

$F_2$ & 5 & 4 & 5 & 2 & 1 & 12 \\

\cline{2-6}

$F_3$ & 6 & 5 & 4 & 7 & 7 & 14 \\

\cline{2-6}

\multicolumn{1}{c}{$b_j$} & \multicolumn{1}{c}{4} & \multicolumn{1}{c}{4} & \multicolumn{1}{c}{6} & \multicolumn{1}{c}{8} & \multicolumn{1}{c}{8} & \multicolumn{1}{c}{} \\


\end{tabular}
\end{center}

\end{table}


What is the optimal transportation schedule ?





\newpage





%22
	\item Find the optimal assignment to obtain the minimum cost for the assignment problem with the following cost matrix.
	

\begin{table}[h]
\def\arraystretch{1.5}

\begin{center}
\begin{tabular}{c|>{\centering}m{1cm}>{\centering}m{1cm}>{\centering}m{1cm}>{\centering\arraybackslash}m{1cm}|}

\multicolumn{1}{c}{} & \multicolumn{1}{c}{$M_1$} & \multicolumn{1}{c}{$M_2$} & \multicolumn{1}{c}{$M_3$} & \multicolumn{1}{c}{$M_4$} \\

\cline{2-5}

$J_1$ & 10 & 24 & 30 & 15 \\

$J_2$ & 16 & 22 & 28 & 12 \\

$J_3$ & 12 & 20 & 32 & 10 \\

$J_4$ & 9 & 26 & 34 & 16 \\

\cline{2-5}

\end{tabular}
\end{center}

\end{table}






\vspace{15pt}






%23
	\item Find the optimal assignment to obtain the minimum cost for the assignment problem with the following cost matrix.
	
\begin{table}[h]
\def\arraystretch{1.5}

\begin{center}
\begin{tabular}{c|>{\centering}m{1cm}>{\centering}m{1cm}>{\centering}m{1cm}>{\centering}m{1cm}>{\centering\arraybackslash}m{1cm}|}

\multicolumn{1}{c}{} & \multicolumn{1}{c}{a} & \multicolumn{1}{c}{b} & \multicolumn{1}{c}{c} & \multicolumn{1}{c}{d} & \multicolumn{1}{c}{e} \\

\cline{2-6}

1 & 2 & 9 & 2 & 7 & 1 \\

2 & 6 & 8 & 7 & 6 & 1 \\

3 & 4 & 6 & 5 & 3 & 1 \\

4 & 4 & 2 & 7 & 3 & 1 \\

5 & 5 & 3 & 9 & 5 & 1 \\

\cline{2-6}


\end{tabular}
\end{center}

\end{table}







\vspace{15pt}








\end{enumerate}
\end{document}