\documentclass[11pt, a4paper]{article}

\usepackage[top=1 in, bottom = 1 in, left = 1 in, right = 1 in ]{geometry}

\usepackage{amsmath, amssymb, amsfonts}
\usepackage{enumerate}
\usepackage{multirow}
\usepackage{hhline}
\usepackage{array}

\title{\textbf{QUESTIONS}}
\author{}
\date{}

\begin{document}

\maketitle

\begin{enumerate}

	\item Given below is the data regarding deaths in two districts. On the basis of the given data, calculate the standardized death rates. Give your comments.
	\begin{table}[h]
	\def\arraystretch{1.5}
	\begin{center}

	\begin{tabular}{|c||c|c||c|c||c|}
	\hline
	\multirow{2}{*}{\textit{Age-range}} & \multicolumn{2}{c||}{\textit{District A}}  & \multicolumn{2}{c||}{\textit{District B}} & \multirow{1}{*}{\textit{Age-distribution of a}} \\
	\hhline{~----}
	& \textit{Population} & \textit{No. of deaths} & \textit{Population} & \textit{No. of Deaths} & \textit{standard 1000} \\
	\hline
	0\textemdash10 & 2000 & 50 & 1000 & 20 & 206 \\
	10\textemdash55 & 7000 & 75 & 3000 & 30 & 583 \\
	55 and above & 1000 & 25 & 2000 & 40 & 211 \\
	\hline

	\end{tabular}
	\end{center}
	\end{table}











	\item Calculate the crude and standardised death rates for the local population from the \newline following data and compare them with crude death rate of the standard population.
	
	\begin{table}[h]
	\def\arraystretch{1.5}
	\begin{center}
	\begin{tabular}{|c||c|c||c|c|}
	\hline
	\textit{Age-group} & \textit{Standard population} & \textit{Deaths} & \textit{Local population} & \textit{Deaths} \\
	\hline
	0\textemdash10 & 600 & 18 & 400 & 16 \\
	\hline
	10\textemdash20 & 1000 & 5 & 1500 & 6 \\
	\hline
	20\textemdash60 & 3000 & 24 & 2400 & 24 \\
	\hline
	60\textemdash100 & 400 & 20 & 700 & 21 \\
	\hline
	
	\end{tabular}
	\end{center}
	\end{table}
	
	
	
	
	
	
	
	
	
	
	
	
	
	
	
	
	
	
	\item The age-specific death rate (A.S.D.R.) for a district together with the population in the same age-groups for the state are given below. Compute a standardised death rate for the district.
	
	\begin{table}[h]
	\def\arraystretch{1.12}
	\begin{center}
	\begin{tabular}{|c||c||c|}

	\hline
	\textit{Age-groups} & \textit{A.S.D.R.(per 1000 in district)} & \textit{Population($\times$ 1000 in state)} \\
	\hline
	0\textemdash5 & 72 & 6855 \\
	\hline
	5\textemdash10 & 35 & 6971 \\
	\hline
	10\textemdash20 & 18 & 12342 \\
	\hline
	20\textemdash30 & 13 & 9816 \\
	\hline
	30\textemdash40 & 15 & 7732 \\
	\hline
	40\textemdash50 & 22 & 6015 \\
	\hline
	50\textemdash60 & 36 & 3923 \\
	\hline
	60\textemdash70 & 42 & 1281 \\
	\hline
	70\textemdash80 & 59 & 825 \\
	\hline
	$>80$ & 79 & 363 \\
	\hline
	
	\end{tabular}
	\end{center}
	\end{table}
	
	
	
	
	
	
	
	
	
	
	
	
	
	
	
	\item Calculate the crude and standardised death rates for the following data. 
	\begin{table}[h]
	\def\arraystretch{1.5}
	\begin{center}
	\begin{tabular}{|c|c|c|c|}

	\hline
	\textit{Age-group} & \textit{Population ($\times$ 1000)}& \textit{Number of Deaths} & \textit{Standard Population per 1000}\\
	\hline
	0\textemdash9 & 21 & 350 & 221 \\
	\hline
	10\textemdash24 & 30 & 102 & 298 \\
	\hline
	25\textemdash44 & 37 & 229 & 285 \\
	\hline
	45\textemdash64 & 17 & 354 & 149 \\
	\hline
	65 and over & 5 & 415 & 47 \\
	\hline
	
	\end{tabular}
	\end{center}
	\end{table}
	
	
	
	
	
	
	
	
	
	
	
	
	
	
	
	
	
	
	
	
	
	
	\item Compute the crude and standardised death rates of the two populations A and B, \newline regarding A as standard population, from the data below :
	
	\begin{table}[h]
	\def\arraystretch{1.5}
	\begin{center}
	\begin{tabular}{|c||c|c||c|c|}
	
	\hline	
	\multirow{2}{*}{\textit{Age-group (Years)}} & \multicolumn{2}{c||}{\textit{A}} & \multicolumn{2}{c|}{\textit{B}} \\
	\hhline{~----}
	& \textit{Population} & \textit{Deaths} & \textit{Population }& \textit{Deaths} \\
	\hline
	\textit{Under} 10 & 20000 & 600 & 12000 & 372 \\
	10 {\textemdash} 20 & 12000 & 240 & 30000 & 660 \\
	20 {\textemdash} 40 & 50000 & 1250 & 62000 & 1612 \\
	40 {\textemdash} 60 & 30000 & 1050 & 15000 & 525 \\
	\textit{Above} 60 & 10000 & 500 & 3000 & 180 \\
	\hline	
	
	\end{tabular}
	\end{center}
	\end{table}
	
	
	
	
	
	
	
	
	
	
	
	
	
	
	\item Suppose the following are the statistics of population and unemployment in - 
	\begin{enumerate}[(a)]
	\item your country as a whole and
	\item the local administrative area in which you live.
	
	\end{enumerate}
	
	\begin{table}[h]
	\def\arraystretch{1.5}
	\begin{center}
	\begin{tabular}{|c||c|c|c|c|}

	\hline
	& \multicolumn{4}{c|}{\textit{Age (in years)}}  \\
	\hhline{~----}
	& 15\textemdash30 & 30\textemdash45 & 45\textemdash60 & above 60 \\
	\hline
	\hline
	\textit{Standard population age-composition} & 250 & 350 & 300 & 100 \\
	\hline
	\textit{Unemployment \%} & 5 & 8 & 12 & 15\\
	\hline
	\hline
	\textit{Local population age-composition} & 300 & 300 & 350 & 50 \\
	\hline
	\textit{Unemployment \%}  & 4 & 9 & 12 & 20 \\
	\hline
		
	\end{tabular}
	\end{center}
	\end{table}
	
	Calculate - 
	\begin{enumerate}[(i)]
	\item the crude rate of unemployment in the total area.
	\item the standardised death rate of unemployment in the local area (taking the age-distribution of the country as standard).
	
	\end{enumerate}
	
	
	
	
	\newpage
	
	
	
	
	
	
	
	
	
	
	
	
	
	
	
	
	
	
	\item The age-specific death rates for Poland and Sweden for 1957 and the age-distribution of a standard population are given below. Compute the standardized death rates for the two countries. Compare the mortality conditions in the two countries.
	
	\begin{table}[h]
	\def\arraystretch{1.5}
	\begin{center}
	\begin{tabular}{|c||c|c||c|}

	\hline
	\multirow{2}{*}{\textit{Age}} & \multicolumn{2}{c||}{\textit{Death rate per thousand}} & \multirow{2}{*}{\textit{Standard population(in thousand)}} \\
	\hhline{~--~}
	& \textit{Poland} & \textit{Sweden} & \\
	\hline
	0\textemdash4 & 18.870 & 4.348 & 119.9 \\
	\hline
	5\textemdash14 & 0.759 & 0.465 & 206.9 \\
	\hline
	15\textemdash24 & 1.385 & 0.767 & 183.2 \\
	\hline
	25\textemdash34 & 2.048 & 1.075 & 147.9 \\
	\hline
	35\textemdash44 & 3.326 & 1.882 & 120.5 \\
	\hline
	45\textemdash54 & 7.006 & 4.669 & 93.9 \\
	\hline
	55\textemdash64 & 18.111 & 12.477 & 70.8 \\
	\hline
	65\textemdash74 & 45.795 & 34.060 & 40.5 \\
	\hline
	75 and over & 124.258 & 116.433 & 16.4 \\
	\hline
	
	\end{tabular}
	\end{center}
	\end{table}
	
	
	
	
	
	
	
	
	
	
	
	
	
	
	
	
	
	\item The mortality experiences in a given calendar year of two populations A and B consisting of men aged 70\textemdash74 as follows :
	\begin{table}[h]
	\def\arraystretch{1.5}
	\begin{center}
	\begin{tabular}{|c||c|c|c||c|c|c|}

	\hline
	\multirow{3}{*}{\textit{Age x}} & \multicolumn{3}{c||}{\textit{Population A}} & \multicolumn{3}{c|}{\textit{Population B}} \\
	\hhline{~------}
	& \textit{No. exposed to} & \textit{Deaths of} & $m_x a$ & \textit{No. exposed to} & \textit{Deaths of} & $m_x b$ \\
	& \textit{risk at age x} & \textit{age x} & & \textit{risk at age x} & \textit{age x} &  \\
	\hline
	70 & 2000 & 64 & 0.032 & 3000 & 81 & 0.027 \\
	\hline
	71 & 1200 & 42 & 0.035 & 3500 & 112 & 0.032 \\
	\hline
	72 & 1400 & 56 & 0.040 & 3000 & 117 & 0.039 \\
	\hline
	73 & 1800 & 81 & 0.045 & 4000 & 184 & 0.046 \\
	\hline
	74 & 1600 & 80 & 0.050 & 8500 & 140 & 0.056 \\
	\hline
	\hline
	total & 8000 & 325 & & 16000 & 634 & \\
	\hline
	
	\end{tabular}
	\end{center}
	\end{table}
	
	Express the mortality of population B at ages 70\textemdash74 as a single percentage ratio of that of population A by means of the following :
	
	\begin{enumerate}[(i)]
	\item indirect standardisation
	\item direct standardisation
	\item comparative mortality index
	\end{enumerate}
	
	Discuss the differences between the results and the reasons for these differences.
	
	
	
	
	
	
	
	\newpage
	
	
	
	
	
	
	
	
	
	
	
	
	
	
	
	
	
	
	
	
	
	
	
	
	
	
	
	
	
	
	
	
	
	
	
	
	
	
	
	
	\item A part of a life table is given here with most of the entries missing. On the basis of the available figures, supply the missing ones.
	
	\begin{table}[h]
	\def\arraystrecth{1.5}
	\begin{center}
	\begin{tabular}
	{
	|>{\centering}m{1cm}|
	>{\centering}m{1cm}|
	>{\centering}m{1cm}|
	>{\centering}m{1.5cm}|
	>{\centering}m{1cm}|
	>{\centering}m{1.9cm}|
	>{\centering\arraybackslash}m{1.2cm}|
	}
	
	\hline
	Age $x$ & $l_{x}$ & $d_{x}$ & $q_{x}$ & $L_{x}$ & $T_{x}$ & $e_{x}^{0}$ \\
	\hline
	10 & 74600 & & .00350 & & &  \\
	
	11 &  & & .00338 & & &  \\
	
	12 &  & & .00361 & & &  \\
	
	13 &  & & .00420 & & &  \\
	
	14 &  & & .00517 & & &  \\
	
	15 &  & & .00530 & & &  \\
	
	16 &  & & .00538 & & &  \\
	
	17 &  & & .00544 & & &  \\
	
	18 &  & & .00549 & & &  \\
	
	19 &  & & .00554 & & 2607040 & \\
	\hline
	
	
	\end{tabular}
	\end{center}
	\end{table}
	
	Hence determine, according to the life table, the probability
	\begin{enumerate}[(a)]
	\item that a child of age 10 will live at least 5 years more.
	\item that two children aged 10 and 11 will each live at least 5 years more.
	\item that of two children aged 10 and 11, at least one will die within 9 years.
	\end{enumerate}
	
	
	
	
	
	
	
	
	
	
	
	
	
	
	
	
	
	
	
	
	
	\item In the second and third columns of the following table are given the age-specific death rates for Kerala and West Bengal for the year 1993. The figures in the fourth column give the estimated age-distribution of the Indian population for the same year.
	
	
	
	\begin{table}[h]
	\def\arraystretch{1.2}
	\begin{center}
	\begin{tabular}
	{|>{\centering}m{3cm}||
	>{\centering}m{2cm}|
	>{\centering}m{3cm}||
	>{\centering\arraybackslash}m{4cm}|}
	
	\hline
	
	\multirow{2}{*}{Age (years l.b.d.)} & \multicolumn{2}{c||}{Death rate (per thousand)} & \multirow{2}{4cm}{\centering Percentage in estimated population}  \\
	
	
	\hhline{~--~}
	
	& Kerala & West Bengal & \\
	\hline
	0 {\textemdash} 4 & 3.4 & 17.0 & 12.8 \\
	
	5 {\textemdash} 9 & 0.1 & 1.5 & 12.1 \\
	
	10 {\textemdash} 14 & 0.3 & 0.9 & 11.2 \\
	
	15 {\textemdash} 19 & 0.8 & 1.7 & 10.5 \\
	
	20 {\textemdash} 24 & 0.9 & 2.3 & 9.7 \\
	
	25 {\textemdash} 29 & 1.1 & 1.7 & 8.2 \\
	
	30 {\textemdash} 34 & 1.7 & 2.4 & 6.9 \\
	
	35 {\textemdash} 39 & 1.7 & 2.3 & 6.2 \\
	
	40 {\textemdash} 44 & 2.4 & 4.0 & 5.0 \\
	
	45 {\textemdash} 49 & 4.1 & 4.8 & 4.4 \\
	
	50 {\textemdash} 54 & 7.4 & 10.1 & 3.6 \\
	
	55 {\textemdash} 59 & 12.2 & 16.9 & 3.0 \\
	
	60 {\textemdash} 64 & 21.6 & 24.6 & 2.4 \\
	
	65 {\textemdash} 69 & 27.3 & 40.5 & 1.8 \\
	
	70+ & 85.5 & 79.5 & 2.2 \\
	\hline
	All ages & 6.0 & 7.4 & 100.0 \\
	\hline
	
	\end{tabular}
	\end{center}
	
	\end{table}
	
	Computer the standardised death rates for Kerala and West Bengal taking all-India \newline population as standard.
	
	
	
	
	
	
	
	
	
	
	
	\item The number of births occuring in Israel in 1988 is shown here classified according to age of mother, together with the female population in each age-group of the child-bearing \newline period :
	
	\begin{table}[h]
	\def\arraystretch{1.4}
	\begin{center}
	\begin{tabular}{|>{\centering}m{2cm}|c|>{\centering\arraybackslash}m{4cm}|}
	
	\hline
	Age & Female Population ($\times$ 1000) & {\centering Number of births to mothers in the age-group} \\
	\hline
	15 {\textemdash} 19 & 200.6 & 4227 \\
	
	20 {\textemdash} 24 & 173.5 & 26099 \\
	
	25 {\textemdash} 29 & 161.7 & 32844 \\
	
	30 {\textemdash} 34 & 160.9 & 23449 \\
	
	35 {\textemdash} 39 & 155.7 & 11588 \\
	
	40 {\textemdash} 44 & 125.6 & 2071 \\
	
	45 {\textemdash} 49 & 87.6 & 122 \\
	\hline
	Total & 1065.6 & 100400 \\
	\hline
	
	
	\end{tabular}
	\end{center}
	\end{table}
	
	
	
	
	The total population of Israel in 1988 was 4441.7 thousand. With the above information, determine 
	\begin{enumerate}[(a)]
	\item the crude birth rate,
	\item the general fertility rate,
	\item the age-specific fertility rates,
	\item the total fertility rate for 1988. Also compute,
	\item the gross reproduction rate, assuming that the sex-ratio at birth was 104.5 male births to 100 female births in 1988.
	\end{enumerate}
	
	
	
	
	
	
	
	
	
	
	
	
	
	\item The quinquennial fertility rates (computed on the basis of female births alone) for England and Wales, 1954, are shown in the following table, together with the survival factor for each 5-year age-group (which is the probability for a newborn female to survive till the mid-point of the age-group and is approximately equal to ${^{f}_{5}L_{x}}/{5^{f}l_{0}}$) :
	
	\begin{table}[h]
	\def\arraystretch{1.5}
	\begin{center}
	\begin{tabular}{|c|c|c|}
	
	\hline
	Age & Fertility rate (female births) & Survival factor \\
	\hline
	
	15 {\textemdash} 19 & 0.0108 & 0.969 \\
	20 {\textemdash} 24 & 0.0662 & 0.967 \\
	25 {\textemdash} 29 & 0.0675 & 0.963 \\
	30 {\textemdash} 34 & 0.0413 & 0.958 \\
	35 {\textemdash} 39 & 0.0216 & 0.952 \\
	40 {\textemdash} 44 & 0.0063 & 0.942 \\
	45 {\textemdash} 49 & 0.0004 & 0.928 \\
	\hline
	
	
	\end{tabular}
	\end{center}
	\end{table}
	
	Compute the $GRR$ and $NRR$ for England and Wales for 1954 on the basis of the above data.
	
	
	
	
	
	
	
	
	
	
	
	
	
	
	
	
	
	
	
	
	\item The following table shows, for a certain country, the 1984 female population in 5-year age-groups, the life table value $^{f}_{5}l_{x}$ according to projected mortality and the projected age-specific fertility rates :
	
	\begin{table}[h]
	\def\arraystretch{1.5}
	\begin{center}
	\begin{tabular}{|c|
	>{\centering}m{3.5cm}|
	>{\centering}m{3.5cm}|
	>{\centering\arraybackslash}m{3.5cm}|}
	
	\hline
	Age l.b.d. & Population in 1984 (in thousands) & $^{f}_{5}l_{x}$ according to projected mortailty (with $^{f}l_{0}$ = 1000) & Projected $ASFR$ (per thousand females) \\
	\hline
	
	0 {\textemdash} 4 & 10136 & 4890 & {\textendash}\\
	5 {\textemdash} 9 & 10006 & 4873 & {\textendash}\\
	10 {\textemdash} 14 & 9065 & 4865 & 0.86 \\
	15 {\textemdash} 19 & 8045 & 4855 & 2.80 \\
	20 {\textemdash} 24 & 6546 & 4839 & 219.90 \\
	25 {\textemdash} 29 & 5614 & 4820 & 179.44 \\
	30 {\textemdash} 34 & 5632 & 4795 & 103.90 \\
	35 {\textemdash} 39 & 6193 & 4758 & 50.03 \\
	40 {\textemdash} 44 & 6345 & 4704 & 30.821 \\
	45 {\textemdash} 49 & 5796 & 4624 & 0.81 \\
	50 {\textemdash} 54 & 5336 & 4505 & {\textendash}\\
	55 {\textemdash} 59 & 4642 & 4334 & {\textendash}\\
	60 {\textemdash} 64 & 4451 & 4093 & {\textendash}\\
	65 {\textemdash} 69 & 3481 & 3742 & {\textendash}\\
	70 {\textemdash} 74 & 2799 & 3253 & {\textendash}\\
	75 {\textemdash} 79 & 1702 & 2604 & {\textendash}\\
	80 {\textemdash} 84 & 1074 & 1792 & {\textendash}\\
	85 {\textemdash} 89 & 411 & 923 & {\textendash}\\
	90 {\textemdash} 94 & 78 & 176 & {\textendash}\\
	95 {\textemdash} 99 & 27 & 10 & {\textendash}\\
	
	\hline
	Total & 97379 & &  \\
	\hline
		
	\end{tabular}
	\end{center}
	\end{table}


	Give your projection of the female population of the country for 1994 (in 5-year \newline age-groups), assuming the effect of migration is negligible and that the proportion of female births among all births is 0.4885.
	
	
	\newpage
	
	
	
	
	
	
	
	
	
	
	
	
	
	\item The population of India, as recorded in each of the last ten decennial censuses, is shown below :
	
	\begin{table}[h]
	\def\arraystretch{1.5}
	\begin{center}
	\begin{tabular}{|>{\centering}m{3cm}|>{\centering\arraybackslash}m{5cm}|}
	
	\hline
	Census year & Population (millions) \\
	\hline
	1911 & 252.0 \\
	1921 & 251.2 \\
	1931 & 278.9 \\
	1941 & 318.5 \\
	1951 & 361.0 \\
	1961 & 439.1 \\
	1971 & 547.0 \\
	1981 & 683.3 \\
	1991 & 846.3 \\
	2001 & 1028.6 \\
	\hline
	\end{tabular}
	\end{center}
	\end{table}
	
	
	
	Fit a logistic curve to the data. in case you find the fit to be unsatisfactory, suggest reasons for the same.
	
	
	
	
	
	
	
	
	
	
	
	
	
	
	
	
	
	
	
	
	
	
	\item Compute 
	\begin{enumerate}[(i)]
	\item $G.F.R.$,
	\item $S.F.R.$,
	\item $T.F.R.$ and
	\item the \textit{gross reproduction rate}, from the data given below :
	\end{enumerate}
	
	\begin{table}[h]
	\def\arraystretch{1.5}
	\begin{center}
	\begin{tabular}{|>{\centering}m{4cm}|c|c|}
	
	\hline
	\textit{Age-group of child bearing females} & \textit{Numer of women ('000)} & \textit{Total Births} \\
	\hline
	15 {\textemdash} 19 & 16.0 & 260 \\
	20 {\textemdash} 24 & 16.4 & 2244 \\
	25 {\textemdash} 29 & 15.8 & 1894 \\
	30 {\textemdash} 34 & 15.2 & 1320 \\
	35 {\textemdash} 39 & 14.8 & 916 \\
	40 {\textemdash} 44 & 15.0 & 280 \\
	45 {\textemdash} 49 & 14.5 & 145 \\
	\hline
	\end{tabular}
	\end{center}
	\end{table}
	
	Assume that the proportion of female births is 46.2 per cent.
	
	
	
	
	
	
	
	\newpage
	
	
	
	
	
	
	
	
	
	
	
	\item The following table gives the population of a country for the year 2001, together with the estimated number of births and deaths based on a special vital statistics enquiry conducted in the country. Calculate :
	\begin{enumerate}[(i)]
	\item crude death rates for the total population and for males and females,
	\item crude birth rate for the total population,
	\item general fertility rate,
	\item total fertility rate,
	\item gross reproduction rate, and
	\item net reproduction rate.
	\end{enumerate}
	
	\begin{table}[h]
	\def\arraystretch{1.5}
	\begin{center}
	\begin{tabular}{|c|c|c|c|c|c|c|c|}
	
	\hline
	\multirow{2}{*}{\textit {Age (l.b.d.)}} & \multicolumn{2}{c|}{\textit{Males}} & \multicolumn{2}{c|}{\textit{Females}} & \multicolumn{2}{c|}{\textit{Births}} & \multirow{2}{*}{\textit{Survival Rates}} \\
	\hhline{~------~}
	& \textit{Population} & \textit{Deaths} & \textit{Population} & \textit{Deaths} & \textit{Males} & \textit{Females} & \\
	\hline
	
	0 {\textemdash} 4 & 442532 & 18623 & 434980 & 17308 & & & \\
	5 {\textemdash} 9 & 419042 & 1809 & 416736 & 1709 & & & \\
	10 {\textemdash} 14 & 393543 & 984 & 384616 & 1638 & & & \\
	15 {\textemdash} 19 & 308269 & 1233 & 314056 & 1329 & 3578 & 3343 & 0.914 \\
	20 {\textemdash} 24 & 257852 & 1289 & 269340 & 1481 & 7293 & 6690 & 0.899 \\
	25 {\textemdash} 29 & 230629 & 1776 & 236187 & 1677 & 6775 & 6361 & 0.844 \\
	30 {\textemdash} 34 & 204188 & 1633 & 203477 & 1465 & 4233 & 4187 & 0.868 \\
	35 {\textemdash} 39 & 182270 & 1588 & 176534 & 1289 & 2999 & 2685 & 0.852 \\
	40 {\textemdash} 44 & 162509 & 1967 & 145037 & 1233 & 593 & 725 & 0.834 \\
	45 {\textemdash} 49 & 128784 & 2138 & 122946 & 1352 & 129 & 128 & 0.819 \\
	50 {\textemdash} 54 & 102971 & 1905 & 96589 & 1188 & & & \\
	55 {\textemdash} 59 & 80717 & 2478 & 78311 & 1605 & & & \\
	60 {\textemdash} 64 & 58899 & 3099 & 58142 & 1980 & & & \\
	65 {\textemdash} 69 & 37797 & 2428 & 39099 & 2468 & & & \\
	70 and above & 45099 & 5981 & 48866 & 7175 & & & \\
	\hline
	\end{tabular}
	\end{center}
	\end{table}
	
	
	
	
	
	
	
	
	
	
	
	
	
	\newpage
	
	
	
	
	
	
	
	
	


	\item Calculate the \textit{general fertility rate}, \textit{total fertility rate} and the \textit{gross reproduction rate} from the following data assuming that for every 100 girls 106 boys are born.
	
	\begin{table}[h]
	\def\arraystretch{1.5}
	\begin{center}
	\begin{tabular}{|c|c|c|}
	
	\hline
	\textit{Age of Women} & \textit{Number of Women} & \textit{Age-S.F.R. (per 1000)} \\
	\hline
	15 {\textemdash} 19 & 212619 & 98.0 \\
	20 {\textemdash} 24 & 198732 & 169.6 \\
	25 {\textemdash} 29 & 162800 & 158.2 \\
	30 {\textemdash} 34 & 145362 & 139.7 \\
	35 {\textemdash} 39 & 128109 & 98.6 \\
	40 {\textemdash} 44 & 106211 & 42.8 \\
	45 {\textemdash} 49 & 86753 & 16.9 \\
	\hline
	\end{tabular}
	\end{center}
	\end{table}
	
	
	
	
	
	
	
	
	
	\vspace{2cm}
	
	
	
	
	
	
	
	
	
	
	
	\item From the following data, calculate the \textit{gross reproduction rate} and the \textit{net reproduction rate}.
	
	\begin{table}[h]
	\def\arraystretch{1.5}
	\begin{center}
	\begin{tabular}{|c|>{\centering}m{7cm}|c|}
	 
	\hline
	\textit{Age-group} & \textit{Number of children born to 1,000 women passing through the age-group} & \textit{Mortality rate (per 1000)} \\
	\hline
	16 {\textemdash} 20 & 150 & 120 \\
	21 {\textemdash} 25 & 1500 & 180 \\
	26 {\textemdash} 30 & 2000 & 150 \\
	31 {\textemdash} 35 & 800 & 200 \\
	36 {\textemdash} 40 & 500 & 220 \\
	41 {\textemdash} 45 & 200 & 230 \\
	46 {\textemdash} 50 & 100 & 250 \\
	\hline
	\end{tabular}
	\end{center}
	\end{table}
	
	
	
	Sex ratio being \textit{males : females :: 52 : 48}.
	
	
	
	
	
	
	
	
	
	\newpage
	
	
	
	
	
	
	
	
	
	
	
	
	
	
	\item Given the following table for $l_x$, the number of rabbits living at age $x$, complete the life table for rabbits.
	
	\begin{table}[h]
	\def\arraystretch{1.5}
	\begin{center}
	\begin{tabular}{|>{\centering}m{1cm}|>{\centering}m{1cm}|>{\centering}m{1cm}|>{\centering}m{1cm}|>{\centering}m{1cm}|>{\centering}m{1cm}|>{\centering}m{1cm}|>{\centering\arraybackslash}m{1cm}|}
	
	\hline
	$x$ & 0 & 1 & 2 & 3 & 4 & 5 & 6 \\
	\hline
	$l_x$ & 100 & 90 & 80 & 75 & 60 & 30 & 0 \\
	\hline
	\end{tabular}
	\end{center}
	\end{table}
	
	$X$, $Y$, $Z$ are three rabbits of age 1, 2 and 3 years respectively. Find the probability that :
	\begin{enumerate}[(i)]
	\item at least one of them will be alive for one year more,
	\item $X$, $Y$, $Z$ will be alive for two years time,
	\item exactly one of the three is alive in two years, and 
	\item all will be dead in two years time.
	\end{enumerate}
	
	
	
	
	
	
	
	
	
	
	
	
	\vspace{2cm}
	
	
	
	
	
	
	
	
	
	
	\item The number of persons dying at age 75 is 476 and the complete expectation of life at 75 and 76 years are 3.92 and 3.66 years. Find the numbers living at ages 75 and 76.
	\end{enumerate}
\end{document}