\documentclass[11pt, a4paper]{article}


\usepackage[top=1 in, bottom = 1 in, left = 1 in, right = 1 in ]{geometry}

\usepackage{amsmath, amssymb, amsfonts}
\usepackage{hyperref}

\usepackage{polyglossia}

\usepackage{fontspec}

\usepackage{enumerate}

\usepackage{graphicx}

\usepackage{float}

\usepackage{tasks}

\usepackage{exsheets}

\usepackage{fontawesome5}

\usepackage{utfsym}

\usepackage{listing}

\usepackage{mathtools}

\SetupExSheets[question]{type=exam}

\newcommand\myperm[2][^n]{\prescript{#1\mkern-2.5mu}{}P_{#2}}
\newcommand\mycomb[2][^n]{\prescript{#1\mkern-0.5mu}{}C_{#2}}


\setdefaultlanguage{english}
\setotherlanguage{bengali}


\newfontfamily{\bengalifont}[Script=Bengali]{kalpurush.ttf}


\begin{document}

\tableofcontents
\pagebreak


	
\section{Logarithm}

\begin{enumerate}


%01
     \item Solve for x : $ \log_x 3 \cdot \log_{\frac{x}{81}} 3 = \log_{\frac{x}{729}} 3. $

%02
	\item \textbengali{যদি} 
		$ y= 10 ^ {\dfrac{1}{1-\log_{10} x}}, z= 10 ^ {\dfrac{1}{1-\log_{10} y}} $ 
		\textbengali{হয়, তবে প্রমাণ কর যে,}
		$ x= 10 ^ {\dfrac{1}{1-\log_{10} z}}.$
 
%03   
     \item \textbengali {যদি} $ x = \dfrac{e^y - e^{-y}}{e^y + e^{-y}} $ \textbengali {হয়, তবে দেখাও যে,} $ y = \dfrac{1}{2} \log_e \dfrac{1+x}{1-x}. $
     

%04
     \item \textbengali{মান নির্ণয় কর }:- $ \log_6 \sqrt{6\sqrt{6\sqrt{6 \cdots \cdots \infty}}}. $
     
%05
     \item \textbengali{প্রমাণ কর যে,} $ \log_{10} 2 > 0.3 $.
     
%06
     \item \textbengali {যদি} $ \dfrac{\log x}{ry - qz} = \dfrac{\log y}{pz - rx} = \dfrac{\log z}{qx - py} $ \textbengali{হয়, তবে প্রমাণ কর যে} $ x^p y^q z^r = 1 $.
     
%07
     \item $ \log_p x = a $, $ \log_q x = b $ \textbengali{হলে দেখাও যে,} $ \log_{\frac{p}{q}} x = \dfrac{ab}{b-a} $.
     
%08
     \item \textbengali{যদি} $ \log_a b = 10 $ \textbengali{ও} $ \log_{6a} 32b = 5 $ \textbengali{হয় তবে} $a$ \textbengali{ও} $b$ \textbengali{এর মান কত?}
     
%09
     \item $ x = \log_a bc$, $ y = \log_b ca $, $ z = \log_c ab $ \textbengali{হলে দেখাও যে}
     		\begin{enumerate}[(i)]
     		\item $ \dfrac{1}{x+1} + \dfrac{1}{y+1} + \dfrac{1}{z+1} = 1 $.
     		\item $ x+y+z = xyz - 2 $.
     		\end{enumerate}
     		
%10
     	\item If $ \log (x^2y^3) = a $ and $ \log \left(\dfrac{x}{y}\right) = b $, find $\log x $ and $\log y $ in terms of $a$ and $b$. 
     	
%11
     	\item Solve :- $\log_4 (x-1) = \log_2 (x-3) $.
     	
%12
     	\item Solve :- $ \log_{(2x+3)} \left( 6x^2 + 23x + 21 \right) + \log_{(3x+7)} \left( 4x^2 + 12x + 9 \right) = 4 $.
     	
%13
     	\item If $ \log_{10} 2 = 0.30103 $, $\log_{10} 3 = 0.47712$, and $\log_{10} 7 = 0.84510 $, find the values of \begin{enumerate}[(i)]
     		\item $ \log_{10} 45 $
     		\item $ \log_{10} 105 $
     	\end{enumerate}
     	
%14
     	\item Prove that, $ \log_2 10 - \log_8 125 = 1 $.
     	
%15
     	\item Show that, $ a^{\log_{a^2} x} \cdot b^{\log_{b^2} y} \cdot c^{\log_{c^2} z}  = \sqrt{xyz}$.
     	
%16
     	\item If $ \log_2 x + \log_4 x + \log_{16} x = \dfrac{21}{4} $, find the value of x.
     	
%17
     	\item Prove that, $ (yz)^{\log \dfrac{y}{z}} \cdot (zx)^{\log \dfrac{z}{x}} \cdot (xy)^{\log \dfrac{x}{y}} = 1 $.
     	
%18
     	\item Show that, $ \dfrac{1}{\log_a bc + 1} + \dfrac{1}{\log_b ca + 1} + \dfrac{1}{\log_c ab + 1} = 1 $.
     	
%19
     	\item Solve : $ \log_5 (5^{\frac{1}{x}} + 125) = \log_5 6 + 1 + \dfrac{1}{2x} $.
     	
%20
     	\item If $a>0$; $c>0$; $b=\sqrt{ac}$; $a$, $c$ and $ ac \neq1  $; $N>0$; prove that, \begin{center}
     	
     	
     $\dfrac{\log_a N}{\log_c N} = \dfrac{\log_a N - \log_b N}{\log_b N - \log_c N}$.
     \end{center}
     
%21
     \item If $ \dfrac{r}{r_1} + \log_e \dfrac{r_2}{r_1} = 1 $ and $ r_2 = er $, then show that, $ \dfrac{r_1}{r} \log_e \dfrac{r_1}{r} = 1 $.
     
%22
     \item If $ \dfrac{\log a}{y+z} = \dfrac{\log b}{z+x} = \dfrac{\log c}{x+y} $, then show that, $ \left( \dfrac{b}{c} \right)^x \cdot \left( \dfrac{c}{a} \right)^y \cdot \left( \dfrac{a}{b} \right)^z = 1 $.
     
%23
     \item Solve : $ x^{\log_{10} x} = 100x $.
     
%24
     \item Solve : $ 2\log_2 \log_2 x + \log_{\frac{1}{2}} \log_2 (2\sqrt{2}x) = 1 $.
     
%25
     \item Solve : $ 4^{\log_9 3} + 9^{\log_2 4} = 10^{\log_x 83} $.
     
%26
     \item If $ (\log_b a \cdot \log_c a - \log_a a) + (\log_c b \cdot \log_a b - \log_b b) + (\log_a c \cdot \log_b c - \log_c c) = 0 $, then show that, \begin{enumerate}[(i)]
     		\item $ a=b=c $.
     		\item $ abc = 1 $.
     \end{enumerate}
     
%27
     \item If $ x = 1 + \log_a (bc) $ , $ y = 1 + \log_b (ca) $, $ z = 1 + \log_c (ab) $, prove that, $ xy + yz + zx = xyz $.
     
%28
     \item Show that, $ \dfrac{\log_a x}{\log_{ab} x} = 1 + \log_a b $.
     
%29
     \item If the logarithm of $a^2$ to the base $b^3$ and the logarithm of $b^8$ to the base $a^{12}$ be equal, find the value of each logarithm.
     
%30
     \item Solve : $ \dfrac{1}{\log_x 10} + 2 = \dfrac{2}{\log_{0.5} 10} $ .   
     
%31
     \item Find the value of $ \log_3 2^{\log_4 3^{\log_5 4^{\log_6 5 \cdots \log_{1024} 1023}}} $ 	.
     
%32
     \item Find the value of $ \log_2 1^{\log_3 2^{\log_4 3 \cdots \infty}} $.
     	
%33
     	\item Solve :- $ x^{\log_2 x} + a^{\log_2 x} = 2a^2(a>1) $.
     	
%34
     	\item Prove that, $ a^{\log b} = b^{\log a} $.
     	
%35
     	\item If $ \dfrac{pq\log (pq)}{p+q} = \dfrac{qr\log (qr)}{q+r} = \dfrac{rp\log (rp)}{r+p} $, then prove that, $ p^p = q^q = r^r $.
     	
%36
     	\item If $ \log_{12} 27 = a $ then find the value of $ \log_{6} 16 $ in the terms of $a$.
     	
%37
     	\item If $ x = 10! $, find the value of $ \dfrac{1}{\log_2 x} + \dfrac{1}{\log_3 x} + \dfrac{1}{\log_4 x} + \cdots + \dfrac{1}{\log_{10} x} $.
     	
%38
     	\item Find the value of $ (25)^{\frac{1}{2} + \log_{\frac{1}{5}} 27+ \log_{25} 81} $.
     	
%39
     	\item $ 2\log_{10} x - \log_x (0.01) $ [$x>1$] \textbengali{রাশিটির ক্ষুদ্রতম মান কত}?
     	
%40
     	\item If $ 2 \log_8 N = P $, $ \log_2 2N = q $ and $ q - p = 4 $, find the value of $ N $.
     	
%41
     	\item If $ a = \log_3 5 $  \& $ b = \log_{17} 25 $, show that $ a>b $.
     	
%42
     	\item If $ x^2 + y^2 = z^2 $, prove that, $ \dfrac{1}{\log_{z-y} x} + \dfrac{1}{\log_{z+y} x} = (2+\sqrt{2}) (2-\sqrt{2}) $.
     	
%43
     	\item $ 5^{(2 - \log_5 2)} $ \textbengali{এর মান কত?}
     	
%44
     	\item Prove that, $ \log_a x \cdot \log_b y \cdot \log_c z = \log_b x \cdot \log_c y \cdot \log_a z$.
     	
%45
     	\item Prove that, $ \log (1^{\frac{1}{5}} + 32^{\frac{1}{5}} + 243^{\frac{1}{5}}) = \dfrac{1}{5} \left( \log 1 + \log 32 + \log 243\right) $.
     	
%46
     	\item Prove that, $ \log_a x + \log_{a^2} x^2 + \log_{a^3} x^3 + \log_{a^4} x^4  + \cdots + \log_{a^n}x^n = \log_a x^n$.
     	
%47
     	\item $ \log_3 \sqrt{6} + \log_3 \sqrt{\dfrac{2}{3}} - \log_3 \log_3 9 $ \textbengali{এর মান কত?}
     	
%48
     	\item If $ x + y = z $, prove that, $ \dfrac{1}{\log_{\sqrt{z} - \sqrt{y}} x} + \dfrac{1}{\log_{\sqrt{z} + \sqrt{y}} x} = 1 $.
     	
%49
     	\item Find the value of $ \log_2 \sqrt[4]{64\sqrt[3]{4^{(-1)} 8^{-\frac{4}{3}}}} $.
     	
%50
     	\item If $ x = \log_b a + \log_a b $, $ y = \log_c b + \log_b c $, $ z = \log_a c + \log_c a $; prove that, $ x^2 + y^2 + z^2 - 4 = xyz $.
     	
%51
     	\item If $ \dfrac{a(b+c-a)}{\log a} = \dfrac{b(c+a-b)}{\log b} = \dfrac{c(a+b-c)}{\log c} $, prove that, $ a^b \cdot b^a = b^c \cdot c^b = a^c \cdot c^a $.
     	
%52
     	\item If $ \log_{12} m = a $, $ \log_{18} m = b $, prove that, $ \log_3 2 = \dfrac{a-2b}{b-2a} $.
     	
%53
     	\item Solve:- $ \dfrac{\log_2 (x+4) + 1}{\log_{\sqrt{2}} (\sqrt{x+3} - \sqrt{x-3})} = 1 $.
  
%54
	\item Prove that, the value of $\log_{10} 3$ lies between $ \dfrac{1}{2} $ and $ \dfrac{2}{5} $.
	
%55
	\item Prove that, $ \dfrac{1}{\log_2 \pi} + \dfrac{1}{\log_6 \pi} > 2 $.
	
%56
	\item Solve:- $ \log_7 \log_5 ( \sqrt{x+5} + \sqrt{x} ) = 0 $.
	
%57
	\item Solve:- $ x + \log_{10} ( 1 + 2x ) = x\log_{10} 5 + \log_{10} 6 $.
	
%58
	\item If $ \log_{40} 4 = a $, $ \log_{40} 5 = b $, show that $ \log_{40} 16 = 4(1-a-b) $.
	\begin{center}
	\textbf{OR} 
	\end{center}
	If $ \log_{40} 4 = a $, $ \log_{40} 5 = b $, find the value of $ \log_{40} 16 $ in terms of $a$ \& $b$.
	
%59
	\item If \,  $ \log (a+b+c) = \log a + \log b + \log c $,then prove that,\\ \\ $ \log \left( \dfrac{2a}{1-a^2} + \dfrac{2b}{1-b^2} + \dfrac{2c}{1-c^2} \right) = \log \dfrac{2a}{1-a^2} + \log \dfrac{2b}{1-b^2} + \log \dfrac{2c}{1-c^2} $.
	
%60
	\item If \, $ b = \dfrac{c+a}{2} $ and $ y^2 = zx $, then prove that, 
	\begin{center}
	$ a^{(b-c)\log_a x} \cdot b^{(c-a)\log_b y} \cdot c^{(a-b)\log_c z} = 1 $.
	\end{center}
	
%61
	\item If $ 2 \log_m x = \log_l x + \log_n x $, show that $ \log n^2 = \log (ln) \cdot \log_l m $.
	
%62
	\item If $ b-a = c-b  $ and $ \dfrac{y}{x} = \dfrac{z}{y} $, prove that $ (b-c)\log x + (c-a)\log y + (a-b)\log z = 0 $.
	
%63
	\item If $x$, $y$, $z$ are in G.P., prove that $ \log_a x + \log_a z = \dfrac{2}{\log_y a} $ where $x$, $y$, $z$, $a$ $>$ $0$.
	
%64
	\item If $ \log_6 15 = a $, $ \log_{12} 18 = b $, $ \log_{25} 24 = c $, show that $ c = \dfrac{5-b}{2(ab+a-2b+1)} $.
	
%65
	\item If $ \log_{12} 18 = x $, $ \log_{24} 54 = y$ show that, $ xy + 5(x-y) = 1 $.
	
%66
	\item If $ 2 \log_{10} 2 = (2-a) $, show that, $ \log_{10} 5 = \dfrac{a}{2} $.
	  			 
	  			 
%67
	\item If $ (ax)^{\log a} = (bx)^{\log b} $, show that $x = \dfrac{1}{ab} $.
	
%68
	\item If $\log_{10} 2 = x$, $\log_{10} 3 = y$, show that $\log_{10} 45 = 2y - x + 1$.
	
%69
	\item If $\log_{10} 2 = x$, show that $\log_{8} 25 = \dfrac{2}{3} \left( \dfrac{1}{x} - 1 \right)$.
	
%70
	\item If $a^2 + b^2 = c^2$, show that $\log_{(c-b)} a + \log_{(c+b)} a = 2 \cdot \log_{(c+b)} a \cdot \log_{(c-b)} a$.


\end{enumerate}


\section{Geometry}

\begin{enumerate}
%01

	\item ABC 
			\textbengali{ও}  
			BDC 
			\textbengali{দুটি ত্রিভুজ একই ভূমি} 
			BC
		   \textbengali{-র একই পাশে অবস্থিত।} 
		   AB, AC, CD, BD 
		   \textbengali{বাহুগুলির মধ্যবিন্দু যথাক্রমে} 
		   P,Q,R,S.
		  \textbengali{প্রমাণ কর যে,} 
		     \begin{center}
		     
		      PQRS
		     \textbengali{ সামান্তরিকের ক্ষেত্রফল =}
		  $ \dfrac{1}{2} \left(\bigtriangleup BDC \sim \bigtriangleup ABC \right). $
		     \end{center}
		    
%02 
	\item ABCD, CDEF \textbengali{ও} EFGH \textbengali{হল তিনটি বর্গক্ষেত্র ।} AF \textbengali{ও} BH \textbengali{পরস্পরকে} O \textbengali{বিন্দুতে ছেদ করেছে। প্রমাণ কর যে,} $ \angle HOF = 45^{\circ}. $

%03	
	\item ABCD \textbengali{একটি বর্গক্ষেত্র।}  \textbengali{এর মধ্যে } P \textbengali{এমন একটি বিন্দু যেন}  PB = PC \textbengali{হয়।}   $ \angle PAD = 15^{\circ} $. \textbengali{প্রমাণ কর যে,} \\ PB = BC = PC. 

%04	
	\item ABCD \textbengali{আয়তক্ষেত্রের}  C \textbengali{বিন্দুগামী একটি বৃত্ত}  AB \textbengali{ও}  AD \textbengali{কে যথাক্রমে}  M \textbengali{ও} N \textbengali{বিন্দুতে ছেদ করে।}  MN \textbengali{জ্যা এর উপরে} C \textbengali{বিন্দু থেকে}  CP \textbengali{লম্ব।} \textbengali{প্রমাণ করতে হবে,} ABCD \textbengali{আয়তক্ষেত্রের ক্ষেত্রফল} = (CP)$^2$.

%05	
	\item $\bigtriangleup$ ABC \textbengali{একটি সূক্ষ্মকোণী ত্রিভুজ।}  $\angle BAC = 30^{\circ} $. H \textbengali{লম্ববিন্দু} ; M, BC \textbengali{বাহুর মধ্যবিন্দু।}  H, M \textbengali{যোগ করে} T \textbengali{বিন্দু পর্যন্ত এমনভাবে বাড়ানো হল যাতে}  HM = MT \textbengali{হয়। প্রমাণ করতে হবে,}  AT = 2BC. [INMO 1995]

%06	
	\item Fermat's Point

%07	
	\item $\bigtriangleup$ ABC \textbengali{এর} AD, BE \textbengali{ও} CF \textbengali{তিনটি মধ্যমা পরস্পরকে}  G \textbengali{বিন্দুতে ছেদ করেছে।}   \textbengali{প্রমাণ কর যে, }
	\begin{enumerate}[(i)]
	\item $8(BE)^2 + 8(CF)^2 - 4(AD)^2 = 9(BC)^2$
	\item $8(BE)^2 + 8(AD)^2 - 4(CF)^2 = 9(AB)^2$	
	\item $8(CF)^2 + 8(AD)^2 - 4(BE)^2 = 9(AC)^2$
	\end{enumerate}
	

%08	
	\item Apollonius' Theorem
	

%09
	\item ABC \textbengali{একটি সমদ্বিবাহু ত্রিভুজ।} A \textbengali{বিন্দুগামী} BC \textbengali{এর সমান্তরাল সরলরেখার ওপর} D \textbengali{একটি বিন্দু।} BCD \textbengali{অপর একটি ত্রিভুজ। প্রমাণ কর যে,} $ BD + CD > AB + AC $.
	
%10
	\item $ \bigtriangleup $ ABC \textbengali{এর} AD \textbengali{মধ্যমা।} $ \angle ADB = 45^{\circ} $ \textbengali{ও}  $ \angle ACB = 30^{\circ} $. $ \angle BAD = $ \textbengali{কত?} [RMO 2005]
	
%11
	\item ABC \textbengali{একটি সমকোণী ত্রিভুজ যার} $ \angle ABC = 90^{\circ} $. BCRS, ACXY, AQPB \textbengali{হল তিনটি বর্গক্ষেত্র যাদের প্রতিটি বাহু যথাক্রমে} $ a,c,b $. \textbengali{প্রমাণ করতে হবে,} $ (XR)^2 + (QY)^2 = 5(PS)^2 $.
	
%12
	\item $ \bigtriangleup $ ABC \textbengali{এর} S \textbengali{পরিকেন্দ্র}, O \textbengali{লম্ববিন্দু} , R \textbengali{পরিব্যাসার্ধ হলে প্রমাণ কর যে,} \\ $ (AB)^2 + (BC)^2 + (AC)^2 = 12R^2 - [ (OA)^2 + (OB)^2 + (OC)^2 ] $.
	
	\newpage
	
%13
	\item $ \bigtriangleup $ ABC \textbengali{এর} $ \angle A $, $ \angle B $, $ \angle C $ \textbengali{কোণের বিপরীত বাহু যথাক্রমে} $ a,b,c $ \textbengali{হলে ও } C \textbengali{বিন্দুগামী উচ্চতার দৈর্ঘ্য } $ h $ \textbengali{হলে প্রমাণ কর যে,} \begin{center}
	$ h = \dfrac{\sqrt{(a+b+c)(a+b-c)(b+c-a)(c+a-b)}}{2c} $.
	\end{center}
	
%14
	\item $ \bigtriangleup $ ABC \textbengali{এর} $ \angle A $, $ \angle B $, $ \angle C $ \textbengali{কোণের বিপরীত বাহু যথাক্রমে} $ a,b,c $ \textbengali{হলে ও } C \textbengali{বিন্দুগামী মধ্যমার দৈর্ঘ্য } $ x $ \textbengali{হলে প্রমাণ কর যে,}
\begin{center}
$ x = \dfrac{\sqrt{2a^2 + 2b^2 - c^2}}{2} $.
\end{center}

%15
	\item $ \bigtriangleup $ ABC \textbengali{এর} $ \angle B = 2\angle C $ \textbengali{হলে নিচের কোনটি সঠিক}?
	\begin{enumerate}[(i)]
		\item $ AC <2AB $.
		\item $ AC = 2AB $.
		\item $ AC > 2AB $.
	\end{enumerate}
	
%16
	\item \textbengali{ব্রহ্মগুপ্তের সূত্র - ত্রিজুজের পরিব্যাসার্ধ নির্ণয়}
	
%17
	\item $ ABCD $ \textbengali{সামান্তরিক}, $ BQ \perp AD $ \textbengali{হলে প্রমাণ কর যে,}  $ (AC)^2 - (BD)^2 = 4 (AQ) (AD) $.
	
%18
	\item $ \bigtriangleup ABC $ \textbengali{এর} $ \angle BAC $ \textbengali{এর সমিদ্বখণ্ডক} $ AE $, $ AD \perp AE $. Prove that, $ AB + AC < BD + DC $.
	
%19
	\item $ \bigtriangleup ABC $ \textbengali{এর} $ AB = 3AC $, $ \angle BAC $\textbengali{এর সমিদ্বখণ্ডক} $ AD, BC $ \textbengali{কে} $D$ \textbengali{বিন্দুতে ছেদ করেছে। বর্ধিত} $ AD $ \textbengali{এর ওপর } $BE$ \textbengali{লম্ব।প্রমাণ কর যে,} $ AD = DE $.
	
%20
	\item $ \bigtriangleup $ ABC \textbengali{এর} $ \angle A $, $ \angle B $, $ \angle C $ \textbengali{কোণের বিপরীত বাহু যথাক্রমে} $ a,b,c $ \textbengali{হলে ও } C \textbengali{বিন্দুগামী কোণসমিদ্বখণ্ডকের দৈর্ঘ্য } $ x $ \textbengali{হলে প্রমাণ কর যে,}
\begin{center}
$ x = \dfrac{\sqrt{ab(a+b+c)(a+b-c)}}{a+b} $.
\end{center}

%21
	\item \textbengali{কোনো বৃত্তের ব্যাস} $ AB $. $ CD \parallel AB, CD $ \textbengali{জ্যা।} $ P, AB $ \textbengali{এর ওপর যেকোনো বিন্দু। প্রমাণ কর যে,} $ (PA)^2 + (PB)^2 = (PC)^2 + (PD)^2 $.
	
%22	
	\item \textbengali{একটি সমকোণী ত্রিভুজের অতিভুজের বর্গ অন্য দুই বাহুর গুণফলের দ্বিগুণের সমান। }   \textbengali{ত্রিভুজটির  সূক্ষ্মকোণদ্বয়ের মান কত?}
	
%23
	\item Ceva's Theorem
	
%24
	\item $ \bigtriangleup ABC  $ \textbengali{এর} $AD$ \textbengali{মধ্যমা।} $ AB $ \textbengali{ও} $ AC $ \textbengali{বাহুর উপর দুটি বর্গক্ষেত্র যথাক্রমে} $ SABR $ \textbengali{ও} $ QACP $. \textbengali{প্রমাণ কর যে,} $ QS = 2AD $.
	
%25
	\item $ \bigtriangleup ABC  $ \textbengali{এর} $AD$, $BE$, $CF$ \textbengali{তিনটি মধ্যমা।} $ AB $, $BC$ \textbengali{ও} $ AC $ \textbengali{বাহুর উপর তিনটি বর্গক্ষেত্র যথাক্রমে} $ PABQ $, $ RBCS $, $ MACN. $ \textbengali{প্রমাণ কর যে,} 
	\begin{center}
	$ (PM)^2 + (QR)^2 + (SN)^2 = 4\Big[ (AD)^2 + (BE)^2 + (CF)^2 \Big] $.
	 \end{center}
	 
%26
	 \item $ \bigtriangleup ABC $ \textbengali{এর} $AD$, $BE$, $CF$ \textbengali{তিনটি মধ্যমা।} $ AB $, $BC$ \textbengali{ও} $ AC $ \textbengali{বাহুর উপর তিনটি বর্গক্ষেত্র যথাক্রমে} $ PABQ $, $ RBCS $, $ MACN. $ \textbengali{যাদের প্রতিটি বাহু যথাক্রমে} $a$, $b$, $c$. \textbengali{প্রমাণ কর যে,} 
	\begin{center}
	$ (PM)^2 + (QR)^2 + (SN)^2 = 3(a^2 + b^2 + c^2). $
	 \end{center}
%27	 
	 \item Stewart Law
	 
%28
	 \item $ \bigtriangleup ABC $ \textbengali{এর} $ \angle B = \angle C = 2\angle A $. \textbengali{প্রমাণ কর যে,}  $ \dfrac{BC}{AB} = \dfrac{\sqrt{5}-1}{2} $.
	 
%29
	 \item $ \bigtriangleup ABC $ \textbengali{সমকোণী ত্রিভুজের} $ BC $ \textbengali{অতিভুজ ও} $ AD \perp BC $. \textbengali{প্রমাণ কর যে,} $ BC + AD > AB + AC $.
	 
%30
	 \item $ \bigtriangleup ABC $ \textbengali{এর} $O$ \textbengali{লম্ববিন্দু}, $S$ \textbengali{পরিকেন্দ্র ও} $SD \perp BC $ \textbengali{হলে প্রমাণ কর যে } $ AO = 2SD  $.
	 
%31
	 \item Euler Line.
	 
%32
	 \item $ABCD$ \textbengali{সামান্তরিকের} $BC$ \textbengali{ও} $CD$ \textbengali{বাহুদ্বয়ের মধ্যবিন্দু যথাক্রমে} $E$\textbengali{ও} $F$. \textbengali{প্রমাণ কর যে,} $ \bigtriangleup AEF = \dfrac{3}{8} \square ABCD $.
	 
%33
	 \item Let $ABC$ be an acute-angled triangle and $CD$ be the altitude through $C$ If $AB = 8$ and $CD = 6$ find the distance between the midpoints of $AD$ and $BC$. [RMO 1993]
	 
%34
	 \item \textbengali{প্রমাণ কর যে, সমদ্বিবাহু ত্রিভুজের ভূমি সংলগ্ন কোণ দুটির অন্তর্দ্বিখণ্ডক ও ভূমির লম্ব সমদ্বিখণ্ডকটি সমবিন্দু হয়।} (\textbengali{ভূমিটি অসমান বাহু})
	 
%35
	 \item \textbengali{প্রমাণ কর যে, কোনো ত্রিভুজের ভূমি সংলগ্ন কোণ দুটির অন্তর্দ্বিখণ্ডক ও ভূমির লম্ব সমদ্বিখণ্ডকটি সমবিন্দু হলে ত্রিভুজটি সমদ্বিবাহু হয়।}
	 
%36
	 \item \textbengali{প্রমাণ কর যে, একটি ট্রাপিজিয়ামের সমান্তরাল বাহুদ্বয়ের মধ্যবিন্দু দুটির সংযোজক সরলরেখা কর্ণদ্বয়ের ছেদবিন্দুগামী।}
	 
%37
	 \item $ \bigtriangleup ABC  $ \textbengali{এর} $\angle BAC$ \textbengali{এর সমদ্বিখণ্ডক} $AO$. $D$,$BC$ \textbengali{-এর মধ্যবিন্দু।} $ BE \perp AO $, $ CF \perp AO $ \textbengali{হলে প্রমাণ কর যে,} $ DE = DF $.
	 
%38
	 \item $ \bigtriangleup ABC  $ \textbengali{এর} $ AB = AC $, $ \angle BAC = 20^{\circ} $, $ BC = AD $, $ D $ \textbengali{বিন্দু} $AB$ \textbengali{এর ওপর অবস্থিত হলে} $\angle ADC $ \textbengali{এর মান নির্ণয় কর।}
	 
%39
	 \item $ \bigtriangleup ABC  $ \textbengali{এর} $ \angle BAC $ \textbengali{-এর বহিঃসমদ্বিখণ্ডকের ওপর} $P$ \textbengali{যেকোনো একটি বিন্দু।} $BCP$ \textbengali{একটি ত্রিভুজ।} \textbengali{প্রমাণ কর যে,} $ PB + PC > AB + AC $.
	 
%40
	 \item $ \bigtriangleup ABC  $ 
	 	 \textbengali{এর}
	 	 $BC$, $CA$ 
	 	 \textbengali{ও}
	 	 $AB$
	 	 \textbengali{বাহুকে যথাক্রমে}
	 	 $X$,$Y$,$Z$
	 	 \textbengali{পর্যন্ত এরূপে বর্ধিত করা হল যাতে}
	 	 $BC = CX$, $CA = AY$, $AB = BZ$
	 	 \textbengali{হয়।}
	 	 $ \bigtriangleup ABC : \bigtriangleup XYZ $ =
	 	 \textbengali{কত?}
	 	 
	 
%41
	 \item $ \bigtriangleup ABC $ \textbengali{এর} $AD$, $BE$ \textbengali{ও} $CF$ \textbengali{তিনটি মধ্যমা}, $G$ \textbengali{ভরকেন্দ্র।} \textbengali{প্রমাণ কর যে}, 
	 \begin{center}
	  $ (AB)^2 + (BC)^2 + (AC)^2 = 3[ (AG)^2 + (BG)^2 + (CG)^2 ] $.
	  \end{center}
	 
%42
	 \item $ \bigtriangleup ABC $ \textbengali{এর} $O$ \textbengali{লম্ববিন্দু}, $H$ \textbengali{পরিকেন্দ্র।} $ AO = AH $ \textbengali{হলে প্রমাণ কর যে,} $ \angle BAC = 60^{\circ} $.
	 
%43
	 \item \textbengali{ত্রিভুজের অন্তর্ব্যাসার্ধ নির্ণয় ।}
	
%44
	 \item $ \bigtriangleup ABC $ \textbengali{এর} $ \angle B $ \textbengali{ও} $ \angle C $ \textbengali{এর  অন্তর্দ্বিখণ্ডকদ্বয় পরস্পরকে} $I$ \textbengali{বিন্দুতে ছেদ করে।} $I$ \textbengali{থেকে} $BC$, $CA$ \textbengali{ও} $AB$ \textbengali{বাহুর ওপর অঙ্কিত লম্ব তিনটি যথাক্রমে} $ID$, $IE$, $IF$. \textbengali{প্রমাণ কর যে,} $ID$ = $IE$ = $IF$.
	 
	 
%45
	 \item $ \bigtriangleup ABC $ \textbengali{এর} $ \angle A $ \textbengali{সমকোণ।} $AB$ \textbengali{এর উপর অঙ্কিত বর্গক্ষেত্র} $ABPQ$ \textbengali{ও} $BC$ \textbengali{এর উপর অঙ্কিত বর্গক্ষেত্র} $BCRS$ \textbengali{যারা} $ \bigtriangleup ABC $ \textbengali{এর বাইরের দিকে অবস্থিত।} $AM$, $BC$ \textbengali{এর উপর লম্ব। বর্ধিত} $AM$, $SR$ \textbengali{কে} $N$ \textbengali{বিন্দুতে ছেদ করে। প্রমাণ কর যে,} $ABPQ$ \textbengali{এর ক্ষেত্রফল} $=$ $BMNS$ \textbengali{এর ক্ষেত্রফল।}
	 
%46
	 \item Pappu's Extenstion on Pythagora's Theorem.
	 
%47
	 \item Let $ABC$ be a triangle with $AB = AC$ and $\angle BAC = 30^{\circ} $. Let $A'$ be the reflection of $A$ in the line $BC$; $B'$ be the reflection of $B$ in the line $CA$; $C'$ be the reflection of $C$ in the line $AB$. Show that, $A'$, $B'$, $C'$ form the vertices of an equilateral triangle. [RMO 1998]
	 
%48
	 \item $ABC$ \textbengali{স্থূলকোণী ত্রিভুজের} $\angle ABC = 100^{\circ}$, $\angle ACB = 65^{\circ} $. $M$ \textbengali{ও} $N$ \textbengali{হল যথাক্রমে} $AC$ \textbengali{ও} $AB$ \textbengali{বাহুর ওপর অবস্থিত এমন দুটি বিন্দু যাতে} $\angle ABM = 20^{\circ} $ \textbengali{ও} $\angle ACN = 10^{\circ}$ \textbengali{হয়। } $\angle MNC $ \textbengali{এর মান কত?}
	 
%49
	 \item Nine Point Circle ( \textbengali{নববিন্দু বৃত্ত} ) .
	 
%50
	 \item \textbengali{কোনো বৃত্তে} $2a$ \textbengali{ও} $2b$ \textbengali{দৈর্ঘ্য বিশিষ্ট দুটি জ্যা পরস্পরকে লম্বভাবে ছেদ করে। যদি কেন্দ্র থেকে ছেদবিন্দুর দূরত্ব } $c$ \textbengali{হয়, তাহলে প্রমাণ কর যে, বৃত্তের ব্যাসার্ধ} $r = \sqrt{\dfrac{a^2 + b^2 + c^2}{2}}$.
	 
%51
	 \item \textbengali{কোনো একটি বৃত্তকে দুটি এককেন্দ্রিক বৃত্তের সাহায্যে সমান} $3$ \textbengali{টি ভাগে বিভক্ত করা হল। ভিতর থেকে বাইরের দিকে তাদের ব্যাসার্ধ যথাক্রমে} $r_1$, $r_2$, $r_3$ \textbengali{হলে প্রমাণ কর যে,} $\dfrac{r_1}{\sqrt{1}} = \dfrac{r_2}{\sqrt{2}} = \dfrac{r_3}{\sqrt{3}}$.
	 
%52
	 \item \textbengali{একটি সমকোণী ত্রিভুজের সমকোণ সংলগ্ন বাহুদ্বয়ের দৈর্ঘ্য} $a$ \textbengali{একক ও } $b$ \textbengali{একক, সমকৌণিক বিন্দু থেকে অতিভুজের ওপর লম্বের দৈর্ঘ্য} $c$ \textbengali{একক হলে প্রমাণ কর যে,} $\dfrac{1}{a^2} + \dfrac{1}{b^2} = \dfrac{1}{c^2}$.
	 
%53
	 \item $\bigtriangleup$ ABC \textbengali{এর} $\angle$ A = $90^{\circ}$, AD $\perp$ BC, AB : AC = 12 : 5 \textbengali{হলে} BD : CD = \textbengali{কত} ?
	 
%54
	 \item $A$, $B$ \textbengali{ও} $C$ \textbengali{কেন্দ্র বিশিষ্ট তিনটি ভিন্ন ব্যাসার্ধের বৃত্ত পরস্পরকে বহিঃস্পর্শ করেছে। প্রথম ও দ্বিতীয় বৃত্তের ব্যাসার্ধের যোগফল} $5$ c.m., \textbengali{দ্বিতীয় ও তৃতীয় বৃত্তের} $6$ c.m. \textbengali{এবং তৃতীয় ও প্রথম বৃত্তের} $7$ c.m. \textbengali{প্রতিটি বৃত্তের ব্যাসার্ধের দৈর্ঘ্য কত ?}
	 
	 
%55
	 \item $ABC$ \textbengali{সূক্ষ্মকোণী ত্রিভুজে}  $\angle B = 50^{\circ}$, $\angle C$ \textbengali{এর অন্তর্দ্বিখণ্ডক} $AB$ \textbengali{বাহুকে} $D$ \textbengali{বিন্দুতে ছেদ করে।} $CD$ \textbengali{এর ওপর} $E$ \textbengali{এমন একটি বিন্দু নেওয়া হল যাতে} $AD = AE$ \textbengali{হয়।} $\angle CAE$ = \textbengali{কত ?}
	 
%56
	 \item $ABCD$ \textbengali{বর্গক্ষেত্রের ভেতরে} $P$ \textbengali{এমন একটি বিন্দু যাতে} $PA$ = 1 unit, $PB$ = 2 units \textbengali{ও} $PC$ = 3 units \textbengali{হয়।} $Q$ \textbengali{হল} $ABCD$ \textbengali{বর্গক্ষেত্রের বাইরে অবস্থিত একটি বিন্দু।} $\bigtriangleup BQC$ \textbengali{বর্গক্ষেত্রের বাইরে অবস্থিত এমন একটি ত্রিভুজ যার} $BQ$ = 2 units \textbengali{ও} $CQ$ = 1 unit.
	 	\begin{enumerate}[(i)]
	 		\item $PQ$ = ?
	 		\item $ \angle PQB $ = ?
	 		\item $ \angle PQC $ = ?
	 		\item $ \angle APB $ = ? 
	 		\begin{flushright}
	 			MTRP 2014
			\end{flushright}	 		 
	 	\end{enumerate}
	 	
	 
%57
	 \item $ABC$ \textbengali{সমবাহু ত্রিভুজের প্রতিটি বাহু} 2 c.m. $BC$ \textbengali{কে ব্যাস করে একটি বৃত্ত আঁকা হল। চিহ্নিত অংশের ক্ষেত্রফল কত ?} 		\begin{flushright}
	 MTRP 2017
\end{flushright}	  
	 
	 
	\begin{figure}[H]
	\centering
	\includegraphics[scale = 0.25]{IMG_20230407_181844_861}\\
	\end{figure}
	
%58
	\item \textbengali{প্রমাণ কর যে, একটি বৃত্তের কোন একটি বহিস্থ বিন্দুগামী ওই বৃত্তের দুটি স্পর্শক বৃত্তে যে স্পর্শ জ্যা উৎপন্ন করে, সেই স্পর্শ জ্যাটিকে ওই বৃত্তের কেন্দ্র ও সেই বহিস্থ বিন্দুগামী সরলরেখাংশ লম্বভাবে সমদ্বিখণ্ডিত করে।}
	 
%59
	\item $O$ \textbengali{কেন্দ্রীয় বৃত্তের} $AB$ \textbengali{একটি জ্যা।} $A$ \textbengali{ও} $B$ \textbengali{বিন্দুতে অঙ্কিত স্পর্শকদ্বয় পরস্পরকে} $P$ \textbengali{বিন্দুতে ছেদ করে।} $P$ \textbengali{বিন্দুগামী একটি বৃত্ত} $AB$ \textbengali{জ্যাকে} $A$ \textbengali{বিন্দুতে স্পর্শ করে। বর্ধিত} $OA$ \textbengali{দ্বিতীয় বৃত্তকে} $D$ \textbengali{বিন্দুতে ছেদ করে। প্রমাণ কর যে, } $OA = AD.$
	
%60
	\item $ABC$ \textbengali{ও} $DEF$ \textbengali{দুটি সদৃশকোণী  ত্রিভুজ।}  \textbengali{প্রমাণ কর যে, } $$\dfrac{\bigtriangleup ABC}{\bigtriangleup DEF} = \dfrac{BC^2}{EF^2} = \dfrac{AB^2}{DE^2} = \dfrac{AC^2}{DF^2}.$$
	
%61
	\item  Given $x:y:z = 3:4:5$. Find $x$, $y$, $z$.
	\begin{figure}[H]
	\centering
	\includegraphics[scale = 0.1]{IMG_20230723_153249_398}\\
	\end{figure}

%62
	\item \textbengali{কোনো বৃত্তস্থ চতুর্ভুজের বিপরীত বাহুগুলি বর্ধিত করার ফলে যে দুটি কোণ উৎপন্ন হয় তাদের অন্তঃসমদ্বিখণ্ডকদ্বয়ের মধ্যবর্তী কোণের মান কত ?}

%63
	\item \textbengali{প্রমাণ কর যে, কোনো ট্রাপিজিয়ামের সমান্তরাল বাহুদ্বয়ের সঙ্গে সমান্তরালভাবে অঙ্কিত একটি সরলরেখা তির্যক বাহুদ্বয়কে বা তাদের বর্ধিত অংশকে সমানুপাতে বিভক্ত করে।} 
	
%64
	\item \textbengali{দুটি বৃত্ত পরস্পরকে ছেদ করে একটি সাধারণ জ্যা উৎপন্ন করেছে।  সাধারণ জ্যায়ের যেকোনো একটি প্রান্তবিন্দুতে অঙ্কিত দুটি সরলরেখার প্রত্যেকটি বৃত্তদ্বয়কে যথাক্রমে} $A$, $B$ \textbengali{ও} $C$, $D$ \textbengali{বিন্দুতে ছেদ করেছে।} $AB$ \textbengali{ও} $CD$ \textbengali{সরলরেখাংশদ্বয় সাধারণ জ্যাটির সঙ্গে সমান কোণে নত। প্রমাণ কর যে, }$AB = CD$.
	
%65
	\item \textbengali{প্রমাণ কর যে, দুটি পরস্পরছেদী বৃত্তের ছেদবিন্দুদ্বয়ের যেকোনো একটি বিন্দুগামী সকল সরলরেখাগুলির মধ্যে যে সরলরেখাটি বৃত্তদ্বয়ের কেন্দ্রের সংযোজক সরলরেখাংশের সমান্তরাল সেটিই ক্ষুদ্রতম সরলরেখা।}
	
%66
	\item Two circles of radius $a$ and $b$ touch each other externally and they also touch a line. A circle of radius $c$ is inscribed in the region in between the circles and the line to touch the both of the circles. Show that, $\dfrac{1}{\sqrt{c}} = \dfrac{1}{\sqrt{a}} + \dfrac{1}{\sqrt{b}}$.
	
%67
	\item Two circles $C_1$ and $C_2$ of radii $a$ and $b$ touch each other externally and they both touch a unit circle $C$ internally. A circle $C_3$ of radius $r$ is inscribed to touch the circles $C_1$, $C_2$ externally and $C_3$ internally. Show that, $r = \dfrac{ab}{1-ab}$.
	
%68
	\item \textbengali{দুটি বৃত্ত পরস্পরকে} $P$ \textbengali{বিন্দুতে অন্তঃস্পর্শ করে।} $ABCD$ \textbengali{সরলরেখাংশ বহিঃস্থ বৃত্তকে} $A$, $D$ \textbengali{ও অন্তঃস্থ বৃত্তকে} $C$ \textbengali{ও} $B$ \textbengali{বিন্তুতে ছেদ করে।} $\angle APB = 20^{\circ}$ \textbengali{হলে} $\angle CPD = $ \textbengali{কত ?}
	
%69
	\item $ABCD$ \textbengali{রম্বসের} $C$ \textbengali{বিন্দুগামী একটি সরলরেখা} $AB$ \textbengali{ও বর্ধিত} $DA$ \textbengali{কে যথাক্রমে} $P$ \textbengali{ও} $Q$ \textbengali{বিন্দুতে ছেদ করে। প্রমাণ কর যে, }
	\begin{enumerate}[(i)]
		\item $\bigtriangleup APQ$, $\bigtriangleup BPC$, $\bigtriangleup DCQ$ \textbengali{প্রত্যেকে পরস্পরের সঙ্গে সদৃশকোণী।}
		\item $PB : DQ = AP^2 : AQ^2 $.
	\end{enumerate}
		

%70
	\item $O$ \textbengali{কেন্দ্রীয় একটি বৃত্তে ত্রিভুজ} $ABC$ \textbengali{অন্তর্লিখিত। বৃত্তের ওপর অবস্থিত } $X$ \textbengali{বিন্দু থেকে} $AB$ \textbengali{বাহুর ওপর} $XP$ \textbengali{লম্ব এবং} $AC$ \textbengali{বাহুর ওপর} $XQ$ \textbengali{লম্ব।} $BK$ \textbengali{বৃত্তটির একটি ব্যাস হলে প্রমাণ কর যে, } $PQ : BC = AX : 2R$  ,\textbengali{যেখানে বৃত্তটির \\ ব্যাসার্ধ = } $R$.
	
%71
	\item In an acute triangle $ABC$; points $D$, $E$, $F$ are located on the sides $BC$, $CA$, $AB$ respectively such that $$\dfrac{CD}{CE} = \dfrac{CA}{CB}, \dfrac{AE}{AF} = \dfrac{AB}{AC}, \dfrac{BF}{BD} = \dfrac{BC}{BA}.$$ Prove that, $AD$, $BE$, $CF$ are altitudes of $ABC$. [RMO 2002]
	
%72
	\item Let $ABC$ be a triangle in which $AB=AC$ and $\angle CAB = 90^{\circ}$. Suppose $M$ and $N$ are points on the hypotenuse $BC$ such that $BM^2 + CN^2 = MN^2.$ Prove that $\angle MAN = 45^{\circ}.$ [RMO 2003]
	
%73
	\item Let $ABC$ be a triangle in which $AB = AC$ and let $I$ be its in-centre. Suppose $BC = AB + AI$. Find $\angle BAC .$ [RMO 2009]
	
%74
	\item Let $AB$ be a triangle and let $BB_1$, $CC_1$ be respectively the bisectors of $\angle B$, $\angle C$ with $B_1$ on $AC$ and $C_1$ on $AB$. Let $E$, $F$ be the feet of perpendiculars drawn from $A$ onto $BB_1$, $CC_1$ respectively. Suppose $D$ is the point at which the incircle of $ABC$ touches $AB$. Prove that, $AD = EF$.
	
%75
	\item Consider in the plane a circle $\Gamma$ with center $O$ and a line $l$ not intersecting circle $\Gamma$. Prove that there is a unique point $Q$ on the perpendicular drawn from $O$ to the line $l$, such that for any point $P$ on the line $l$, $PQ$ represents the length of the tangent from $P$ to the circle $\Gamma$. [RMO 2004]
	
%76
	\item Euler's Theorem : \textbengali{কোনো ত্রিভুজের পরিব্যাসার্ধ} $R$, \textbengali{অন্তঃব্যাসার্ধ} $r$, \textbengali{পরিকেন্দ্র} $S$ \textbengali{ও অন্তঃকেন্দ্র} $I$ \textbengali{হলে প্রমাণ কর যে, } $SI^2 = R^2 - 2Rr$.
	
%77
	\item Euler's Theorem : \textbengali{কোনো ত্রিভুজের পরিব্যাসার্ধ} $R$, \textbengali{বহিঃব্যাসার্ধ} $r_1$, \textbengali{পরিকেন্দ্র} $S$ \textbengali{ও বহিঃকেন্দ্র} $I_1$ \textbengali{হলে প্রমাণ কর যে, } $S{I_1}^2 = R^2 + 2Rr_1$.
	
%78
	\item $\bigtriangleup ABC$ \textbengali{এর} $S$ \textbengali{পরিকেন্দ্র,} $I$ \textbengali{অন্তঃকেন্দ্র,} $O$ \textbengali{লম্ববিন্দু হলে প্রমাণ কর যে,} $\angle SAI = \angle IAO.$
	
%79
	\item $\bigtriangleup ABC$ \textbengali{এর} $\angle BAC = 90^{\circ}$, $AD \perp BC$. $\angle ABC$ \textbengali{ও} $\angle CAD$ \textbengali{কোণের  অন্তঃসমদ্বিখণ্ডকদ্বয় যথাক্রমে} $BE$ \textbengali{ও} $AF$. $BE$, $AD$ \textbengali{কে} $E$ \textbengali{বিন্দুতে ও} $AF$, $CD$ \textbengali{কে} $F$ \textbengali{বিন্দুতে ছেদ করে। প্রমাণ কর যে,} $EF \parallel AC.$
	
%80
	\item $\bigtriangleup ABC$ \textbengali{সমদ্বিবাহু যার} $AC=BC$. $BP \perp AC$, $PN \perp BC$. \textbengali{প্রমাণ কর যে,} $AB^2 = AN^2 + PN^2$.
	
%81
	\item $O$ \textbengali{কেন্দ্রীয় বৃত্তের} $AB$ \textbengali{একটি ব্যাস।} $AB$ \textbengali{ব্যাসের একই পাশে} $P$ \textbengali{ও} $Q$ \textbengali{দুটি এমন বিন্দু যে} $Q$, $AP$ \textbengali{চাপের মধ্যে ও} $P$, $BQ$ \textbengali{চাপের মধ্যে অবস্থিত। বর্ধিত} $AQ$ \textbengali{ও বর্ধিত} $BP$ \textbengali{পরস্পরকে} $Y$ \textbengali{বিন্দুতে এবং} $AP$ \textbengali{ও} $BQ$ \textbengali{পরস্পরকে} $X$ \textbengali{বিন্দুতে ছেদ করে। প্রমাণ কর যে,} $P$ \textbengali{ও} $Q$ \textbengali{বিন্দুতে অঙ্কিত স্পর্শকদ্বয়} $XY$ \textbengali{এর মধ্যবিন্দুগামী।}
	
%82
	\item \textbengali{প্রমাণ কর যে, কোনো ত্রিভুজের পরিব্যাসার্ধ তার বাহুগুলির মধ্যবিন্দু গুলির সংযোজক সরলরেখাংশগুলি দ্বারা গঠিত ত্রিভুজের পরিব্যাসার্ধের দ্বিগুণ।}
	
%83
	\item $AB$ \textbengali{সরলরেখাংশের} $A$ \textbengali{ও} $B$ \textbengali{বিন্দুতে যথাক্রমে} $RA$ \textbengali{ও} $QB$ \textbengali{লম্ব।} $AQ$ \textbengali{ও} $BR$ \textbengali{পরস্পরকে} $O$ \textbengali{বিন্দুতে ছেদ করে।} $OT \perp AB$. \textbengali{প্রমাণ কর যে,} $OT$, $\angle QTR$ \textbengali{কে সমদ্বিখণ্ডিত করে।}
	
%84
	\item $ABCD$ \textbengali{ট্রাপিজিয়ামের} $AD \parallel BC$. \textbengali{কর্ণদ্বয়} $AC$ \textbengali{ও} $BD$ \textbengali{এর ছেদবিন্দু} $F$. $F$ \textbengali{বিন্দুগামী} $AD$ \textbengali{এর সমান্তরাল সরলরেখা} $AB$ \textbengali{ও} $CD$ \textbengali{কে যথাক্রমে} $E$ \textbengali{ও} $G$ \textbengali{বিন্দুতে ছেদ করেছে। প্রমাণ কর যে,} $EF = FG$.
	
%85
	\item $ABCD$ \textbengali{একটি সামান্তরিক। প্রমাণ কর যে,} $AB^2 + BC^2 + CA^2 + AD^2 = AC^2 + BD^2$.
	
%86
	\item $ABCD$ \textbengali{বর্গক্ষেত্রের} $AB$, $BC$, $CD$ \textbengali{ও} $DA$ \textbengali{বাহুগুলির মধ্যবিন্দুগুলি হল যথাক্রমে} $E$, $F$, $G$ \textbengali{ও} $H$. $AF$, $CE$ \textbengali{পরস্পরকে} $P$ \textbengali{এবং} $AG$ \textbengali{ও} $CH$ \textbengali{পরস্পরকে} $Q$ \textbengali{বিন্দুতে ছেদ করলে প্রমাণ কর যে,} $APCQ$ \textbengali{একটি রম্বস।}
	
%87
	\item \textbf{Morley's Theorem} : The points of intersection of the adjacent trisectors of the angles of any triangle form the vertices of an equilateral triangle.
	
%88
	\item \textbengali{একটি বৃত্তে} $AB$ \textbengali{ও} $CD$ \textbengali{হল দুটি পরস্পর লম্বভাবে অবস্থিত ব্যাস। বৃত্তের ওপর অবস্থিত} $P$ \textbengali{একটি যেকোনো বিন্দু। প্রমাণ কর যে,} $4 \bigtriangleup PCD = PA^2 \sim PB^2$.
	
%89
	\item $ABCD$ \textbengali{চতুর্ভুজের} $AC$ \textbengali{ও} $BD$ \textbengali{কর্ণদ্বয় পরস্পরকে} $O$ \textbengali{বিন্দুতে ছেদ করে। একই সমতলে অবস্থিত একটি} $\bigtriangleup PQR$ \textbengali{এর} $PQ$ \textbengali{ও} $PR$ \textbengali{বাহুদ্বয় যথাক্রমে} $BD$ \textbengali{ও} $AC$ \textbengali{এর সঙ্গে সমান ও সমান্তরাল। প্রমাণ কর যে,} $ABCD$ \textbengali{চতুর্ভুজের ক্ষেত্রফল} $=$ $\bigtriangleup PQR$ \textbengali{এর ক্ষেত্রফল।}
	
%90
	\item $ABCD$ \textbengali{চতুর্ভুজের} $AB$ \textbengali{ও} $CD$ \textbengali{বাহুর ওপর যথাক্রমে অবস্থিত} $E$,$F$ \textbengali{এবং} $G$,$H$ \textbengali{বিন্দুগুলি বাহুদ্বয়কে সমত্রিখণ্ডিত করে। প্রমাণ কর যে,} $EFGH$ \textbengali{চতুর্ভুজের ক্ষেত্রফল} $= \dfrac{1}{2}$ $\big( AEHD$ \textbengali{চতুর্ভুজের ক্ষেত্রফল} $+$ $BCGF$ \textbengali{চতুর্ভুজের ক্ষেত্রফল} $\big)$.
	
%91
	\item $P$ \& $Q$ are two points on $BC$ of $\bigtriangleup ABC$ such that $BP = QC$. If the bisector of $\angle B$ meets $AP$, $AQ$ \& $AC$ respectively at $X$, $Y$ and $Z$, show that, $\dfrac{PX}{AX} + \dfrac{QY}{AY} = \dfrac{CZ}{AZ}$.
	
%92
	\item $M$ \textbengali{ও} $N$ \textbengali{কেন্দ্রীয় দুটি বৃত্ত পরস্পরকে} $A$ \textbengali{ও} $B$ \textbengali{বিন্দুতে ছেদ করেছে।} $PQ$ \textbengali{ও} $RS$ \textbengali{হল বৃত্তদ্বয়ের সরল সাধারণ স্পর্শকদ্বয়।} \textbengali{বর্ধিত} $BA$, $PQ$ \textbengali{কে} $D$ \textbengali{বিন্দুতে ছেদ করে। প্রমাণ কর যে, } $PQ^2 + AB^2 = CD^2$.
	
%93
	\item Let $\Gamma$ be a circle with center $O$ and $P$ be any point on its plane. Then show that, the power of $P$ w.r.t. $\Gamma$ is $OP^2 - R^2$ where $R$ is the radius of $\Gamma$.
	
%94
	\item $O$ \textbengali{কেন্দ্রীয় একটি বৃত্তে} $AB$ \textbengali{ও} $BC$ \textbengali{দুটি জ্যা।} $AB$ \textbengali{এর ওপর অবস্থিত} $D$ \textbengali{এমন একটি বিন্দু যাতে} $\angle DCB = 40^{\circ}$ \textbengali{হয়।} $OC$ \textbengali{ব্যাসার্ধটি} $\angle DBC$ \textbengali{কোণের সমদ্বিখণ্ডক।} $\angle ABC = 30^{\circ}$. $\angle CDO = $ \textbengali{কত ?}
	
%95
	\item $ABCD$ \textbengali{সামান্তরিকের} $AB$ \textbengali{বাহুর সমান্তরাল একটি সরলরেখা} $QP$. $AP$, $BQ$ \textbengali{পরস্পরকে} $R$ \textbengali{এবং} $CQ$, $DP$ \textbengali{পরস্পরকে} $S$ \textbengali{বিন্দুতে ছেদ করেছে। প্রমাণ কর যে,} $RS \parallel AD$.
	
%96
	\item $ABCD$ \textbengali{একটি বৃত্তস্থ চতুর্ভুজ যার} $AB > CD$, $AD > BC$. $P$ \textbengali{এবং} $Q$ \textbengali{হল যথাক্রমে} $AB$ \textbengali{ও} $AD$ \textbengali{এর ওপর অবস্থিত এমন দুটি বিন্দু যে} $BP = CD$ \textbengali{ও} $DQ = BC$ \textbengali{হয়।} $M$, $PQ$ \textbengali{এর মধ্যবিন্দু। প্রমাণ কর যে,} $\angle BMD = 90^{\circ}$.
	
%97
	\item \textbengali{জ্যামিতিক উপায়ে প্রমাণ কর যে,} $3 < \pi < 4$.
	
%98
	\item $ABC$ is an isosceles triangle where $\angle A = 20^{\circ}$, $AB = AC$. $D$ \& $E$ are points on $AB$ \& $AC$ respectively such that $\angle BCD = 60^{\circ}$ \& $\angle CBE = 70^{\circ}$. Find $\angle BED$.
	
	
%99
	\item $\bigtriangleup ABC$ \textbengali{এর তিনটি মধ্যমা} $AD$, $BE$, $CF$ \textbengali{হলে দেখাও যে,} $$3(AB^2 + BC^2 + CA^2) = 4(AD^2 + BE^2 + CF^2)$$.
	
%100
	\item \textbengali{কোনো বৃত্তের} $AC$ \textbengali{ও} $BD$ \textbengali{দুটি জ্যা পরস্পরকে} $O$ \textbengali{বিন্দুতে ছেদ করেছে।} $A$ \textbengali{ও} $B$ \textbengali{বিন্দুতে অঙ্কিত স্পর্শক দুটি পরস্পরকে} $P$ \textbengali{বিন্দুতে এবং} $C$ \textbengali{ও} $D$ \textbengali{বিন্দুতে অঙ্কিত স্পর্শক দুটি পরস্পরকে} $Q$ \textbengali{বিন্দুতে ছেদ করলে প্রমাণ কর যে,} $\angle P + \angle Q = 2 \angle BOC$.
	

%101
	\item $\bigtriangleup ABC$ \textbengali{এর} $\angle C = 90^{\circ}$. $\bigtriangleup ABC$ \textbengali{এর অন্তঃবৃত্ত} $AB$, $BC$ \textbengali{ও} $CA$ \textbengali{কে যথাক্রমে} $F$, $D$, $E$ \textbengali{বিন্দুতে ছেদ করে।} $AD$ \textbengali{অন্তঃবৃত্তকে} $P$ \textbengali{বিন্দুতে ছেদ করে।} $\angle BPC = 90^{\circ}.$ \textbengali{দেখাও যে,} $AE + AP = PD$.
	
%102
	\item Let $ABC$ be a triangle, $AD$ the altitude through $A$ and $AK$ the circumdiameter through $A$. Then show that, $\angle DAK = \angle B - \angle C$. Further show that, the angular bisector $AX$ of $\angle A$ bisects $\angle DAK$.
	
%103
	\item If the internal bisector of $\angle A$ of $\bigtriangleup ABC$ meets $BC$ at $X$, then show that the difference between $\angle AXB$ and $\angle AXC$ is the same as the difference between $\angle B$ and $\angle C$.
	
%104
	\item If $m_a$, $m_b$, $m_c$ are the lengths of the medians of $\bigtriangleup ABC$ through $A$, $B$, $C$ then prove that,
		\begin{enumerate}[(i)]
			\item $2m_a^2 = b^2 + c^2 - \dfrac{a^2}{2}$
			
			\item $2m_b^2 = c^2 + a^2 - \dfrac{b^2}{2}$
		
			\item $2m_c^2 = a^2 + b^2 - \dfrac{c^2}{2}$
		\end{enumerate}
		
%105
	\item Prove that, $m_a ^2 + m_b^2 + m_c^2 = \dfrac{3}{4}\left( a^2 + b^2 + c^2 \right)$.
	
%106
	\item Prove that, $GA^2 + GB^2 + GC^2 = \dfrac{1}{3} \left(a^2 + b^2 + c^2 \right)$ where $G$ is the centroid of any triangle $\bigtriangleup ABC$.
	
%107
	\item If $P$ is any point in the plane of $\bigtriangleup ABC$, then $PA^2 + PB^2 + PC^2 = GA^2 + GB^2 + GC^2 + 3PG^2$, where $G$ is the centroid of $\bigtriangleup ABC$.
	
%108
	\item If $G$ is the centroid, $R$ is the circumradius and $S$ is the circumcenter of $\bigtriangleup ABC$, show that, $$SG^2 = R^2 - \dfrac{1}{9} \left( a^2 + b^2 + c^2 \right).$$
	
%109
	\item The incenter $I$ and the excenter $I_a$ opposite to $A$ divide the bisector $AU$ harmonically, where $U$ is the point of intersection of the internal bisector of $\angle A$ and $BC$.
	
%110
	\item In a quadrilateral $ABCD$, the diagonals $AD$ and $BC$ meet at $O$. If it is given that $OA = OC$ and $OB = OD$, prove that, $BC = AD$ and that $\angle ACB = \angle CAD$.
	
%111
	\item In $\bigtriangleup ABC$, $\angle A = 60^{\circ}$, $AB > AC$, $O$ is the circumcenter and $H$ is the orthocenter. $M$, $N$ are points on the line segments $BH$ and $HF$ respectively such that $BM = CN$. Determine the value of $\dfrac{MH + NH}{OH}$.

%112
	\item In the acute angled triangle $ABC$, let $D$, $E$, $F$ be the feet of the altitudes through $A$, $B$, $C$ respectively and $H$ be the orthocenter of $\bigtriangleup ABC$. Prove that, $\dfrac{AH}{AD} + \dfrac{BH}{BE} + \dfrac{CH}{CF} = 2$.
	
%113
	\item \textbengali{কোনো সমকোণী ত্রিভুজের সমকোণ সংলগ্ন দুটি বাহুর একটি অপরটির দ্বিগুণ হলে প্রমাণ কর যে, উহার একটি কোণ} $30^{\circ}$ \textbengali{এর কম হবে।}
	
%114
	\item \textbengali{একটি সমকোণী ত্রিভুজের একটি সূক্ষ্মকোণ} $15^{\circ}$ \textbengali{ও অতিভুজ} $x$. \textbengali{ত্রিভুজটির ক্ষেত্রফল কত ?}
	
%115
	\item $P$ is a point on the minor arc $AB$ of the circumcircle of the square $ABC$. Prove that, $\dfrac{PA + PC}{PB+PD} = \dfrac{PD}{PC}$.
	
%116
	\item \textbf{[Langley's Problem]} $ABC$ \textbengali{সমদ্বিবাহু ত্রিভুজে} $\angle A = 20^{\circ}$, $AB = AC$. $AB$ \textbengali{ও} $AC$ \textbengali{এর ওপরে যথাক্রমে} $E$ \textbengali{ও} $D$ \textbengali{দুটি বিন্দু যাতে} $\angle DBE = 20^{\circ}$ \textbengali{ও} $\angle DCE = 30^{\circ}$ \textbengali{হয়।} $\angle BDE$ \textbengali{কত ?}
	
%117
	\item The sides $BC$, $CA$ and $AB$ of a triangle $ABC$ are extended to the points $C'$, $A'$, and $B'$ as twice of their corresponding lengths. Find the ratio of the areas of $\bigtriangleup A'B'C'$ and $\bigtriangleup ABC$.
	
%118
	\item Fifteen distinct points are designated on $\bigtriangleup ABC :$, the 3 vertices $A$, $B$, $C$; 3 other points on side $\overline{AB}$; 4 other points on side $\overline{BC}$; and 5 other points on side $\overline{CA}$. Find the number of triangles with positive area whose vertices are among these 15 points.
	
%119
	\item Let $ABCD$ be a square and let $E$ and $F$ be points on $\overline{AB}$ and $\overline{BC}$ respectively. The line through $E$ parallel to $\overline{BC}$ and the line through $F$ parallel to $\overline{AB}$ divide $ABCD$ into two squares and two non-square rectangles. The sum of the areas of the two squares is $\dfrac{9}{10}$ of the area of square $ABCD$. Find $\dfrac{AE}{EB} + \dfrac{EB}{AE}$.

%120
	\item $H$ is the orthocenter of a triangle $ABC$. Prove that, reflection of $H$ w.r.t. $BC$ lies on the circumcenter of $\bigtriangleup ABC$.
	
%121
	\item $H$ is the orthocenter of $\bigtriangleup ABC$. $A'$, $B'$, $C'$ are respectively the reflections of $H$ w.r.t. $BC$, $CA$ and $AB$. Prove that, $H$ is the incenter of $\bigtriangleup A'B'C'$.
	
%122
	\item Prove that, $r_1 = \dfrac{\bigtriangleup}{S-a}$, $r_2 = \dfrac{\bigtriangleup}{S-b}$, $r_3 = \dfrac{\bigtriangleup}{S-c}$.
	
%123
	\item Show that, $\dfrac{1}{r} = \dfrac{1}{r_1} + \dfrac{1}{r_2} + \dfrac{1}{r_3}$.
	
%124
	\item $AD$, $BE$, $CF$ are three cevians of $\bigtriangleup ABC$; $D$, $E$, $F$ being on $BC$, $AC$, $AB$ respectively. $DP$, $EQ$, $FR$ are three cevians of $\bigtriangleup DEF$; $P$, $Q$, $R$ being on $EF$, $DF$, $DE$ respectively. Prove that, $AP$, $BQ$, $CR$ are concurrent.
	
%125
	\item $ABC$ is a right-angled triangle where $\angle B = 90^{\circ}$. $BD \perp AC$. $BE$ is the internal angle bisector of $\angle ABD$. A line is drawn through $C$ and parallel to $BE$. Extended $BD$ intersects this line at $F$. Extended $FE$ intersects $AB$ at $X$. Prove that, $AX = BX$.
	
%126 
	\item $D$, $E$, $F$ are points on sides $BC$, $CA$, $AB$ respectively of $\bigtriangleup ABC$ such that $BD : DC = CE : EA = AF : FB = 1 : X$. $P$, $Q$, $R$ are points of itersections of $(AD, CF);(BE, AD);(BE,CF)$ respectively. Find $[\bigtriangleup PQR] : [\bigtriangleup ABC]$; where $[\bigtriangleup XYZ]$ denotes the area of $\bigtriangleup XYZ$.
	
%127
	\item \textbf{[Desgrate's Theorem]} $ABCD$ is a quadrilateral. Diagonal $DB$ is extended and $O$ is a point taken on it. One among two lines from $O$ intersects $AB$, $AD$ at $P$, $Q$ and the other intersects $BC$, $CD$ at $R$, $S$. Extended $RP$ and extended $SQ$ intersect at $X$. Prove that, $X$, $A$, $C$ are collinear.
	
%128
	\item Solve for the radius $r$ in terms of $a$, $b$, $c$ in the following figure.
	\begin{figure}[H]
	\centering
	\includegraphics[scale=0.1]{IMG_20231026_015712_749}
	\end{figure}
	
%129
	\item In a semi-circle of radius $r$ with $AD$ as diameter; $B$, $C$, $D$ are points on the semi-circle such that $AB = BC = \dfrac{r}{2}$ and $CD = x$. Find $x:r$.
	
%130
	\item $H$ is the orthocenter of $\bigtriangleup ABC$. $A_1$, $B_1$, $C_1$ are the circumcenteres of $\bigtriangleup BCH$, $\bigtriangleup ACH$, $\bigtriangleup ABH$ respectively. $R_1$, $R_2$, $R_3$ are the circumradii of $\bigtriangleup BCH$, $\bigtriangleup ACH$, $\bigtriangleup ABH$ respectively. $R$ be the circumradius of $\bigtriangleup ABC$. Prove that, 
	\begin{enumerate}[(i)]
		\item $\bigtriangleup ABC \cong \bigtriangleup A_1B_1C_1$
		\item $R = R_1 = R_2 = R_3$
	
	\end{enumerate}
	We denote the circumcircles of $\bigtriangleup ABC$, $\bigtriangleup BCH$, $\bigtriangleup ACH$, $\bigtriangleup ABH$ as $\lambda$, $\lambda_1$, $\lambda_2$, $\lambda_3$ respectively. Then establish that $\lambda_1$, $\lambda_2$, $\lambda_3$ are reflections of $\lambda$ w.r.t. $BC$, $AC$, $AB$ respectively. And also show tha, the nine-point circle of $\bigtriangleup ABC$ is same as that of $\bigtriangleup A_1B_1C_1$.
	
%131
	\item In a triangle $ABC$, $D$ and $E$ are points of $AB$ and $AC$ respectively such that $AD = AH$ and $AE = AO$. $H$ and $O$ are the orthocenter and circumcenter of $\bigtriangleup ABC$ respectively. Prove that, $\bigtriangleup ADE$ is isosceles.
	
%132
	\item $E$, $F$ are two points on $AC$, $AB$ respectively such that $EF \parallel BC$. Suppose $BE$, $CF$ intersect at $P$. Show that, $AP$ is the median through the vertex $A$.
	
%133
	\item In the following figure, $AE = 5$, $AD = 4$, $\angle B = 90^{\circ}$, radius of the drawn circle = 6. Find $BC$.
	\begin{figure}[h]
	\centering
	\includegraphics[scale=0.15]{IMG_20231102_173356_826}
	\end{figure}

%134
	\item $L$, $M$, $N$ are the midpoints of sides $BC$, $CA$, $AB$ respectively of $\bigtriangleup ABC$. Extended $LN$ touches the tangent to the circumcircle of $\bigtriangleup ABC$ at $Q$ and extended $LM$ touches the said tangent at $P$. Prove that, $BQ \parallel PC$.
	
%135
	\item In $\bigtriangleup ABC$, $AB = 2$, $AC = \sqrt{5} + 1$, $\angle A = 54^{\circ}$. $AC$ is extended to $D$ such that $CD = \sqrt{5} - 1$. $M$ is the midpoint of $BD$. Find $\angle ACM$.
\end{enumerate}
		     
		  
		  
		  
		  
		  
		  
		  
		  
		  
		  
		  
		  
		  
		  
		  
		  
		  
		  
		  
		  
		  
		  
		  
		  
		  
		  
		     

\section{Miscellaneous}

\begin{enumerate}

%01

	\item \textbengali{যদি} 
	$ ab^2 + bc^2 + ca^2 = 0 $
	\textbengali{হয়  যখন} $ a,b,c \neq 0 $, 
	\textbengali{তবে} $ \left( \dfrac{a}{b} + \dfrac{b}{c} \right) + \left( \dfrac{b}{c} + \dfrac{c}{a} \right) + \left( \dfrac{c}{a} + \dfrac{a}{b} \right) +1 $
	\textbengali{এর মান কত?}

%02	
	\item $ 0<a<1 $ \textbengali{অর্থাৎ} $a$ \textbengali{সংখ্যাটি} $0$ \textbengali{ও} $1$ \textbengali{এর মধ্যে অবস্থিত হলে কোনটি সঠিক?}  \begin{enumerate}[A.]
	\item $a^2 < a$
	\item $a^2 = -a$
	\item $a^2 \geq a$
	\item $a^2 \geq 1$
	\end{enumerate}
	
%03
	\item \textbengali{শ্রীধর আচার্যের সূত্র}
	
%04
	\item If $ xyz = 1 $, show that, $ \left( x + \dfrac{1}{x} \right)^2 + \left( y + \dfrac{1}{y} \right)^2 + \left( z + \dfrac{1}{z} \right)^2 = 4 + \left( x + \dfrac{1}{x} \right) \left( y + \dfrac{1}{y} \right) \left( z + \dfrac{1}{z} \right) $.
	
%05
	\item $ \dfrac{a}{b-c} + \dfrac{b}{c-a} + \dfrac{c}{a-b} = 0 $ \textbengali{হলে} $ \dfrac{a}{(b-c)^2} + \dfrac{b}{(c-a)^2} + \dfrac{c}{(a-b)^2} $ \textbengali{এর মান নির্ণয় কর।}
	
%06
	\item $ a + b + c = 0 $ \textbengali{হলে} $ \left( \dfrac{a}{b-c} + \dfrac{b}{c-a} + \dfrac{c}{a-b} \right) \left( \dfrac{a-b}{c} + \dfrac{b-c}{a} + \dfrac{c-a}{b} \right) $ \textbengali{এর মান নির্ণয় কর।}
	
%07
	\item $ p(x+y)^2 = 5 $, $ q(x-y)^2 = 3 $ \textbengali{হলে} $ p^2(x+y)^2 + 4pqxy - q^2(x-y)^2  $ \textbengali{এর মান} $p$ \textbengali{ও} $q$ \textbengali{এর মাধ্যমে নির্ণয় কর।}
	
%08
	\item If $ x+y+z = 6 $, $ xy + yz + zx = 9 $, show that, $ \dfrac{1}{1-x} + \dfrac{1}{1-y} + \dfrac{1}{1-z} = 0 $.
	
%09
	\item $ \dfrac{x}{a-x} + \dfrac{y}{b-y} + \dfrac{z}{c-z} = 0 $ \textbengali{হলে} $ \dfrac{a}{a-x} + \dfrac{b}{b-y} + \dfrac{c}{c-z}  $ \textbengali{এর মান নির্ণয় কর।}
	
%10
	\item $k+l+m = 1$, $3(kl+lm+mk) = 1$ \textbengali{হলে} $k+l-2m$ \textbengali{এর মান কত?}
	
%11
	\item $x^2 + y^2 + z^2 = 6x - 8y - 25$ \textbengali{হলে} $x+y+z$ \textbengali{এর মান কত?}
	
%12
	\item $ \dfrac{x}{x-1} + \dfrac{y}{y-1} + \dfrac{z}{z-1} = 0 $ \textbengali{হলে} $ \dfrac{1}{1-x} + \dfrac{1}{1-y} + \dfrac{1}{1-z} $ \textbengali{এর মান কত?}
	
%13
	\item $ a+b+c = 1 = 3(ab+bc+ca) $ \textbengali{এবং} $abc = \dfrac{1}{27}$ \textbengali{হলে}
	\begin{enumerate}[(i)]
		\item $a$, $b$, $c$ \textbengali{এর মান কত?}
		\item $\dfrac{a}{b+c} + \dfrac{b}{c+a} + \dfrac{c}{a+b}$ \textbengali{এর মান কত?}
	\end{enumerate}
	
%14
	\item \textbengali{দেখাও যে,} $ \left( \dfrac{2}{x} - \dfrac{x}{2} \right) $ \textbengali{এর উৎপাদকগুলির সমষ্টি} $ \left( \dfrac{x}{2} + \dfrac{2}{x} \right) $.
	
%15
	\item If $ x + \dfrac{1}{x} = -1 $, find the value of $ x^{2017} + \dfrac{1}{x^{2017}} $.
	
%16
	\item $ p+q+r = 9 $, $ p^2 + q^2 + r^2 = 27 $, $ p^3 + q^3 + r^3 = 81 $, $ pqr $ = \textbengali{কত?}
	
%17
	\item If $ x+y+z = 0 $, show that, $ \left( \dfrac{yz}{2x^2 + yz} + \dfrac{zx}{2y^2 + zx} + \dfrac{xy}{2z^2 + xy} \right) = 1 $.
	
%18
	\item If $ x^3 + \dfrac{3}{x} = 4(a^3 + b^3) $ and $ 3x + \dfrac{1}{x^3} = 4(a^3 - b^3) $, show that $a^2 - b^2 = 1$.
	
	
%19
	\item If $ a+b+c = 0 $, prove that, $ a^7 + b^7 + c^7 = 7abc(ab+bc+ca)^2 $.	
	
%20
	\item Find the value of $\left( \sqrt{a-2\sqrt{a-1}} - \sqrt{a+2\sqrt{a-1}} \right)$ where $1 \leq a \leq 2$.
	
%21
	\item $-1 \leq \dfrac{3*x - 4}{7} \leq 5$ \textbengali{হলে} $x$ \textbengali{এর ক্ষুদ্রতম ও বৃহত্তম মান কত ?}
	
%22
	\item $ \left( x + \dfrac{1}{x} \right)^2 = 3 $ \textbengali{হলে} $ x^{36} + x^{30} + x^{26} + x^{20} + x^{18} + x^{12} + x^{6} + 1 $ = \textbengali{কত} ?
	
%23
	\item $ \left( x - \dfrac{1}{x} \right) = 1$ \textbengali{হলে} $ \dfrac{x^4 - \dfrac{1}{x^2}}{3x^2 + 5x - 3} $ = \textbengali{কত} ?
	
%24
	\item \textbengali{একটি বর্গক্ষেত্রের ভেতরে স্তম্ভ ও সারি বরাবর সমান তিনভাগ করা হল। তাদের প্রত্যেকটিতে} $1$ \textbengali{থেকে} $9$ \textbengali{পর্যন্ত পূর্ণসংখ্যার একটিকে এমনভাবে রাখা হল যাতে প্রত্যেক স্তম্ভ বরাবর, সারি বরাবর ও দুটি কর্ণ বরাবর সকল যোগফল সমান হয়। তবে প্রমাণ কর যে, একদম মাঝখানে রাখা সংখ্যাটি অবশ্যই} $5$ \textbengali{হবে।} 
	\begin{flushright}
		 MTRP 2014
	\end{flushright}
	
%25
	\item $a$ \textbengali{ও} $b$ \textbengali{দুটি ধনাত্মক বাস্তব সংখ্যা।} $ a\sqrt{a} + b\sqrt{b} = 183 $ \textbengali{ও} $ a\sqrt{b} + b\sqrt{a} =182 $. $ \dfrac{9}{5} \left( a+b \right) $ \textbengali{এর মান কত ?}
	 \begin{flushright}
	 	PRMO 2017
	\end{flushright}	  
	
	
%26
	\item $x$, $y$, $z$ \textbengali{বাস্তব ধনাত্মক সংখ্যা।} $x^2 + 4y^2 + 16z^2 = 48$ \textbengali{ও} $xy + 4yz + 2zx = 24$ \textbengali{হলে} $x^2 + y^2 + z^2 = $\textbengali{কত?}
	\begin{flushright}
	 	PRMO 2017
	\end{flushright}	
	
	
%27
	\item $\sqrt[3]{3} - \sqrt[3]{2}$ \textbengali{এর করণী নিরসক উৎপাদক কী?}
	
%28
	\item $\sqrt[3]{3} - \sqrt{2}$ \textbengali{এর করণী নিরসক উৎপাদক কী?}

%29 
	\item $\alpha$, $\beta$ are the two roots of the equation $x^2 -6x - 2 = 0$. If $a_n = \alpha^n - \beta^n$, \\ show that, $\dfrac{a_{10} - 2a_8}{2a_9} = 3.$
	
%30
	\item A root of the equation $4x^2 + 2x - 1 = 0$ is $\alpha$. $f(x) = 4x^3 - 3x + 1$. Find $2[f(\alpha) + \alpha].$
	
%31
	\item 20 \textbengali{টি চলকের মধ্যক (গড়)} 85. \textbengali{দুটি চলককে ভুল করে} 57 \textbengali{ও} 60 \textbengali{এর স্থানে} 75 \textbengali{ও} 70  \textbengali{নেওয়া হয়েছে। সঠিক মধ্যক কত ?}
	
%32
	\item 120 \textbengali{জন ছাত্রছাত্রীর গড় ওজন} 56 kg. \textbengali{ছাত্রদের গড় ওজন} 60 kg. \textbengali{ছাত্রীদের গড় ওজন} 50 kg. \textbengali{ছাত্র ও ছাত্রীদের সংখ্যা কী কী ?}
	
%33
	\item 3.2, 5.8, 7.9 \textbengali{ও} 4.5 \textbengali{চলকের পরিসংখ্যা যথাক্রমে} $x, x+2, x-3, x+6.$ \textbengali{গড়} 4.876 \textbengali{হলে} $x=$ \textbengali{কত ?}
	
%34
	\item If $x = \dfrac{\sqrt{a+2b} + \sqrt{a-2b}}{\sqrt{a+2b} - \sqrt{a-2b}}$, show that, $bx^2 -ax+b = 0.$
	
%35
	\item Find the value of $\dfrac{x+\sqrt{20}}{x-\sqrt{20}} + \dfrac{x+\sqrt{12}}{x-\sqrt{12}}$, given that, $x = \dfrac{4\sqrt{15}}{\sqrt{5} + \sqrt{3}}.$
	
%36
	\item \textbengali{সংখ্যাগুরুমান} (mode) \textbengali{নির্ণয়ের সূত্র।}
	
%37
	\item If $x + y + z = 4xyz$, show that, $\dfrac{x^2}{1 - 4x^2} + \dfrac{y^2}{1 - 4y^2} + \dfrac{z^2}{1 - 4z^2} = \dfrac{16x^2 y^2 z^2}{(1 - 4x^2) (1 - 4y^2) (1 - 4z^2)}.$

%38
	\item \textbengali{হীরকের দাম তার ওজনের বর্গের সঙ্গে সরলভেদে থাকে। সোনার ওপর হীরক বসিয়ে তৈরি তিনটি সমান ওজনের আংটির দাম যথাক্রমে} $x$ \textbengali{টাকা},  $y$ \textbengali{টাকা এবং} $z$ \textbengali{টাকা এবং আংটি তিনটিতে হীরকের ওজন যথাক্রমে} $3$, $4$ \textbengali{ও} $5$ \textbengali{ক্যারেট। দেখাও যে, এক ক্যারেট হীরকের দাম} $\left( \dfrac{x+z}{2} - y \right)$ \textbengali{টাকা।  (প্রতিটি আংটি তৈরির পারিশ্রমিক সমান)}

%39
	\item \textbengali{হীরকের মূল্য তার ওজনের বর্গের সঙ্গে সমানুপাতিক।} $8000$ \textbengali{টাকা মূল্যের একটি হীরকখণ্ড ভেঙে} $3$ \textbengali{টি খণ্ডে বিভক্ত করা হল। খণ্ড} $3$ \textbengali{টির ওজনের অনুপাত} $8:7:5$. \textbengali{ভাঙার ফলে কত ক্ষতি হল তা নির্ণয় কর।}
	
%40
	\item \textbengali{রিজার্ভ ব্যাংকের চলমান সিঁড়ি বেয়ে দুই ব্যাক্তি ওপরে উঠছিলেন। তাঁদের গতিবেগের অনুপাত} $1:2$. \textbengali{তাঁরা যথাক্রমে} $18$ \textbengali{টি ও } $27$ \textbengali{টি ধাপ অতিক্রম করে উপরে উঠলেন। চলমান সিঁড়িতে মত ধাপের সংখ্যা কত ?}

%41
	\item If $(a+b+c)x = (b+c-a)y = (c+a-b)z = (a+b-c)w$, then prove that, $x(yz + zw + yw) = yzw$.
	
%42
	\item If $x^2 - 2x + 4 = 0$, find out $x^6$ and $x$.
	
%43
	\item If $x^3 + \dfrac{1}{x^3} = 2$, find the value of $\left( x + \dfrac{1}{x} \right)$.
	
%44
	\item Show that, $\dfrac{5 + \sqrt{5}}{\sqrt{5 + 3\sqrt{5}}} = \sqrt[4]{20}$.
	
%45
	\item \textbengali{বর্গ বা বর্গমূল না করে প্রমাণ কর যে,} $\sqrt{5} + \sqrt{3} > \sqrt{6} + \sqrt{2}$.
	
%46
	\item $10\%$ \textbengali{হার সুদে} $8100$  \textbengali{টাকা ধার করে এক বছরের মধ্যে দুটি সমান কিস্তিতে শোধ করলে প্রতিটি কিস্তির পরিমাণ ক্কত ?}
	
%47
	\item \textbengali{একটি সন্দেশের বাক্সের দৈর্ঘ্য} $12$ $c.m.$, \textbengali{প্রস্থ} $10$ $c.m.$ \textbengali{ও উচ্চতা} $7$ $c.m.$ \textbengali{ওই বাক্সের মধ্যে} $2$ $c.m.$ \textbengali{বাহুবিশিষ্ট ঘনকাকার কতগুলি সন্দেশ রাখা যাবে ?}
	
%48
	\item \textbengali{একটি আয়তঘনাকার বাক্সের দৈর্ঘ্য} $6$ $c.m.$, \textbengali{প্রস্থ} $6$ $c.m.$ \textbengali{ও উচ্চতা} $5$ $c.m.$ \textbengali{ওই বাক্সের মধ্যে} $3$ $c.m.$ \textbengali{ব্যাসের কতগুলি গোলক রাখা যাবে ?}
	
%49
	\item $\left(x + \sqrt{x^2 - bc}\right) \left(y + \sqrt{y^2 - ca}\right) \left(z + \sqrt{z^2 - ab}\right) = \left(x - \sqrt{x^2 - bc}\right) \left(y - \sqrt{y^2 - ca}\right) \left(z - \sqrt{z^2 - ab}\right)$ \textbengali{হলে দেখাও যে প্রত্যেক পক্ষের মান} $\pm abc$ \textbengali{এর সমান।}
	
%50
	\item $x+\dfrac{1}{y} = y+\dfrac{1}{z} = z+\dfrac{1}{x}$ \textbengali{হলে দেখাও যে} $xyz = \pm 1$.

%51
	\item \textbengali{যদি} $a(b-c)x^2 + b(c-a)xy + c(a-b)y^2 = 0$ \textbengali{সমীকরণের বামপক্ষ একটি পূর্ণবর্গ রাশিমালা হয়, তবে প্রমাণ কর যে,} $\dfrac{1}{a} + \dfrac{1}{c} = \dfrac{2}{b}$.
	
%52
	\item $a^2 + b^2 + c^2 = x^2 + y^2 + z^2 = ax + by + cz$ \textbengali{হলে প্রমাণ কর যে,} $\dfrac{x}{a} = \dfrac{y}{b} = \dfrac{z}{c}$.
	
%53
	\item $a\left(x - y\right) + a^2 = b\left(y - z\right) + b^2 = c\left(z - x\right) + c^2$ \textbengali{হলে প্রমাণ কর যে, প্রত্যেকটির মান} $= \dfrac{a+b+c}{\frac{1}{a} + \frac{1}{b} + \frac{1}{c}}$.
	
%54
	\item If $2x = \sqrt{a} + \dfrac{1}{\sqrt{a}}$, show that, $\dfrac{\sqrt{x^2 - 1}}{x - \sqrt{x^2 - 1}} = \dfrac{1}{2} \left(a-1\right)$.
	
%55
	\item If $\dfrac{a}{b+c} + \dfrac{b}{c+a} + \dfrac{c}{a+b} = 1$, prove that, $\dfrac{a^2}{b+c} + \dfrac{b^2}{c+a} + \dfrac{c^2}{a+b} = 0$.
	
%56
	\item If $\dfrac{a^2}{b+c} + \dfrac{b^2}{c+a} + \dfrac{c^2}{a+b} = 0$, prove that, $\dfrac{a}{b+c} + \dfrac{b}{c+a} + \dfrac{c}{a+b} = 1$, provided $\left( a+b+c \right) \neq 0$.

%57
	\item If $a+b+c = 1$, $ab+bc+ca = \dfrac{1}{3}$, $abc = \dfrac{1}{27}$, prove that, $\dfrac{1}{a+bc} + \dfrac{1}{b+ca} + \dfrac{1}{c+ab} = \dfrac{27}{4}$.
	
%58
	\item If $\dfrac{by+cz}{b^2 + c^2} = \dfrac{cz+ax}{c^2+a^2} = \dfrac{ax+by}{a^2+b^2}$, prove that, $\dfrac{x}{a} = \dfrac{y}{b} = \dfrac{z}{c}$.
	
%59
	\item If $\dfrac{p}{a} + \dfrac{q}{b} + \dfrac{r}{c} = 1$ and $\dfrac{a}{p} + \dfrac{b}{q} + \dfrac{c}{r} = 0$, prove that, $\dfrac{p^2}{a^2} + \dfrac{q^2}{b^2} + \dfrac{r^2}{c^2} = 1$.
	
%60
	\item If $x\left( b-c \right) + y \left( c-a \right) + z\left( a-b \right) = 0$, show that, $\dfrac{bz-cy}{b-c} = \dfrac{cx-az}{c-a} = \dfrac{ay-bx}{a-b}$.
	
%61
	\item If $xy+yz+zx=1$, show that, $\left(1+x^2\right) \left(1+y^2\right)\left(1+z^2\right) = \{ \left(x+y\right)\left(y+z\right)\left(z+x\right) \}^2$.
	
%62
	\item If $x+y+z=1$, show that, $\dfrac{x+yz}{\left(x+y\right)\left(z+x\right)} + \dfrac{y+zx}{\left(y+z\right)\left(x+y\right)} + \dfrac{z+xy}{\left(z+x\right)\left(y+z\right)} = 3$.
	
%63
	\item If $a^2 - b^2 = b^2 - c^2 = c^2 - a^2$, prove that, $\dfrac{ab-c^2}{a-b} + \dfrac{bc-a^2}{b-c} + \dfrac{ca-b^2}{c-a} = 0$.
	
%64
	\item If $a+b+c=0$, prove that, $\dfrac{a^2}{2a^2 + bc} + \dfrac{b^2}{2b^2+ca} + \dfrac{c^2}{2c^2+ab} = 1$.

%65
	\item If $a+b+c=0$, prove that, $\dfrac{a^2+b^2+c^2}{a^3+b^3+c^3} + \dfrac{2}{3}\left( \dfrac{1}{a} + \dfrac{1}{b} + \dfrac{1}{c} \right) = 0$.
	
%66
	\item If $x = by + cz$, $y = cz + ax$, $z = ax + by$, prove that, $\dfrac{1}{1+a} + \dfrac{1}{1+b} + \dfrac{1}{1+c} = 2$.
	
%67
	\item If $ab+bc+ca = 0$, prove that, $\dfrac{1}{a^2 - bc} + \dfrac{1}{b^2 - ca} + \dfrac{1}{c^2 - ab} = 0$.
	
%68
	\item If $a^2 = by + cz$, $b^2 = cz + ax$, $c^2 = ax + by$, prove that, $\dfrac{x}{x+a} + \dfrac{y}{y+b} + \dfrac{z}{z+c} = 1$.
	
%69
	\item If $a+b+c=5$, $ab+bc+ca=8$, $abc=-7$, find the value of $\left( \dfrac{a^2}{b} + \dfrac{b^2}{a} \right) + \left( \dfrac{b^2}{c} + \dfrac{c^2}{b} \right) + \left( \dfrac{c^2}{a} + \dfrac{a^2}{c} \right)$.
	
%70
	\item If $\dfrac{a-b}{c} + \dfrac{b-c}{a} + \dfrac{c+a}{b} = 1$ and $a-b+c \neq 0$, show that, $\dfrac{1}{a} = \dfrac{1}{b} + \dfrac{1}{c}$.
	
%71
	\item If $\dfrac{b+c}{a} = \dfrac{c+a}{b} = \dfrac{a+b}{c}$, show that, $a+b+c=0$ or $a=b=c$.
	
%72
	\item If $\dfrac{1}{a} + \dfrac{1}{b} + \dfrac{1}{c} = \dfrac{1}{a+b+c}$, prove that, $\dfrac{1}{a^5} + \dfrac{1}{b^5} + \dfrac{1}{c^5} = \dfrac{1}{a^5 + b^5 + c^5} = \dfrac{1}{(a+b+c)^5}$.
	
%73
	\item If $a+b+c=0$, show that, $(a^2 + b^2 + c^2)^2 = 2(a^4 + b^4 + c^4)$.
	
%74
	\item If $x=a(b-c)$, $y=b(c-a)$, $z=c(a-b)$, show that, $\left( \dfrac{x}{a} \right)^3 + \left( \dfrac{y}{b} \right)^3 + \left( \dfrac{z}{c} \right)^3 = \dfrac{3xyz}{abc}$.
	
%75
	\item If $x=a^2-bc$, $y=b^2-ca$ and $z=c^2-ab$, prove that, $x^3+y^3+z^3-3xyz = \\ (a^3+b^3+c^3-3abc)^2$.

%76
	\item If $a+c=2b$, prove that, $a^2(b+c)+b^2(c+a)+c^2(a+b)=\dfrac{2}{9}(a+b+c)^3$.
	
%77
	\item If $x=b+c-a$, $y=c+a-b$, $z=a+b-c$, prove that, $x^3+y^3+z^3-3xyz = 4(a^3+b^3+c^3-3abc)$.

%78
	\item If $(a^2+b^2+c^2)(x^2+y^2+z^2) = (ax+by+cz)^2$, prove that, $\dfrac{x}{a} = \dfrac{y}{b} = \dfrac{z}{c}$.
	
%79
	\item \textbengali{একটি শূন্যগর্ভ জাহাজের ওজন এবং উহার অন্তর্গত মালপত্রের ওজন যথাক্রমে জাহাজের দৈর্ঘ্যের বর্গ ও ঘনের সাথে সরলভেদে আছে। যদি} $l_1$ \textbengali{দৈর্ঘ্যবিশিষ্ট জাহাজের মালপত্রসহ ওজন} $w_1$ \textbengali{এবং} $l_2$ \textbengali{ও} $l_3$ \textbengali{দৈর্ঘ্যবিশিষ্ট জাহাজের মালপত্রসহ ওজন} $w_2$ \textbengali{ও} $w_3$ \textbengali{হয় তবে প্রমাণ কর যে,} $$\dfrac{w_1}{{l_1}^2}\left( l_2 - l_3 \right) + \dfrac{w_2}{{l_2}^2}\left( l_3 - l_1 \right) + \dfrac{w_3}{{l_3}^2}\left( l_1 - l_1 \right) = 0$$.
	
%80
	\item If $a^{\frac{1}{3}} + b^{\frac{1}{3}} + c^{\frac{1}{3}} = 0$, prove that, $(a+b+c)^3 = 27abc$.
	
%81
	\item \textbengali{কোনো এক লীগের প্রতিযোগিতায় একটি দিনে যতগুলি খেলা হয় তা যুগ্মভাবে ওই দিন এবং বাকি দিনগুলির সঙ্গে ওই দিনের যোগফলের সহিত সমানুপাতে থাকে। যদি পরপর তিনদিন } $6$, $5$ \textbengali{এবং} $3$ \textbengali{টি খেলা হয়ে থাকে তবে কোন কোন দিন ওই খেলাগুলি হয়েছিল এবং প্রতিযোগিতাটি কত দিনের ছিল ?}
	
%82
	\item If $\dfrac{ay-bx}{c} = \dfrac{cx-az}{b} = \dfrac{bz-cy}{a}$, prove that, $\dfrac{x}{a} = \dfrac{y}{b} = \dfrac{z}{c}$.
	
%83
	\item If $a+b+c=0$, prove that, $a^5 + b^5 + c^5 = \dfrac{5}{6} (a^2 + b^2 + c^2) (a^3 + b^3 + c^3)$.
	
%84
	\item If $ax+by+cz=p$, $bx+cy+az=q$, $cx+ay+bz=r$, prove that, $p^3+q^3+r^3-3pqr = (a^3+b^3+c^3-3abc)(x^3+y^3+z^3-3xyz)$.
	
%85
	\item $x$, $y$, $z$ \textbengali{এমন তিনটি চলরাশি যে} $(y+z-x)$ \textbengali{এর মান ধ্রুবক এবং} $(z+x-y)(x+y-z)\propto yz$. \textbengali{প্রমাণ কর যে,} $(x+y+z) \propto yz$.
	
%86
	\item $(x+y) \propto z$ \textbengali{যখন} $y$ \textbengali{ধ্রুবক এবং} $(z+x) \propto y$ \textbengali{যখন} $z$ \textbengali{ধ্রুবক। প্রমাণ কর যে,} $(x+y+z) \propto yz$ \textbengali{যখন} $y$, $z$ \textbengali{উভয়েই চল।}
	
%87
	\item $x$ \textbengali{ও} $y$ \textbengali{দুটি ভিন্ন বাস্তব রাশি এবং} $x \propto y(x+y)$ \textbengali{ও} $y \propto x(x-y)$. \textbengali{প্রমাণ কর যে,} $(x^2 - y^2)$ \textbengali{এর মান} $x$ \textbengali{ও} $y$ \textbengali{এর ওপর নির্ভর করে না।}
	
%88
	\item $\dfrac{x}{y} \propto x-y $ \textbengali{ও} $\dfrac{y}{x} \propto x^2 + xy + y^2 $ \textbengali{হলে প্রমাণ কর যে,} $x^3 - y^3 = $ \textbengali{ধ্রুবক।}
	
%89
	\item If $u^2 + v^2 \propto x^2 + y^2$ and $uv \propto xy$, prove that, $u+v \propto x+y$ when $\dfrac{u}{v} + \dfrac{v}{y} = \dfrac{x}{y} + \dfrac{y}{x}$.
	
%90
	\item If $x \propto y$ and $y \propto z$ and $x=a$ when $y=b$, $z=c$ and $x = a'$ when $y = b'$, $z=c'$, prove that, $\dfrac{a^2 + b^2 + c^2}{aa' + bb' + cc'} = \dfrac{aa' + bb' + cc'}{(a')^2 + (b')^2 + (c')^2}$.
	
%91
	\item If $x \propto y + z$, $y \propto z + x$, $z \propto x + y$, and $a$, $b$, $c$ are three constants, prove that, $\dfrac{a}{a+1} + \dfrac{b}{b+1} + \dfrac{c}{c+1} = 1$, when $x+y+z \neq 0$.
	
%92
	\item If $ax^2 + 2hxy + by^2 \propto u^2$ and $lx+my \propto u$, prove that, $x \propto y$.
	
%93
	\item If $(x+y+z)(y+z-x)(z+x-y)(x+y-z) \propto x^2y^2$, prove that, $x^2 + y^2 = z^2$ or $x^2 + y^2 - z^2 \propto xy$.
	
%94
	\item $x$, $y$, $z$ are three variables such that $(x+y+z)$ is constant. $(x+z-y)(x+y-z) \propto yz$. Prove that, $(y+z-x) \propto yz$.
	
%95
	\item If $a+b+c=0$, prove that, $\dfrac{1}{a^2 + b^2 - c^2} + \dfrac{1}{b^2 + c^2 - a^2} + \dfrac{1}{c^2 + a^2 - b^2} = 0$.
	
%96
	\item If $(y+z) \propto x$ and $(z+x) \propto y$, prove that, $(x+y) \propto z$.



%97	
	\item Find the area of the shaded part in the following figure where $PQRS$ is a square and the length of each of the sides of the square is $x$. $P$, $Q$, $R$, $S$ respectively are the centers of ${\stackrel{{\mbox{$\Large{\frown}$}}}{SQ}}$, ${\stackrel{{\mbox{$\Large{\frown}$}}}{PR}}$, ${\stackrel{{\mbox{$\Large{\frown}$}}}{SQ}}$, ${\stackrel{{\mbox{$\Large{\frown}$}}}{PR}}$.
		
	\begin{figure}[H]
	\centering
	\includegraphics[scale = 0.15]{IMG_20230915_184133_844}\\
	\end{figure}


%98
	\item If $x+y : \sqrt{xy} = 4 : 1$, find $x:y$.
	
%99
	\item If $a : b = b : c$, show that, $a^2 b^2 c^2 \left( \dfrac{1}{a^3} + \dfrac{1}{b^3} + \dfrac{1}{c^3} \right) = a^3 + b^3 + c^3$.
	
%100
	\item If $a : b = b : c$, prove that, $\dfrac{abc(a+b+c)^3}{(ab+bc+ca)^3} = 1$.
	
%101
	\item If $3x - 4y \propto \sqrt{xy}$, prove that, $x^2 + y^2 \propto xy$.
	
%102
	\item If $\dfrac{x^2 - yz}{a} = \dfrac{y^2 - zx}{b} = \dfrac{z^2 - xy}{c}$, prove that, $x = y = z$.
	
%103
	\item \textbengali{কোনো ব্যক্তির পেনশনের পরিমাণ তার চাকুরী জীবনের বর্গমূলের সাথে সমানুপাতে থাকে। দুজন ব্যক্তির মধ্যে প্রথম ব্যক্তি দ্বিতীয় ব্যক্তি অপেক্ষা} $9$ \textbengali{বছর বেশি চাকরি করেন এবং} $500$ \textbengali{টাকা বেশি পেনশন পান। যদি প্রথম ব্যক্তি দ্বিতীয় ব্যক্তি অপেক্ষা} $4\dfrac{1}{4}$ \textbengali{বছর বেশি চাকরি করতেন তাহলে তাদের পেনশনের অনুপাত হত} $9:8$. \textbengali{তারা কত বছর চাকরি করেছেন? প্রত্যেকে কত টাকা পেনশন পেয়েছিলেন?}
	
%104
	\item If $a+b+c = 6$ and $ab+bc+ca = 9$, prove that, $\dfrac{1}{1-a} + \dfrac{1}{1-b} + \dfrac{1}{1-c} = 0$.
	
%105
	\item If $x^2 + y^2 + z^2 = xy + yz + zx$, prove that, $\dfrac{x^2}{yz} + \dfrac{y^2}{zx} + \dfrac{z^2}{xy} = 3$.
	
%106
	\item If $2s = a+b+c$, prove that, $(s-a)^3 + (s-b)^3 + 3(s-a)(s-b)c = c^3$.
	
%107
	\item If $2s = a+b+c$, prove that, $\dfrac{1}{s-a} + \dfrac{1}{s-b} + \dfrac{1}{s-c} - \dfrac{1}{s} = \dfrac{abc}{s(s-a)(s-b)(s-c)}$.
	
%108
	\item If $(a+b)^\frac{1}{3} + (b+c)^\frac{1}{3} + (c+a)^\frac{1}{3} = 0$, show that, $(a+b+c)^3 = 9(a^3 + b^3 + c^3)$.
	
%109
	\item If $\dfrac{1}{y} - \dfrac{1}{x} \propto \dfrac{1}{x-y}$, show that, $3x + y \propto \sqrt{xy}$.
	
%110
	\item If $2x^2 + 3y^2 \propto xy$, prove that, $9x^4 + 4y^4 \propto x^2y^2$.
	
%111
	\item If $2x + 3y \propto \sqrt{xy}$, prove that, $x \propto y$.
	
%112
	\item If $\dfrac{1}{x} + \dfrac{1}{y} \propto \dfrac{1}{x+y}$, prove that, $(x^2 + y^2) \propto xy$.
	
%113
	\item If $\dfrac{1}{x} - \dfrac{1}{y} \propto \dfrac{1}{x-y}$, prove that, $(x^2 + y^2) \propto xy$ and $x \propto y$.
	
%114
	\item If $\left( x^3 - \dfrac{1}{y^3} \right) \propto \left( x^3 + \dfrac{1}{y^3} \right)$, prove that, $x \propto \dfrac{1}{y}$.
	
%115
	\item $x \propto (y+z)$	, $y \propto (z+x)$, $z \propto (x+y)$ \textbengali{এবং} $a$, $b$, $c$ \textbengali{যথাক্রমে তিনটি ভেদের ধ্রুবক হলে দেখাও যে,} $ab+bc+ca+2abc = 1$.
	
%116
	\item If $(a+b+c) \propto (a+b-c)$ and $(a^2 + b^2 + c^2) \propto (a^2 + b^2 - c^2)$, prove that, $a \propto b$ and $b \propto c$.
	
%117
	\item \textbengali{যদি} $r_1$, $r_2$, $r_3$ \textbengali{ব্যাসার্ধবিশিষ্ট বৃত্তগুলির কেন্দ্রে যথাক্রমে} $l_1$, $l_2$, $l_3$ \textbengali{দৈর্ঘ্যের বৃত্তচাপগুলির দ্বারা উৎপন্ন কোণগুলির বৃত্তীয় মানগুলি} $a_1$, $a_2$, $a_3$ \textbengali{হয়, তবে প্রমাণ কর যে,} $\dfrac{1}{n} \left( a_1 r_1 + a_2 r_2 + a_3 r_3 \right)$ \textbengali{ব্যাসার্ধবিশিষ্ট কোনো বৃত্তের কেন্দ্রে} $\left( l_1 + l_2 + l_3 \right)$ \textbengali{দৈর্ঘ্যবিশিষ্ট কোনো বৃত্তচাপ যে কোণ উৎপন্ন করে তার বৃত্তীয় পরিমাপ হবে} $n$ \textbengali{রেডিয়ান।} 
	
%118
	\item \textbengali{যদি} $r_1$, $r_2$, $r_3$ \textbengali{ব্যাসার্ধবিশিষ্ট বৃত্তগুলির কেন্দ্রে যথাক্রমে} $l_1$, $l_2$, $l_3$ \textbengali{দৈর্ঘ্যের বৃত্তচাপগুলির দ্বারা উৎপন্ন কোণগুলির বৃত্তীয় মানগুলি} $\theta_1$, $\theta_2$, $\theta_3$ \textbengali{হয়, তবে প্রমাণ কর যে,} $\left( r_1 + r_2 + r_3 \right)$ \textbengali{ব্যাসার্ধবিশিষ্ট কোনো বৃত্তের কেন্দ্রে} $\left( l_1 + l_2 + l_3 \right)$ \textbengali{দৈর্ঘ্যবিশিষ্ট কোনো বৃত্তচাপ যে কোণ উৎপন্ন করে তার বৃত্তীয় পরিমাপ হবে} $\left( \dfrac{r_1 \theta_1 + r_2 \theta_2 + r_3 \theta_3}{r_1 + r_2 + r_3} \right)$ \textbengali{রেডিয়ান।} 
	
%119
	\item \textbengali{কোনো দ্বিঘাত সমীকরণ} $ax^2 + bx + c = 0 \, [a \neq 0]$ \textbengali{-এ} $b^2 = 9ac$ \textbengali{হলে সমীরণটির বীজদ্বয়ের মধ্যে সম্পর্ক কী?}
	
%120
	\item If $x = cy + bz$, $y = az + cx$, $z = bx + ay$, prove that, $\dfrac{x^2}{1-a^2} = \dfrac{y^2}{1-b^2} = \dfrac{z^2}{1-c^2}$.
	
%121
	\item If $x = bz + cy$, $y = cx + az$, $z = ay + bx$, prove that, $\dfrac{x}{\sqrt{1-a^2}} = \dfrac{y}{\sqrt{1-b^2}} = \dfrac{z}{\sqrt{1-c^2}}$.
	
%122
	\item If $x = \dfrac{1}{2} \left( \sqrt{\dfrac{a}{b}} - \sqrt{\dfrac{b}{a}}\right)$, find the value of $\dfrac{2a\sqrt{1+x^2}}{x+\sqrt{1+x^2}}$.
	
%123
	\item If $\sqrt{\left(x - \sqrt{a^2 - b^2}\right)^2 + y^2} + \sqrt{\left(x + \sqrt{a^2 - b^2}\right)^2 + y^2} = 2a$, prove that, $\dfrac{x^2}{a^2} + \dfrac{y^2}{b^2} = 1$.
	
%124
	\item If $y - mx = \sqrt{a^2 m^2 + b^2}$ and $x + my = \sqrt{a^2 + b^2 m^2}$, prove that, $x^2 + y^2 = a^2 + b^2$.
	
%125
	\item If $\dfrac{b}{x} + \dfrac{a}{y} \propto \dfrac{1}{ax+by}$, prove that, $x \propto y$.
	
%126
	\item If $x > y$, prove that, $\sqrt{y + \sqrt{2xy - x^2}} + \sqrt{y - \sqrt{2xy - x^2}} = \sqrt{2x}$.
	
%127
	\item \textbengali{একটি লম্ব বৃত্তাকার শঙ্কুর তির্যক উচ্চতা} $7 \, c.m.$ \textbengali{এবং সমগ্রতলের ক্ষেত্রফল} $147.84 \, sq. \, c.m.$ \textbengali{শঙ্কুটির ভূমির ব্যাসার্ধের দৈর্ঘ্য নির্ণয় কর।}
	
%128
	\item $1 \, c.m.$ \textbengali{ও} $6  \, c.m.$ \textbengali{দৈর্ঘ্যের ব্যাসার্ধ বিশিষ্ট দুটি নিরেট গোলককে গলিয়ে}  $1 \, c.m.$ \textbengali{পুরু ফাঁপা গোলকে পরিণত করা হলে, নতুন গোলকটির বাইরের বক্রতলের ক্ষেত্রফল কত ?}
	
%129
	\item \textbengali{কোনো মূলধনের} $2$ \textbengali{বছরের সুদ ও চক্রবৃদ্ধি সুদ যথাক্রমে} $8400$ \textbengali{টাকা ও} $8652$ \textbengali{টাকা হলে মূলধন ও বার্ষিক সুদের হার কত ?}
	
%130
	\item $a \propto \dfrac{1}{b}$ \textbengali{হলে প্রমাণ কর যে} $(a+b)$ \textbengali{এর মান ক্ষুদ্রতম যখন} $a = b$.
	
%131
	\item If $x = \dfrac{\sqrt{3}}{2}$, prove that, $\dfrac{\sqrt{1+x} + \sqrt{1-x}}{\sqrt{1+x} - \sqrt{1-x}} = \sqrt{3}$.
	
%132
	\item \textbengali{একটি হস্টেলের ব্যয় আংশিক ধ্রুবক ও আংশিক ওই হস্টেলবাসী লোকসংখ্যার সঙ্গে সরলভেদে আছে। লোকসংখ্যা} $120$ \textbengali{হলে ব্যয় হয়} $2000$ \textbengali{টাকা এবং লোকসংখ্যা} $100$ \textbengali{হলে ব্যয় হয়} $1700$ \textbengali{টাকা। ব্যয়} $1800$ \textbengali{টাকা হলে লোকসংখ্যা কত হবে ?}
	
%133
	\item \textbengali{মালগাড়ি সংযুক্ত না থাকলে একটি ইঞ্জিন ঘণ্টায়} $40$ \textbengali{মাইল বেগে যেতে পারে এবং এর সঙ্গে মালগাড়ি সংযুক্ত থাকলে এর গতিবেগ যে পরিমাণে হ্রাস পায় তা তার সঙ্গে সংযুক্ত মালগাড়ির সংখ্যার বর্গমূলের সঙ্গে সরলভেদে থাকে।} $16$ \textbengali{টি মালগাড়ি সংযুক্ত থাকলে তার গতিবেগ হয়} $20$ \textbengali{মাইল / ঘণ্টা।}
	\begin{enumerate}[(a)]
		\item \textbengali{ইঞ্জিনটি সবচেয়ে বেশি কতগুলি বগি নিয়ে চলতে সক্ষম থাকবে ?}
		\item \textbengali{সবচেয়ে কম কতগুলি বগি যোগ করলে চলতে অক্ষম হবে ?}
	
	\end{enumerate}
	
%134
	\item If $(x+z) :(y+z) = \left(\dfrac{x}{y} +2 \right) : \left( \dfrac{y}{x} + 2 \right)$, show that, $(x-z) : (y - z) = x^2 : y^2$.
	
%135
	\item \textbengali{কোনো পূর্ণবর্গ সংখ্যার ভাগ প্রক্রিয়ায় বর্গমূল করার সময়} $2$ \textbengali{গুণ করতে হয় কেন ?}
	
%136
	\item If $x^2 + y^2 + z^2 = 25$, $a^2 + b^2 + c^2 = 36$ and $ax + by + cz = 30$, find the value of $\dfrac{x+y+z}{a+b+c}$.
	
%137
	\item If $x = cy + bz$, $y = cx + az$ and $z = bx + ay$, prove that, $a^2 + b^2 + c^2 + 2abc = 1$.
	
%138
	\item If $x + y + xy = 5$, $y + z + yz = 7$ and $z + x + zx = 11$, prove that, $(1+x)^2(1+y)^2(1+z)^2 = 576$.
	
%139
	\item If $2x = a + \sqrt{\dfrac{4b^3 - a^3}{3a}}$ and $2y = a - \sqrt{\dfrac{4b^3 - a^3}{3a}}$, prove that, $x^3 + y^3 = b^3$.
	
%140
	\item If $\dfrac{a^2 - bc}{a^2 + bc} + \dfrac{b^2 - ca}{b^2 ++ ca} + \dfrac{c^2 - ab}{c^2 + ab} = 1$, prove that, $\dfrac{a^2}{a^2 + bc} + \dfrac{b^2}{b^2 + ca} + \dfrac{c^2}{c^2 + ab} = 2$.
	
%141
	\item \textbengali{সরল করঃ} $\dfrac{2 + \sqrt{3}}{\sqrt{2} + \sqrt{2 + \sqrt{3}}} + \dfrac{2 - \sqrt{3}}{\sqrt{2} - \sqrt{2 - \sqrt{3}}}$.
	
%142
	\item \textbengali{সরল করঃ} $\dfrac{\sqrt{26 - 15\sqrt{3}}}{5\sqrt{2} - \sqrt{38 + 5\sqrt{3}}}$.
	
%143
	\item Evaluate : $\dfrac{(4 + \sqrt{15})^\frac{3}{2} + (4 - \sqrt{15})^\frac{3}{2}}{(6 + \sqrt{35})^\frac{3}{2} + (6 - \sqrt{35})^\frac{3}{2}}$.
	
%144
	\item If $2x = \sqrt{\dfrac{p}{q}} - \sqrt{\dfrac{q}{p}}$, prove that, $\dfrac{2p\sqrt{x^2 + 1}}{x + \sqrt{x^2 + 1}} = p+q$.
	
%145
	\item If $x = y \sqrt{1 - z^2} + z \sqrt{1 - y^2}$, prove that, $(x + y + z)(y + z -x)(z + x - y)(x + y - z) = 4x^2 y^2 z^2$.
	
%146
	\item $x$ \textbengali{মুলদ রাশি}, $\sqrt{y}$ \textbengali{অমূলদ রাশি এবং} $\sqrt[3]{x + \sqrt{y}} = a + \sqrt{b}$ \textbengali{হলে দেখাও যে,} $\sqrt[3]{x - \sqrt{y}} = a - \sqrt{b}$.
	
%147
	\item If $x(x+1) = 1$, find the value of $(x-1)^3 - \dfrac{1}{(x-1)^3}$.
	
%148
	\item If $(x-7)(x-5) = 1$, find the value of $(x-5)^3 - \dfrac{1}{(x-5)^3}$.
	
%149
	\item If $a$, $b$, $c$ are real numbers, find the minimum and maximum values of $\dfrac{ab + bc + ca}{a^2 + b^2 + c^2}$.
	
%150
	\item If $\dfrac{a+b}{c} + \dfrac{b-c}{a} + \dfrac{c-a}{b} = 1$ and $(a+b-c) \neq 0$, prove that, $\dfrac{1}{a} + \dfrac{1}{c} = \dfrac{1}{b}$.
	
%151
	\item If $\dfrac{x+y}{x-y} = \dfrac{u}{v}$, prove that, $\dfrac{x^2 + xy}{xy - y^2} = \dfrac{u^2 - uv}{uv - v^2}$.
	
%152
	\item If $x+y+z = xyz$, prove that, $\dfrac{y+z}{1-yz} + \dfrac{z+x}{1-zx} + \dfrac{x+y}{1-xy} = \dfrac{y+z}{1-yz} \cdot \dfrac{z+x}{1-zx} \cdot \dfrac{x+y}{1-xy}$.
	
%153
	\item If $ab+bc+ca = abc$, prove that, $\dfrac{b+c}{bc(a-1)} + \dfrac{c+a}{ca(b-1)} + \dfrac{a+b}{ab(c-1)} = 1$.

%154
	\item \textbengali{যদি} $ax^2 + bx + c = 0$ \textbengali{সমীকরণের একটি বীজ অপরটির বর্গ হয়, তবে প্রমাণ কর যে,} $b^3 + ac^2 + a^2 c = 3abc$.
	
%155
	\item $ax^2 + bx + c = 0$ \textbengali{সমীকরণের বীজদ্বয়} $\alpha$ \textbengali{ও} $\beta$ \textbengali{হলে, প্রমাণ কর যে,} $a(x-\alpha) (x-\beta) = ax^2 + bx + c$.
	
%156
	\item $x^2 - px + q = 0$ \textbengali{সমীকরণের বীজদ্বয়ের সমষ্টি তাদের অন্তরের তিনগুণ হলে দেখাও যে,} $2p^2 = 9q$.
	
%157
	\item \textbengali{যদি} $x^2 - px + q = 0$ \textbengali{সমীকরণের বীজদ্বয়ের অন্তর} $1$ \textbengali{হয়, হবে দেখাও যে,} $p^2 + 4q^2 = (1+2q)^2$.
		\begin{center}
		\textbengali{অথবা}
		\end{center}
		\textbengali{যদি} $x^2 - px + q = 0$ \textbengali{সমীকরণের বীজদ্বয় দুটি ক্রমিক অখণ্ড সংখ্যা হয়,}  \textbengali{তবে দেখাও যে,} $p^2 + 4q^2 = (1+2q)^2$.
		
%158
	\item $x^2 + px + q = 0$ \textbengali{সমীকরণের একটি বীজ অপরটির বর্গ। প্রমাণ কর যে,} $p^3 + q + q^2 = 3pq$.
	
%159
	\item \textbengali{যদি} $ax^2 + bx + c = 0$ \textbengali{সমীকরণের বীজদ্বয়} $\alpha$ \textbengali{ও} $\beta$ \textbengali{হয়, তবে দেখাও যে,} $\dfrac{1}{a\alpha + b} + \dfrac{1}{a\beta + b} = \dfrac{b}{ac}$.
	
%160	
	\item If $\dfrac{b}{a+b} = \dfrac{a+c-b}{b+c-a} = \dfrac{a+b+c}{2a+b+2c}$ and $(a+b+c) \neq 0$, prove that, $\dfrac{a}{2} = \dfrac{b}{3} = \dfrac{c}{4}$.
	
%161
	\item If $\dfrac{5a-3b}{a} = \dfrac{4a+b-2c}{a+4b-2c} = \dfrac{a+2b-3c}{4a-4c}$, prove that, $\dfrac{a}{2} = \dfrac{b}{3} = \dfrac{c}{4}$.
	
%162
	\item Find the minimum value of $|z| + |z-1|$.
	
%163
	\item Find the minimum and maximum value of $|z|$ satisfying the equation $\left| z - \dfrac{4}{z} \right|= 2$.
	
%164
	\item If $x = \dfrac{\sqrt{a^2 + b^2} + \sqrt{a^2 - b^2}}{\sqrt{a^2 + b^2} - \sqrt{a^2 - b^2}}$, show that $b^2x^2 - 2a^2x+b^2 = 0$.
	
%165
	\item Simplify : $\dfrac{3\sqrt{2}}{\sqrt{3} + \sqrt{6}} - \dfrac{4\sqrt{3}}{\sqrt{6} + \sqrt{2}} + \dfrac{\sqrt{6}}{\sqrt{3} + \sqrt{2}}$.
	
%166
	\item If $x = \dfrac{\sqrt[3]{a}}{\sqrt[3]{b}} + \dfrac{\sqrt[3]{b}}{\sqrt[3]{a}}$, show that, $a(bx^3 - 3bx -a) = b^2$.
	
%167
	\item If $a = \dfrac{1}{2} \left[ x + \sqrt{\dfrac{4y^3 - x^3}{3x}} \right]$, $b = \dfrac{1}{2} \left[ x - \sqrt{\dfrac{4y^3 - x^3}{3x}} \right]$; show that $a^3 + b^3 = y^3$.
	
%168
	\item Prove that, the simplified form of $\sqrt{2} \left[ 2x + \sqrt{x^2 - y^2} \right] \cdot \left[ \sqrt{x - \sqrt{x^2 - y^2}} \right]$ is $\left[ \sqrt{(x+y)^3} - \sqrt{(x-y)^3} \right]$.
	
%169
	\item If $a$, $b$, $c$, $d$ are positive integers with a sum of $63$, what is the minimum value of \\ $(ab+bc+cd) ?$
	
%170
	\item $S = \{1, 2, \cdots , n\}$. Averages of the numbers of each of the subsets of $S$ are taken. Let $T_n$ be the no. of subsets having integer average. Prove that, $(T_n - n)$ is an even number.
	
%171
	\item $S = \{1, 2, \cdots , n\}$. Consider all the $r$-elment subsets of $S$. Then take the minimum element of each subset. Prove that, the average of all these numbers $= \dfrac{n+1}{r+1}$.
	
%172
	\item Suppose $(a, b, c)$ si an order-triplet such that $abc = 2310$. Find $\sum \limits_{\substack{abc = 2310 \\ a, b, c \in \mathbb{N}}} (a+b+c)$.
	
%173
	\item A plane is coloured with colours blue and red. Show that, all the equilateral triangles that can be formed on the plane have vertices of same coloured points.
	
%174
	\item The points on a straight line is coloured either Red or Blue. Show that, three equidistant points should have the same colour.
	
%175
	\item Let $a_3, a_4, \ldots, a_{2005}$ be real numbers with $a_{2006} \neq 0$. Prove that, there are not more than $2004$ real numbers $x$ such that $1+x+x^2 + a_3x^3 + a_4x^4 + \ldots + a_{2006}x^{2006} = 0$.

\end{enumerate}





\section{Factorization}

\begin{enumerate}

%01

	\item $ x^2 + 4x + 1 .$ (\textbengali{মধ্যপদ বিশ্লেষণের মাধ্যমে})

%02

	\item $ (a^2 - b^2) (x^2 + y^2) + 2(a^2 + b^2)xy. $
	
%03
	\item $ x^4 - 3x - 2 $.
	
%04
	\item $ x^4 -21x + 8 $.
	
%05
	\item $ (x-3)(x-4) - \dfrac{34}{33^2} $.
	
%06
	\item $ (a+b+c)^3 - a^3 - b^3 - c^3 $.
	
%07
	\item $ a(b^2 + c^2) + b(c^2 + a^2) + c(a^2 + b^2) + 3abc $.
	
%08
	\item $ a^4(b-c) + b^4(c-a) + c^4(a-b) $.
	
%09
	\item $ a(b+c)^2 + b(c+a)^2 + c(a+b)^2 - 4abc $.
	
%10
	\item $ a(b^3 - c^3) + b(c^3 - a^3) + c(a^3 - b^3) $.
	
%11
	\item $ x(1+y^2) (1+z^2) + y(1+z^2) (1+x^2) + z(1+x^2) (1+y^2) + 4xyz $.
	
%12
	\item $ x^4 + 5x^3 + 11x^2 + 13x + 6 $.

%13
	\item $ x^4 + 4x^3 - 2x^2 - 12x + 9 $.
	
%14
	\item $ 2a^3 + 11a^2 - 26a - 35 $.
	
%15
	\item $ a^4 - 6a^3 + 7a^2 + 6a - 8 $.
	
%16
	\item $ 4a^4 - 12a^3 - 7a^2 + 32a - 16 $.
	
%17
	\item $ x^6 - 8x^3 + 27 $.
	
%18
	\item $ x^6 + 14x^3 - 1 $.
	
%19
	\item $ x^4 - 4x^3 - 11x^2 + 12x + 9 $.

%20
	\item $ a^2(b+c) + b^2(c+a) + c^2(a+b) + 2abc $.

%21
	\item $ a^2(b+c) + b^2(c+a) + c^2(a+b) + 3abc $.

%22
	\item $ ab(a-b) + bc(b-c) + ca(c-a) $.

%23
	\item $ (a+b+c)(ab+bc+ca) - abc $.
	
%24
	\item $ (x-a)^3 (b-c)^3 + (x-b)^3 (c-a)^3 + (x-c)^3 (a-b)^3 $.

%25
	\item $ (x+1) (x+3) (x-4) (x-12) - 24x^2 $.
	
%26
	\item $ \dfrac{a}{b} + \dfrac{b}{c} + \dfrac{c}{a} + \dfrac{a}{c} + \dfrac{c}{b} + \dfrac{b}{a} + 3 $.
	
%27
	\item $ a(a+1)x^2 + (a+b)xy - b(b-1)y^2 $.
	
%28
	\item $ x^4 - 5x^3y + 6x^2y^2 - 5xy^3 + y^4  $.
	
%29
	\item $ x^4 + x^3 - 2x^2 - x + 1  $.
	
%30
	\item $ n^4 + 6n^3 + 11n^2 + 6n + 1 $.
	
%31
	\item $ x^8 + 98x^4 + 1 $.

%32
	\item $ x^4 + 3x +20 $.
	
%33
	\item $ (x^2 - 3)(x + 1)^2 + x^2 $.
	
%34
	\item $ x^4 - 7x - 12 $.



\end{enumerate}





\section{Construction}

\begin{enumerate}

%01
	\item \textbengali{একটি ত্রিভুজের পরিসীমা} $14$ $c.m.$, \textbengali{ভূমি সংলগ্ন কোণদ্বয় } $80^{ \circ} $ \textbengali{ও}  $70^{\circ} $. \textbengali{এই ত্রিভুজের সমান ক্ষেত্রফল বিশিষ্ট একটি সামান্তরিক অঙ্কন কর যার একটি কোণ} $ 60^{\circ} $. 
	
%02
	\item \textbengali{একটি নির্দিষ্ট বৃত্তকে একটি নির্দিষ্ট ত্রিভুজের সদৃশকোণী ত্রিভুজে অন্তর্লিখিত  কর।} \begin{center}
	\textbengali{অথবা}
	\end{center}
	\textbengali{একটি নির্দিষ্ট বৃত্তে একটি নির্দিষ্ট ত্রিভুজের সদৃশকোণী করে একটি ত্রিভুজ পরিলিখিত কর।}
	
%03
	\item \textbengali{একটি ত্রিভুজ অঙ্কন কর যার ভূমি} $ 5$ $ c.m. $,\textbengali{অন্য দুটি বাহুর সমষ্টি} $ 8 $ $c.m. $ \textbengali{ও} $ 5 $ $ c.m. $ \textbengali{বাহু সংলগ্ন কোণ দুটির অন্তর} $ 30^{\circ} $.

%04
	\item \textbengali{একটি ত্রিভুজের সদৃশ ও ওপর একটি ত্রিভুজের ক্ষেত্রফলের সমান করে একটি ত্রিভুজ অঙ্কন কর।}
	
%05
	\item \textbengali{একটি ত্রিভুজ} $ABC$ \textbengali{এর মধ্যে ভূমি} $BC$ \textbengali{এর সঙ্গে সমান্তরাল এমন একটি সরলরেখা নির্ণয় কর যেটি ত্রিভুজটিকে সমান ক্ষেত্রফল বিশিষ্ট দুটি অংশে বিভক্ত করে।}
	
%06
	\item \textbengali{একটি ত্রিভুজ} $ABC$ \textbengali{এর ভূমি} $BC$ \textbengali{-এর সঙ্গে লম্ব এমন একটি সরলরেখা নির্ণয় কর যেটি ত্রিভুজটিকে সমান ক্ষেত্রফল বিশিষ্ট দুটি অংশে বিভক্ত করবে।}
	
%07
	\item $O$ \textbengali{কেন্দ্রীয় একটি বৃত্তের বহিঃস্থ বিন্দু} $P$ \textbengali{থেকে বৃত্তের ওপর একটি স্পর্শক অঙ্কন কর, কেন্দ্র} $O$ \textbengali{কে ব্যবহার না করে।}
	
%08
	\item $R$ \textbengali{ও} $r$ $(R > r)$ \textbengali{ব্যাসার্ধবিশিষ্ট দুটি বৃত্তের সরল সাধারণ স্পর্শক অঙ্কন কর।} 
	
%09
	\item $R$ \textbengali{ও} $r$ $(R > r)$ \textbengali{ব্যাসার্ধবিশিষ্ট দুটি বৃত্তের তির্যক সাধারণ স্পর্শক অঙ্কন কর।} 
	
%10
	\item $AB$ \textbengali{একটি নির্দিষ্ট সরলরেখার ওপর} $C$ \textbengali{একটি যেকোনো নির্দিষ্ট বিন্দু।} $C$ \textbengali{বিন্দুগামী একটি যেকোনো সরলরেখা} $CD$ \textbengali{-এর ওপর অবস্থিত} $P$ \textbengali{এমন একটি বিন্দু যে, } $\dfrac{AP}{PB} = \dfrac{AC}{BC}.$ $P$ \textbengali{বিন্দুটি নির্ণয় কর।}
	
%11
	\item \textbengali{একটি ত্রিভুজ এবং অপর একটি ত্রিভুজের উচ্চতা প্রদত্ত রয়েছে। প্রথম ত্রিভুটির ক্ষেত্রফলের সমান করে দ্বিতীয় ত্রিভুজটি অঙ্কন কর। এখানে প্রথম ত্রিভুজের উচ্চতা} $>$ \textbengali{দ্বিতীয় ত্রিভুজের উচ্চতা।}
	
%12
	\item \textbengali{একটি ত্রিভুজ} $ABC$ \textbengali{এর সমান ক্ষেত্রফল বিশিষ্ট অপর একটি ত্রিভুজ অঙ্কন কর যেখানে দ্বিতীয় ত্রিভুজটির উচ্চতা প্রদত্ত। এখানে} $\bigtriangleup ABC$ \textbengali{এর} $BC$ \textbengali{ভূমি সাপেক্ষে উচ্চতা} $<$ \textbengali{দ্বিতীয় ত্রিভুজটির উচ্চতা, দ্বিতীয় ত্রিভুজটির ভূমি ও} $BC$ \textbengali{বাহু একই সরলরেখায় অবস্থিত।}
	
%13
	\item \textbengali{একটি ত্রিভুজ আঁক  যার পাদত্রিভুজের শীর্ষবিন্দুগুলি দেওয়া আছে।}
	
%14
	\item \textbengali{একটি নির্দিষ্ট বিন্দুগামী বৃত্ত আঁক যাহা একটি নির্দিষ্ট প্রদত্ত রেখা ও একটি নির্দিষ্ট প্রদত্ত বৃত্তকে স্পর্শ করে।}
	
%15
	\item $ABCD$ \textbengali{সামান্তরিকের কর্ণদ্বয়} $AC$ \textbengali{ও} $BD$ \textbengali{পরস্পরকে} $O$ \textbengali{বিন্দুতে ছেদ করেছে। সামান্তরিকের মধ্যে একটি বিন্দু} $P$. $P$ \textbengali{বিন্দুগামী একটি সরলরেখা নির্ণয় কর যা সামান্তরিকটিকে সমান ক্ষেত্রফলবিশিষ্ট দুটি অংশে বিভক্ত করে।}
	
%16
	\item $\bigtriangleup ABC$ \textbengali{এর সমান ক্ষেত্রফলবিশিষ্ট একটি সামান্তরিক আঁক যার একটি কোণ নির্দিষ্ট এবং সন্নিহিত বাহুর অনুপাত} $3:2$.
	
%17
	\item \textbengali{একটি বৃত্ত অঙ্কন কর যাহা দুটি ছেদী সরলরেখাকে স্পর্শ করেছে এবং একটি নির্দিষ্ট বিন্দুগামী।}
	
%18
	\item \textbengali{তিনটি সমান্তরাল সরলরেখা প্রদত্ত রয়েছে। এমন একটি সমবাহু ত্রিভুজ অঙ্কন করতে হবে যার শীর্ষবিন্দুগুলি প্রদত্ত তিনটি সমান্তরাল সরলরেখার ওপর অবস্থিত হবে।}
	
%19
	\item \textbengali{যেকোনো একটি ত্রিভুজের সমান ক্ষেত্রফলবিশিষ্ট অপর একটি সমবাহু ত্রিভুজ অঙ্কন কর।}
	
%20
	\item \textbengali{একটি বর্গক্ষেত্রের সমান ক্ষেত্রফলবিশিষ্ট একটি সমবাহু ত্রিভুজ অঙ্কন কর।}
	
%21
	\item \textbengali{একটি ত্রিভুজ আঁক যার ভূমি, পরিকেন্দ্র ও অপর বাহুদ্বয়ের সমষ্টি প্রদত্ত রয়েছে।}
	
%22
	\item \textbf{\textbengali{মাধ্যমিক ছেদ} / Medial Section} : $AB$ \textbengali{একটি রেখাংশ।} $AB$ \textbengali{কে}  $X$ \textbengali{বিন্দুতে এমনভাবে বিভক্ত কর যেন} $AB \cdot BX = AX^2$ \textbengali{হয়।}
	
%23
	\item \textbengali{একটি সমদ্বিবাহু ত্রিভুজ অঙ্কন কর যাহার ভূমিসংলগ্ন কোণদ্বয়ের প্রত্যেকে শীর্ষকোণের দ্বিগুণ।}
	
%24
	\item \textbengali{চতুর্ভুজের কোনো কৌণিক বিন্দু থেকে সরলরেখা টেনে চতুর্ভুজটিকে সমদ্বিখণ্ডিত কর।}
	
%25
	\item \textbengali{কোনো ত্রিভুজের শীর্ষকোণ} $60^{\circ}$, \textbengali{শীর্ষকোণ সংলগ্ন বাহুদ্বয়ের অনুপাত} $3:2$ \textbengali{এবং ভূমির দৈর্ঘ্য} $5 c.m.$ \textbengali{হলে ত্রিভুজটি অঙ্কন কর।}
	
%26
	\item Through a given point outside a given circle draw a secant so that the chord determined by it subtends an angle at the center equal to the acute angle between the secant and the diameter through the given point.
	
%27
	\item Draw two line segments $OA$ and $OB$ with $OA = 5$ $c.m.$, $OB = 8$ $c.m.$ and $\angle AOB = 60^{\circ}$. Then construct a circle such that it touches $OA$ at $A$ and $OB$ at any point [let $R$]. Find relatin between $OR$ and $OA$.
	
%28
	\item Draw $\bigtriangleup ABC$ with $AB = 5$ $c.m.$, $BC = 7$ $c.m.$ and $CA = 3$ $c.m.$ Then construct a circle such that it touches $AB$ at $B$ and passes through the point $C$.
	
%29
	\item Construct a right-angled triangle with hypotenuse $9$ $c.m.$ and difference between the other two sides as $5$ $c.m.$
	
%30
	\item Construct a circle passing through a fixed point and at the same time touching two parallel straight lines.
	
%31
	\item Divide a fixed straight line-segment into two parts such that the difference of the area of the two squares drawn on those two parts is always equal to the area of the fixed square.
	
%32
	\item Draw the mid-proportional of an $8$ $c.m.$ line-segment and one-third of it.
	
%33
	\item Construct a parallelogram of area twice as that of an equilateral triangle of sides $5\sqrt{2}$ $c.m.$
	
%34
	\item \begin{enumerate}[(a)]
		\item In $\bigtriangleup ABC$, $AD$ is a median. You have an unmarked straight edge like
		\begin{figure}[h]
		\centering
		\includegraphics[scale=0.15]{IMG_20231102_145518_873}
		\end{figure}
	\\ With the help of these two, draw a line parallel to $BC$.
		\item Now you are given two intersecting circles of different radius. Draw two parallel chords in any one of the circles.
		\item Find out the center of any one of the circles.
	\end{enumerate}

%35
	\item You are given a vertex $A$, center of the nine-point circle $N$ and centroid $G$. Construct the triangle.
\end{enumerate}




\section{Indices}
\begin{enumerate}

%01
	\item \textbengali{সমাধান কর} :- $ a^{2x^2} + a^{2x+12} = 2\cdot a^{x^2+x+6} $.
	
%02
	\item If $ ax^{10} = by^{10} = cz^{10} $ and $ \dfrac{1}{x} + \dfrac{1}{y} + \dfrac{1}{z} = 1 $, then prove that,\begin{center} $ \left( ax^9 + by^9 + cz^9 \right)^{\frac{1}{10}} = a^{\frac{1}{10}} + b^{\frac{1}{10}} + c^{\frac{1}{10}} $. \end{center}
	
%03
	\item Find the value of $x$ : $ \left(\sqrt{3} + \sqrt{2} \right)^x + \left(\sqrt{3} - \sqrt{2} \right)^x = 10 $.
	
%04
	\item If $ a+b+c = 0 $, show that $ \sqrt[bc]{\dfrac{x^{a^2}}{x^{bc}}} \cdot \sqrt[ac]{\dfrac{x^{b^2}}{x^{ac}}} \cdot \sqrt[ab]{\dfrac{x^{c^2}}{x^{ab}}}  = 1 $.
	
%05
	\item Solve :- $ 6(4^x + 9^x) = 13 \cdot 6^x $.
	
%06
	\item Solve :- $ \dfrac{2^x + 2^{-x}}{2^x - 2^{-x}} = \dfrac{16^{\frac{1}{x}} + 16^{-\frac{1}{x}} }{16^{\frac{1}{x}} - 16^{-\frac{1}{x}}}$.
	
%07
	\item Find the simplest value of $ \big[ 1-\{1-(1-x^3)^{-1}\}^{-1}\big]^{-\frac{1}{3}} $ when $ x = 0.1 $.
	
%08
	\item Solve :- $ 5^{13 - 2x} + 2^{x-2} = 2^{x+2} + 5^{11-2x} $.
	
%09
	\item If $ 2^x + 2^{x+2} = 5 $, find the value of $ (x+1) $.
	
%10
	\item $ \left(  3^{3^{n}} - 2^{3^{n}} \right) \div \left(  3^{3^{n-1}} - 2^{3^{n-1}} \right) $ \textbengali{এর মান কত?}
	
%11
	\item Solve :- $ 6^{3-4x} \cdot 4^{x+5} = 8 $ when $ \log 2 = 0.3010 $ and $ \log 3 = 0.4771 $.
	
%12
	\item If $ \sqrt{x^2 + \sqrt[3]{x^4y^2}} + \sqrt{y^2 + \sqrt[3]{x^2y^4}} = a $, show that $ x^{\frac{2}{3}} + y^{\frac{2}{3}} = a^{\frac{2}{3}} $.
	
%13
	\item If $ x = 2 + 2^{\frac{1}{3}} + 2^{\frac{2}{3}} $, show that $ x^3 - 6x^2 + 6x - 2 = 0 $.


\end{enumerate}


\section{Trigonometry}

\begin{enumerate}

%01
	\item Prove that, $ 1 + \sin^2 \alpha + \sin^2 \beta > \sin \alpha + \sin \beta + \sin \alpha \sin \beta $ [By using algebra].
	
%02
	\item $\tan 2\theta + \cot 2\theta = 2$ \textbengali{হলে} $ \theta$ \textbengali{-এর বৃত্তীয় মান কত} ?
	
%03
	\item $\tan^2 \theta - \sin^2 \theta = p$ \textbengali{হলে} $\tan^2 \theta \cdot \sin^2 \theta$ = \textbengali{কত} ?
	
%04
	\item If $\tan^2 \theta = 1 - a^2$, prove that $\sec \theta + \tan^3 \theta \cdot \mathrm{cosec\,} \theta = (2 - a^2)^{\dfrac{3}{2}}$.


%05
	\item If $\sin \theta + \sin^2 \theta + \sin^3 \theta = 1$, prove that $ \cos^6 \theta - 4\cos^4 \theta + 8 \cos^2 \theta = 4 $.
	
%06
	\item Find the value of $$ \dfrac{\sin^2 20^{\circ} + \sin^2 70^{\circ}}{\cos^2 20^{\circ}+\cos^2 70^{\circ}} + \dfrac{\sin(90^{\circ}-\theta) \sin \theta}{\tan \theta} + \dfrac{\cos(90^{\circ}-\theta) \cos \theta}{\cot \theta} .$$
	
%07
	\item If $ \theta + \phi = 60^{\circ} $, show that $ \cos \theta = \sin (30^{\circ} + \phi) $.
	
%08
	\item In a triangle $ABC$, prove that $$ \dfrac{a}{\sin A} = \dfrac{b}{\sin B} = \dfrac{c}{\sin C} = 2R .$$
	
%09
	\item If $ \tan n\theta = n \tan \theta $, prove that $ \left( \dfrac{\sin n\theta}{\sin \theta} \right)^2 = \dfrac{n^2}{1+(n^2 - 1)\sin^2 \theta} .$
	
%10
	\item \textbengali{একই সমতলে অবস্থিত} $R$ \textbengali{ও} $r$ $(R>r)$ \textbengali{ব্যাসার্ধবিশিষ্ট দুইটি চাকা} $2s$ \textbengali{দৈর্ঘ্যবিশিষ্ট একটি মেখলার} (belt) \textbengali{দ্বারা সরলভাবে টান-টান করিয়া সংযুক্ত রহিয়াছে। মেখলাটির সরলরৈখিক অংশ কেন্দ্রদ্বয়ের সংযোজক রেখার সহিত যে কোণ উৎপন্ন করে তাহার পূরক কোণের বৃত্তীয় মান} $\theta$ \textbengali{হলে প্রমাণ কর যে,} 
	$$ s = \pi R + (R - r) (\tan \theta - \theta). $$
	
%11
	\item If $\sec \alpha = \sec \beta \sec \gamma + \tan \beta \tan \gamma$, prove that, $\sec \beta = \sec \alpha \sec \gamma \pm \tan \gamma \tan \alpha $.
	
%12
	\item If $\dfrac{\cos^4 A}{\cos^2 B} + \dfrac{\sin^4 A}{\sin^2 B} = 1$, prove that, $\dfrac{\cos^4 B}{\cos^2 A} + \dfrac{\sin^4 B}{\sin^2 A} = 1$.
	
%13
	\item If $\dfrac{\sin^4 \theta}{a} + \dfrac{\cos^4 \theta}{b} = \dfrac{1}{a+b}$, show that, $\dfrac{\sin^8 \theta}{a^3} + \dfrac{\cos^8 \theta}{b^3} = \dfrac{1}{(a+b)^3}$.
	
%14
	\item If $a\cos \theta - b \sin \theta = c$, prove that, $a\sin \theta + b \cos \theta = \pm \sqrt{a^2 + b^2 - c^2}$.
	
%15
	\item Prove that, $\dfrac{(1 - \tan x)^2}{(1 - \cot x)^2} = \dfrac{1 + \tan^2 x}{1 + \cot^2 x}$.
	
%16
	\item Prove that, $\dfrac{(\mathrm{cosec\,} \theta \tan \phi)^2 + 1}{(\mathrm{cosec\,} \psi \tan \phi)^2 + 1} = \dfrac{1 + (\cot \theta \sin \phi)^2}{1 + (\cot \psi \sin \phi)^2}$.
	
%17
	\item Find the value of $\cot \theta$ where it is given that $$(l^2 - m^2) \sin \theta + 2lm\cos \theta - (l^2 + m^2) = 0.$$
	
%18
	\item If $\sin \theta = \dfrac{\sin \alpha + \sin \beta}{1 + \sin \alpha \sin \beta}$, prove that, $\cos \theta = \pm \dfrac{\cos \alpha \cos \beta}{1 + \sin \alpha \sin \beta}.$
	
	
%19
	\item Find the value of $\theta$ where $\dfrac{3\cos \theta - 4 \sin^2 \theta \cos \theta}{4\sin \theta \cos^2 \theta - \sin \theta} = \tan 60^{\circ}.$
	
%20
	\item If $\sin 2A = 2 \sin A \cos A$, prove that, $$\sin x = 2^x \cos \dfrac{x}{2} \cos \dfrac{x}{x^2} \cos \dfrac{x}{2^3} \cdots \cos \dfrac{x}{2^n} \sin \dfrac{x}{2^n}$$ and if $x = \dfrac{\pi}{2(2^n + 1)}$, again show that, $$2^n \sin x \cos 2x \cos 2^2x \cdots \cos 2^{n-1}x = 1.$$
	
%21
	\item If $\sin A + \cos B = c$ and $\sin B + \cos A = d$, show that, $$c \sin A + d \cos A = c \cos B + d \sin B = \dfrac{1}{2}(c^2 + d^2).$$
	
%22
	\item If $x \sin \alpha = y \cos \alpha = \dfrac{2\tan \alpha}{1 - \tan^2 \alpha}$, prove that, $(x^2 - y^2)^2 = 4(x^2 + y^2).$
	
%23
	\item If $(a^2 - b^2) \sin \theta + 2ab \cos \theta = a^2 + b^2$, show that, $\tan \theta = \pm \left( \dfrac{a^2 - b^2}{2ab} \right).$
	
%24
	\item If $\mathrm{cosec\,} \alpha - \sin \alpha = m^3$ and $\sec \alpha - \cos \alpha = n^3$, show that, $m^2n^2(m^2 + n^2) = 1.$
	
%25
	\item Show, if $\sin \theta = \dfrac{(x+y)^2}{4xy}$ possible or not, where $x\neq y$ and $x$, $y$ are two real numbers.
	
%26
	\item If $\mathrm{cosec\,} \theta - \sin \theta = m$ and $\sec \theta - \cos \theta = n$, find the value of $(m^2n)^{\frac{2}{3}} + (mn^2)^{\frac{2}{3}}.$

%27
	\item If $\cos^2 \alpha - \sin^2 \alpha = \tan^2 \beta$, show that, $\cos^2 \beta - \sin^2 \beta = \tan^2 \alpha.$
	
%28
	\item If $p_n = \sin^n \theta + \cos^n \theta$ and $p_6 - p_4 = kp_2$, find the value of $k$.
	
%29
	\item If $\sin^2 \theta = \cos^3 \theta$, show that, $\cot^6 \theta - \cot^2 \theta = 1.$
	
%30
	\item If $\cos \theta + \sin \theta = \sqrt{2}\cos \theta$, show that, $\cos \theta - \sin \theta = \sqrt{2}\sin \theta.$
	
%31
	\item If $a(\tan \theta + \cot \theta) = 1$ and $\sin \theta + \cos \theta = b$, show that, $2a = b^2 - 1.$
	
%32
	\item If $\tan \theta + \sin \theta = m$ and $\tan \theta - \sin \theta = n$, show that, $m^2 - n^2 = 4\sqrt{mn}.$
	
%33
	\item If
		\begin{align*}
			m^2 + m_1 ^2 + 2m m_1 \cos \theta &= 1, \\
			n^2 + n_1 ^2 + 2n n_1 \cos \theta &= 1, \\
			mn + m_1 n_1 + (m_1 n + m n_1) \cos \theta &= 0;
		\end{align*}
		show that, $$ m^2 + n^2 = m_1 ^2 + n_1 ^2 = \mathrm{cosec^2\,} \theta. $$
		
%34
	\item \textbengali{সকাল} $8$ \textbengali{টার সময় একটি স্তম্ভের ছায়ার দৈর্ঘ্য} $16$ $c.m.$ \textbengali{দুপুর} $2$ \textbengali{টোর সময় ওই স্তম্ভের ছায়ার দৈর্ঘ্য} $9$ $m.$ \textbengali{স্তম্ভটির উচ্চতা নির্ণয় কর।}
	
%35
	\item \textbengali{প্রমাণ কর যে,} $\sin \theta = x + \dfrac{1}{x}$ \textbengali{সমাধানযোগ্য নয়।}
	
%36
	\item \textbengali{একটি} $r$ \textbengali{ব্যাসার্ধবিশিষ্ট গোলকাকার বেলুন একজন পর্যবেক্ষকের চোখে} $\alpha$ \textbengali{কোণ উৎপন্ন করে। যদি বেলুনটির কেন্দ্রের উন্নতি কোণ} $\beta$ \textbengali{হয়, তবে প্রমাণ কর যে, বেলুনটির কেন্দ্রের উচ্চতা} $= r \, \mathrm{cosec\,} \dfrac{\alpha}{2} \sin \beta$.
	
%37
	\item \textbengali{একজন লোক কোনো পাহাড়কে} $45^{\circ}$ \textbengali{উন্নতি কোণে দেখল। পাহাড়ের ঢাল} $30^{\circ}$ \textbengali{কোণে নত। সেই ঢাল বেয়ে} $1$ $km.$ \textbengali{যাওয়ার পর সেই ব্যক্তি পাহাড়কে} $60^{\circ}$ \textbengali{উন্নতি কোণে দেখল। পাহাড়টির উচ্চতা কত?}
	
%38
	\item From teh top of a mountain the angles of depression of three consecutive milestones on a straight road are observed to be $\alpha$, $\beta$, $\gamma$ respectively. Find the height of the mountain.
	
%39
	\item If $a \sin^2 \theta + b \cos^2 \theta = c$, $b \sin^2 \phi + a \cos^2 \phi = d$ and $a \tan \theta = b \tan \phi$, show that, $\dfrac{1}{a} + \dfrac{1}{b} = \dfrac{1}{c} + \dfrac{1}{d}$.
	
%40
	\item If $a\sin \theta + b \cos \theta = a \, \mathrm{cosec}\, \theta + b \sec \theta$, show that, L.H.S. = R.H.S. = $\left( a^{\frac{2}{3}} - b^{\frac{2}{3}} \right) \sqrt{a^{\frac{2}{3}} + b^{\frac{2}{3}}}$.
	
%41
	\item If $x = \dfrac{2 \sin \theta}{1 + \sin \theta + \cos \theta}$, find the value of $\dfrac{1-\cos \theta + \sin \theta}{1 + \sin \theta}$.
	
%42
	\item If $\left( 1 + 4x^2 \right) \cos A = 4x$, show that, $\mathrm{cosec} \, A + \cot A = \dfrac{1+2x}{1-2x}$.
	
%43
	\item If $2y \cos \theta = x \sin \theta$ and $2x \sec \theta - y \mathrm{cosec} \, \theta = 3$, show that, $x^2 + 4y^2 = 4$.
	
%44
	\item Eliminate $\alpha$ and $\beta$ from the following equations.
	\begin{align*}
	a \sin \alpha &= b \sin \beta \\
	a \cos^2 \alpha + b \cos^2 \beta &= 1 \\
	a \cot^2 \alpha + b \cot^2 \beta &= 1.	
	\end{align*}

%45
	\item State TRUE or FALSE : $\dfrac{\sin \theta \tan \theta}{1 - \cos \theta} < 2$.
	
%46
	\item If $\tan \alpha = \dfrac{\sin \beta - \cos \beta}{\sin \beta + \cos \beta}$, prove that, $\sin \beta + \cos \beta = \sqrt{2} \cos \alpha$.
	
%47
	\item If $\sin \theta + \tan \theta = a$ and $\cos \theta + \cot \theta = b$, prove that, $\dfrac{1}{(1+a)^2} + \dfrac{1}{(1+b)^2} = \dfrac{1}{(1-ab)^2}$.
	
%48
	\item \textbengali{একটি স্তম্ভের গোড়া থেকে কিছু দূরে একটি বিন্দু থেকে ওই স্তম্ভের উন্নতি কোণ} $\theta$ \textbengali{যেখানে} $\tan \theta = \dfrac{3}{4}$. \textbengali{ওই বিন্দু থেকে} $192$ m. \textbengali{পিছিয়ে যাওয়ায় উন্নতি কোণ} $\phi$ \textbengali{হল যেখানে} $\tan \theta = \dfrac{5}{12}$. \textbengali{স্তম্ভটির উচ্চতা নির্ণয় কর।}
	
%49	
	\item If $\tan \theta = \dfrac{2n}{n^2 - 1}$, prove that, $(n+2)\sin \theta + (2n - 1)\cos \theta = (2n+1)$.
	
%50
	\item If $A + B = 45^{\circ}$, find the value of $n$ from the following relation $\displaystyle{\prod \limits_{i = 1^{\circ}}^{45^{\circ}} \left( 1 + \tan i \right) = 2^n}$.
	
%51
	\item Find the minimum value of $2^{\sin^2 \theta} + 2^{\cos^2 \theta}$.
	
%52
	\item Find the minimum value of $2^{\sin \theta} + 2^{\cos \theta}$.
	
%53
	\item If $\dfrac{\cos 30^{\circ} - \sin 20^{\circ}}{\cos 40^{\circ} + \cos 20^{\circ}} = k \cos 40^{\circ} \cos 80^{\circ}$, find the value of $k$.
	
%54
	\item Prove that, $\sqrt{3} \cot 20^{\circ} - 4 \cos 20^{\circ} = 1$.
	
%55
	\item If $\sqrt{n-1} \tan \alpha = \sqrt{n+1} \tan \beta$, express $\cos 2\alpha$ in terms of $\cos 2\beta$.
	
%56
	\item If $\alpha$, $\beta$ are two angles satisfying the relation $a\cos 2\theta + b \sin 2\theta = c$, show that, 
	\begin{enumerate}[(i)]
		\item $\cos^2  \alpha + \cos^2 \beta = 1 + \dfrac{ac}{a^2 + b^2}$
		\item $\tan \alpha + \tan \beta = \dfrac{2b}{c+a}$
		\item $\tan \alpha \tan \beta = \dfrac{c-a}{c+a}$.
	
	\end{enumerate}
	
	
\end{enumerate}

\section{Functional Equations}

\begin{enumerate}

%01
	\item $ f(x+2) = 2x^2 +5x+7 $ \textbengali{হলে} $ f(1) $ \textbengali{কত}?
	

%02
	\item $ 4f(x) + 3f(-x) = 7 - 3x $ \textbengali{হলে} $ f(-1) = $  \textbengali{কত?}

\end{enumerate}

\section{System of Equations}

\begin{enumerate}

%01
	\item Solve :- $ 2^x + 2^y = 12  $; $ x+y = 5 $.
%02
	\item Solve :- $ x^y = y^x $, $ x^a = y^b $.
	
%03
	\item Solve :- $ x^y = y^x $, $ x = 2y $.
	
%04
	\item Solve :- $ 999x + 888y = 1332 $, $ 888x + 999y = 555 $.

%05
	\item Reduce $\theta$ from the following relations:-
	\begin{align*}
	x\cos \theta - y \sin \theta &= 0 \\
	x\cos^3 \theta + y \sin^3 \theta &= \sin \theta \cos \theta.
	\end{align*}
	
%06
	\item Solve :- $\left( x - 9 \right) \left( x - 12 \right) = \dfrac{81}{64}$.
	
%07
	\item Solve :- $\dfrac{\sqrt[9]{24+x}}{x} + \dfrac{\sqrt[9]{24+x}}{24} = \dfrac{128}{3} \sqrt[9]{x}$.
	
%08
	\item Find the value of $a$ and $x$ in the following equation. $$a^x + x^a = 321.$$
	
%09
	\item $x + y = 2$, $ \dfrac{1}{x} + \dfrac{1}{y} = 2$ \textbengali{সমীকরণ দুটিকে অপনয়ন পদ্ধতিতে সমাধান কর।}
	
%10
	\item Solve for $x$ : $(x-2)(x-4) = \dfrac{45}{22^2}$.
	
%11
	\item Solve the system in $\mathbb{R^+}$ :
	\begin{align*}
	a + b + c + d &= 12 \\
	abcd &= 27 + ab + bc + ca + ad + bd + cd
	\end{align*}

%12
	\item Solve the system in the set of real numbers :
	\begin{align*}
	\dfrac{4x^2}{4x^2 + 1} &= y \\
	\dfrac{4y^2}{4y^2 + 1} &= z \\
	\dfrac{4z^2}{4z^2 + 1} &= x \\
	\end{align*}
	
%13
	\item Solve for all real numbers $x$ :
	$$(x^2 - 7x + 11)^{x^2 - 13x + 42} = 1.$$
	
%14
	\item Solve over the integers : $615 + x^2 = 2^y$.
	
%15
	\item Find out all the natural number solution $(x, y)$ of $x^3 + y^3 = x^2 + 42xy + y^2$.
\end{enumerate}


\section{Coordinate Geometry}

\begin{enumerate}

%01

	\item Find out the circumcentre of the triangle formed by the points $(-3,1)$, $(1,3)$, $(3,0)$.
	
%02
	\item Show that the points $ (2,2) $; $(-2,-2)$;$(-2\sqrt{3},2\sqrt{3})$ are the vertices of an equilateral triangle.
	
%03
	\item Find the ratio in which the point $(1,2)$ divides the line segment joining the points $(-3,8)$ $\&$ $(7,-7)$.
	
%04
	\item $(7, -10)$ \textbengali{ও} $(2, 5)$  \textbengali{বিন্দু দুটির সংযোজক রেখাংশকে}  $3x + 2y = 7$ \textbengali{সমীকরণের সরলরেখা কী অনুপাতে বিভক্ত করে ? বিভক্তকারী বিন্দুটির স্থানাংক নির্ণয় কর।}
	
%05
	\item $AB$ \textbengali{রেখাংশকে} $C$ \textbengali{ও} $D$ \textbengali{বিন্দু দুটি সমান তিনভাগে বিভক্ত করে।} $A$ \textbengali{ও} $B$ \textbengali{বিন্দু দুটির স্থানাংক যথাক্রমে} $(-2, 6)$ \textbengali{ও} $(7, -15)$ \textbengali{হলে} $C$ \textbengali{ও} $D$ \textbengali{বিন্দুর স্থানাংক নির্ণয় কর।}
	
%06
	\item The straight line $\dfrac{x}{a} + \dfrac{y}{b} = 1$ intersects the co-ordinate axes at two points $A$ \& $B$. The perpendicular straight line to $AB$ cuts both the axes at point $P$ \& $Q$ respectively. Find the locus of the point of intersection of $AQ$ \& $BP$.
	

\end{enumerate}



\section{Mensuration}

\begin{enumerate}

%01
	\item \textbengali{একটি বর্গাকার কাগজকে অর একটি কৌণিক বিন্দু থেকে বিপরীত বাহু পর্যন্ত একটি রেখাংশ বরাবর দুটি ভাগে ভাগ করা হল। এই খন্ডদুটির ক্ষেত্রফলের অনুপাত} $3:1$ \textbengali{হলে ছোট খণ্ডটি এবং মূল কাগজটির পরিসীমার অনুপাত কী হবে?}
	
%02
	\item $3$ \textbengali{টি লম্ব বৃত্তাকার চোঙের প্রত্যেকটির উচ্চতা} $20\,c.m.$ \textbengali{এবং ব্যাস} $12\,c.m.$ \textbengali{যদি চোঙ তিনটি পরস্পরকে স্পর্শ করে থাকে তবে তাদের দ্বারা সীমাবদ্ধ অংশের আয়তন নির্ণয় কর।}

%03
	\item $4\,c.m.$ \textbengali{ব্যাসার্ধবিশিষ্ট একটি চোঙাকৃতি জার অর্ধেক জলপূর্ণ ছিল। একটি} $3\,c.m.$ \textbengali{ব্যাসার্ধবিশিষ্ট গোলাকৃতি মার্বেল জারের মধ্যে ফেলা হল এবং দেখা গেল যে মার্বেলটি ঠিক সম্পূর্ণভাবে জলে নিমজ্জিত রয়েছে। জারের উচ্চতা নির্ণয় কর।}
	
%04
	\item $5\,c.m.$ \textbengali{ব্যাসার্ধ ও} $12\,c.m.$ \textbengali{উচ্চতাবিশিষ্ট একটি শঙ্কু আকৃতির পাত্র দিয়ে একটি গোলককে ঠিক ঢেকে দেওয়া যায়। গোলকটির ব্যাসার্ধ নির্ণয় কর।}
	
%05
	\item \textbengali{একটি লম্ব বৃত্তাকার চোঙ ও একটি শঙ্কুর ভূমিতলের ব্যাস সমান। এদের উচ্চতা সমান। যদি চোঙের বক্রতলের ক্ষেত্রফল শঙ্কুটির সমগ্রতলের ক্ষেত্রফলের সমান হয়, তবে শঙ্কুর উচ্চতা ও ব্যাসের অনুপাত কত ?}

\end{enumerate}


\section{Inequality}

\begin{enumerate}

%01
	\item Prove that for any positive real numbers $a$, $b$, $c$; we have,
$$\dfrac{1}{a+b} + \dfrac{1}{b+c} + \dfrac{1}{c+a} \geq \dfrac{9}{2(a+b+c)}.$$

%02
	\item Prove that for any positive real numbers $a$, $b$, $c$; we have,
$$(a+2)(b+3)(c+6) \geq 48 \sqrt{abc}.$$

%03
	\item If $a$, $b$, $c$ $>$ $0$, prove that, 
	$$\dfrac{ab}{c^2} + \dfrac{bc}{a^2} + \dfrac{ca}{b^2} \geq 3.$$
	
%04
	\item If $a$, $b$, $c$ $>$ $0$, prove that, 
	$$\dfrac{ab}{c} + \dfrac{bc}{a} + \dfrac{ca}{b} \geq a+b+c. $$
	
%05
	\item If $a$, $b$, $c$ $>$ $0$ and $a^2 + b^2 + c^2 = 3$, prove that, 
	$$\dfrac{1}{1+ab} + \dfrac{1}{1+bc} + \dfrac{1}{1+ca} \geq \dfrac{3}{2}.$$
	
%06	
	\item Prove that if $x$, $y$, $z$ are real numbers with $z > 0$, then $$\dfrac{x^2 + y^2 + 12z^2 + 1}{4z} \geq x + y + 1.$$

%07	
	\item Prove that the inequality $(3a + b + c)^2 \geq 12a(b+c)$ holds for any real numbers $a,b,c$.
	
%08
	\item If $x, y, z > 0$, prove that, 
	\begin{enumerate}[(i)]
	\item $\dfrac{1}{\sqrt{xy}} + \dfrac{1}{\sqrt{yz}} + \dfrac{1}{\sqrt{zx}} \leq \dfrac{1}{x} + \dfrac{1}{y} + \dfrac{1}{z}$.
	
	\item $\dfrac{x^2}{y^2} + \dfrac{y^2}{z^2} + \dfrac{z^2}{x^2} \geq \dfrac{x}{z} + \dfrac{z}{y} + \dfrac{y}{x}$.
	
	\item $\dfrac{x^2}{y} + \dfrac{y^2}{z} + \dfrac{z^2}{x} \geq x\sqrt{\dfrac{y}{z}} + y\sqrt{\dfrac{z}{x}} + z\sqrt{\dfrac{x}{y}}$.
	
	\item $x^4 + y^4 + z^4 \geq xyz(\sqrt{xy} + \sqrt{yz} + \sqrt{zx})$.
	
	\end{enumerate}

%09
	\item If $x, y > 0$, prove that, $\dfrac{1}{x+y} \leq \dfrac{1}{4x} + \dfrac{1}{4y}$.
	
%10
	\item Let $a, b, c \geq 0$ and $a+b+c < 3$, prove that, $$\dfrac{a}{1+a^2} + \dfrac{b}{1+b^2} + \dfrac{c}{1+c^2} \leq \dfrac{3}{2} \leq \dfrac{1}{1+a} + \dfrac{1}{1+b} + \dfrac{1}{1+c}.$$
	
%11
	\item Prove that the inequality $x^4 + y^4 + 8 \geq 8xy$ for positive real numbers $x, y$.
	
%12
	\item Prove that if $a$ and $b$ are positive real numbers, then $$\left( 1 + \dfrac{a}{b} \right)^n + \left( 1 + \dfrac{b}{a} \right)^n \geq 2^{n+1}.$$
	
%13
	\item Prove that if $x + y + z = 1$, then $$8 \left( \dfrac{1}{2} - xy - yz - zx \right) \Bigg\{\dfrac{1}{(x+y)^2} + \dfrac{1}{(y+z)^2} + \dfrac{1}{(z+x)^2} \Bigg\} \geq 9.$$
	
%14
	\item Let $a, b, c$ be positive real numbers. Prove that, $$\dfrac{(2a + b + c)^2}{2a^2 + (b+c)^2} + \dfrac{(2b + c + a)^2}{2b^2 + (c+a)^2} + \dfrac{(2c + a + b)^2}{2c^2 + (a+b)^2} \leq 8.$$
	
%15
	\item Prove that, if $a_1, a_2, \ldots a_n$ are distinct positive real numbers and $a_1 + a_2 + \ldots + a_n = S$, then
	$$\dfrac{S}{S - a_1} + \dfrac{S}{S - a_2} + \ldots + \dfrac{S}{S - a_n} > \dfrac{n^2}{n-1}.$$
	
%16
	\item Find the minimum value of 
	$$\dfrac{a_1}{1 + a_2 + a_3 + \ldots a_{2009}} + \dfrac{a_2}{1 + a_1 + a_3 + \ldots a_{2009}} + \dfrac{a_{2009}}{1 + a_1 + a_2 + \ldots a_{2008}}$$ where $a_1, a_2, \ldots, a_{2009} > 0$ and $a_1 + a_2 + \ldots a_{2009} = 1$.
	
%17
	\item Prove that the inequality
$$a^2 + b^2 + c^2 \geq \dfrac{(a+b+c)^2}{3} \geq ab + bc + ca$$ holds for any $a, b, c \in \mathbb{R}$.

%18
	\item If $x, y, z > 0$, prove that, $$\dfrac{x^2}{y^2} + \dfrac{y^2}{z^2} + \dfrac{z^2}{x^2} \geq \dfrac{x}{y} + \dfrac{y}{z} + \dfrac{z}{x}.$$
	
%19
	\item If $x^3 + y^3 = 2$, prove that, $x + y \leq 2$.
	
%20
	\item If $a, b, c, d > 0$, prove that, $$\dfrac{a}{b+c+d} + \dfrac{b}{a+c+d} + \dfrac{c}{a+b+d} + \dfrac{d}{a+b+c} \geq \dfrac{4}{3}.$$
	
%21
	\item Let $a, b, c > 0$. Prove that, $$\dfrac{a^3 - a +2}{b+c} + \dfrac{b^3 - b +2}{c + a} + \dfrac{c^3 - c +2}{a + b} \geq 3.$$
	
%22
	\item If $a+b \geq 1$, prove that, $a^4 + b^4 \geq \dfrac{1}{8}$.
	
%23
	\item If $a, b, c$ are positive real numbers such that $a+b+c = 1$, prove that, $$(1+a)(1+b)(1+c) \geq 8 (1-a)(1-b)(1-c).$$
	
%24
	\item Let $a, b, c$ be positive real numbers such that $a^2 + b^2 + c^2 + (a+b+c)^2 \leq 4$. Prove that, $$\dfrac{ab+1}{(a+b)^2} + \dfrac{bc+1}{(b+c)^2} + \dfrac{ca+1}{(c+a)^2} \geq 3.$$
	
%25
	\item For $a, b, c, x, y, z > 0$; prove the following inequality :
	$$\dfrac{x}{ay+bz} + \dfrac{y}{az+bx} + \dfrac{z}{ax+by} \geq \dfrac{3}{a+b}.$$
	
%26
	\item $a, b, c$ are the sides of a triangle. Prove that, $$a^2(b+c-a) + b^2(c+a-b) + c^2(a+b-c) \leq 3abc.$$
	
%27
	\item Let $x, y, z$ be positive real numbers such that $x + y + z = 1$. Prove that, $$\dfrac{x+y}{\sqrt{z+xy}} + \dfrac{y+z}{\sqrt{x+yz}} + \dfrac{z+x}{\sqrt{y+zx}} \geq 3.$$
	
%28
	\item In an acute angled $\bigtriangleup ABC$, the values of $\tan A$, $\tan B$, $\tan C$ are denoted by $a$, $b$, $c$ respectively. Prove the following inequality involving $\tan A$, $\tan B$, $\tan C$ :
	$$\left( \dfrac{1}{a} + \dfrac{1}{b} + \dfrac{1}{c} \right) \left( \dfrac{a+b}{ab-1} + \dfrac{b+c}{bc-1} + \dfrac{c+a}{ca-1} \right) \geq 9.$$
	
%29
	\item For $x, y, z > 0$, prove that, $$\dfrac{x^2 - z^2}{y+z} + \dfrac{y^2 - x^2}{z+x} + \dfrac{z^2 - y^2}{x+y} \geq 0.$$
	
%30
	\item For $a, b, c > 0$, prove that, $$\dfrac{ab}{a+b+2c} + \dfrac{bc}{b+c+2a} + \dfrac{ca}{c+a+2b} \leq \dfrac{a+b+c}{4}.$$
	
%31
	\item For $a, b, c > 0$, prove that, $$\sqrt{\dfrac{a+b}{c}} + \sqrt{\dfrac{b+c}{a}} + \sqrt{\dfrac{c+a}{b}} \geq 3\sqrt{2}.$$
	
%32
	\item For $a, b, c > 0$, prove that, $$\dfrac{abc}{(1+a)(a+b)(b+c)(c+16)} \leq \dfrac{1}{81}.$$
	
%33
	\item For $a, b, c, d > 0$ and $a+b+c+d = 4$, prove that, $$\dfrac{1}{a^2 + 1} + \dfrac{1}{b^2 + 1} + \dfrac{1}{c^2 + 1} + \dfrac{1}{d^2 + 1} \geq 2.$$
	
%34
	\item For $a, b, c \in \mathbb{R}$, prove that, $$a^5 + b^5 + c^5 \geq a^4b + b^4c + c^4a.$$
	
%35
	\item For $a, b, c > 0$, prove that, $$\dfrac{a^2 + c^2}{b} + \dfrac{b^2 + c^2}{a} + \dfrac{a^2 + b^2}{c} \geq 2(a+b+c).$$
	
%36
	\item For $a, b, c > 0$, prove that, $$(a^2 + b^2 + c^2)^3 \leq 3 (a^3 + b^3 + c^3)^2.$$
	
%37
	\item For $a, b, c \in \mathbb{R^+}$, prove that, $$\dfrac{(a+2b+3c)^2}{a^2 + 2b^2 + 3c^2} \leq 6.$$
	
%38
	\item If $a, b, c > 0$ and $\dfrac{1}{a} + \dfrac{1}{b} + \dfrac{1}{c} = 3$, show that, 
	$$\dfrac{1}{\sqrt{a^3+b}} + \dfrac{1}{\sqrt{b^3+c}} + \dfrac{1}{\sqrt{c^3+a}} \leq \dfrac{3}{\sqrt{2}}$$.
	
%39
	\item If $x, y, z$ are all positive real numbers, prove that, $x(1+y) + y(1+z) + z(1+x) \geq 6\sqrt{xyz}$.
	
%40
	\item If $a, b, c > 0$, prove that, $a^4 + b^4 + c^4 \geq a^2bc + ab^2c + abc^2$.
	
%41
	\item Prove that, $\sum \limits_{k = 1}^{100} \dfrac{1}{\sqrt{k}} < 20$.
	
%42
	\item If $a, b, c > 0$, prove that, $(a+b)(b+c)(c+a) \geq 8abc$.
	
%43
	\item If $a+b+c = 1$ and $a, b, c > 0$, prove that, $$\left( 1 + \dfrac{1}{a} \right) \left( 1 + \dfrac{1}{b} \right) \left( 1 + \dfrac{1}{c} \right) \geq 64.$$
	
%44
	\item Let $a, b, c$ be positive real numbers. Prove that, $$\dfrac{a}{(b+c)^2} + \dfrac{b}{(c+a)^2} + \dfrac{c}{(a+b)^2} \geq \dfrac{9}{4(a+b+c)}.$$
	
%45
	\item \textbf{[IMO 1995]} Let $a, b, c$ be positive real numbers with $abc  = 1$. Prove that, $$\dfrac{1}{a^3(b+c)} + \dfrac{1}{b^3(c+a)} + \dfrac{1}{c^3(a+b)} \geq \dfrac{3}{2}.$$
	
%46
	\item Let $a, b, c, x, y, z$ be positive real numbers such that $x + y + z = 1$. Prove that, $$ax + by + cz + 2\sqrt{(xy+yz+zx)(ab+bc+ca)} \leq a+b+c.$$
	
%47	
	\item If $a, b > 0$ and $a+b=1$, show that, $a^ab^b + a^bb^a \leq 1$.
	
%48
	\item If $a, b, c > 0$, prove that, $a^a b^b c^c \geq (abc)^{\frac{a+b+c}{3}}$.
	
%49
	\item If $a, b, c, d > 0$ and $(a^2 + b^2)^3 = c^2 + d^2$, prove that, $\dfrac{a^3}{c} + \dfrac{b^3}{b} \geq 1$.
	
%50
	\item Prove that, $\sqrt{a^2 + (1-b)^2} + \sqrt{b^2 + (1-c)^2} + \sqrt{c^2 + (1-a)^2} \geq \dfrac{3\sqrt{2}}{2}$.
	
%51
	\item Give a geometric proof of the following inequalities, for $x, y > 0$ $$\sqrt{\dfrac{x^2+y^2}{2}} \geq \dfrac{x+y}{2} \geq \sqrt{xy} \geq \dfrac{2}{\frac{1}{x} + \frac{1}{y}}.$$

%52
	\item If $a, b, c \in (0, 1)$, prove that, $\sqrt{abc} + \sqrt{(1-a)(1-b)(1-c)} < 1$.
	
%53
	\item If $x_1, x_2, \ldots , x_n \in \mathbb{R^+}$ and $x_1 + x_2 + \ldots + x_n = 1$, prove that, $\sum \limits_{i = 1}^{n} \dfrac{x_i}{1-x_i} \geq \dfrac{n}{n-1}$.
	
%54
	\item Four random points are taken on the four sides of a $1 \times 1$ square. The lengths of the side of the quadrilateral made by joining the four points are $a, b, c, d$ respectively. Show that, $2\sqrt{2} \leq a+b+c+d \leq 4$. \textbf{OR} Find the minimum and maximum values of the perimeter of the quadrilateral.
	
%55
	\item If $a, b, c$ are sides of a triangle, show that, $a^2 + b^2 + c^2 < 2ab + 2bc + 2ca$.
	
%56
	\item Let $a_1, a_2, \ldots, a_n$ be a random arrangement of $(1, 2, \ldots, n)$. Prove that, 
	\begin{enumerate}[(i)]
		\item $\dfrac{a_1^2}{1} + \dfrac{a_2^2}{2} + \ldots + \dfrac{a_n^2}{n} \geq \dfrac{n(n+1)}{2}$.
		\item $\dfrac{a_1}{1^2} + \dfrac{a_2}{2^2} + \ldots + \dfrac{a_n}{n^2} \geq \dfrac{1}{1} + \dfrac{1}{2} + \ldots + \dfrac{1}{n}$.
	
	\end{enumerate}
	
%57
	\item Prove that, $m^3 + 1 > m^2 + m$ where $m \neq 1$ \& $m > -1$.
	
%58
	\item If $x, y > 0$, prove that, $x^5 + y^5 > x^4y+xy^4$ where $x \neq y$.
	
%59
	\item If $x, y, z$ are distinct real numbers, prove that, $$2016x^2 + 2016y^2 + 6z^2 > 2(2013xy + 3yz + 3zx).$$
	
%60
	\item Find all real numbers $x$ and $y$ so that, $x^2 + 2y^2 + \dfrac{1}{2} \leq x(2y + 1)$.
	
%61
	\item For $x, y \in \mathbb{R}$, prove that, $3(x + y + 1)^2 + 1 \geq 3xy$.
	
%62
	\item For $x, y, z \in \mathbb{R}$ such that $xy + yz + zx = -1$, prove that, $x^2 + 5y^2 + 8z^2 \geq 4$.
	
%63
	\item If $x, y, z \in \mathbb{R}$, prove that, $$\dfrac{x^2 + yz}{y + z} + \dfrac{y^2 + zx}{z + x} + \dfrac{z^2 + xy}{x + y} \geq x+y+z.$$
	
%64
	\item If $a, b, c$ are distinct real numbers, prove that, $$\left( \dfrac{a}{b-c} \right)^2 + \left( \dfrac{b}{c-a} \right)^2 + \left( \dfrac{c}{a-b} \right)^2 \geq 2 .$$
	
%65
	\item If $x, y \in (0, 1)$, prove that, $$\dfrac{1}{1-x^2} + \dfrac{1}{1-y^2} \geq \dfrac{2}{1-xy}.$$
	
%66	
	\item If $x, y \in \mathbb{R^+}$, prove that, $$\dfrac{1}{(1+x)^2} + \dfrac{1}{(1+y)^2} \geq \dfrac{1}{1+xy}.$$
	
%67
	\item If $x, y, z \in \mathbb{R^+}$, show that, $x^4 + y^4 + z^2 \geq \sqrt{8}xyz$.
	
%68
	\item If $a, b \in \mathbb{R^+}$, show that, $a^4 + b^4 + 8 \geq 8ab$.
	
%69
	\item If $a, b \in \mathbb{R}$ and $a \neq 0$, show that, $a^2 + b^2 + \dfrac{1}{a^2} + \dfrac{1}{b^2} \geq \sqrt{3}$.
	
%70
	\item If $x, y, z \in \mathbb{R^+}$ and $x+y+z = 1$, prove that, $$xy(x+y)^2 + yz(y+z)^2 + zx(z+x)^2 \geq 4xyz.$$
	
%71
	\item If $a, b, c \in \mathbb{R^+}$ and $abc = 1$, prove that, $$\dfrac{1+ab}{1+a} + \dfrac{1+bc}{1+b} + \dfrac{1+ca}{1+c} \geq 3.$$
	
%72
	\item If $a, b, c \in \mathbb{R^+}$, show that, $\dfrac{a^3}{b} + \dfrac{b^3}{c} + \dfrac{c^3}{a} \geq ab+bc+ca$.
\end{enumerate}





\section{Number Theory}

\begin{enumerate}

%01
	\item When each of $702, 787, 855$ is divided by the positive integer $m$, the remainder is always the positive integer $r$. When each of $412, 722, 815$ is divided by the positive integer $n$, the remainder is always the positive integer $s(\neq r)$. Find $(m+n+r+s)$.
	
%02
	\item Find the number of rational numbers $r$, $0 < r < 1$ such that when $r$ is written as a fraction in lowest terms, the numerator and denominator have a sum of 1000.
	
%03
	\item You are given two bags both having natural numbers. Total no. of numbers in two bags is a prime number. Sum of all the numbers of two bags is 2004. Now, the number 170 is shifted from Bag 1 to Bag 2. For this shifting, the average of the numbers in Bag 1 and that of Bag 2 both increase by 1. Find the total number of numbers.
	
%04
	\item Prove that, $gcd(4m+3, 3m+2) = 1$.
	
%05
	\item Prove that, $gcd(a, b) = gcd(a, a-b) = gcd(b, a-b)$.
	
%06
	\item Show that, 5 consecutive numbers have a coprime with respect to the other 4 numbers.
	
%07
	\item If $a = bq + r$, show that, $gcd(a, b) = gcd(b, r)$.
	
%08
	\item If $a|c$, $b|c$ and $gcd(a, b) = 1$, show that, $ab|c$.
	
%09
	\item If $a|bc$, $gcd(a, b) = 1$, show that, $a|c$.
	
%10
	\item If $\dfrac{1}{p} = \cdot \overline{a_1 a_2 a_3 \cdots a_r}$, show that, $10^r = k \cdot p + 1$.
	
%11
	\item If $n-3|n^3-3$, find all possible values of $n$ where $n \in \mathbb{N}$.
	
%12
	\item $p_1, p_2, p_3$ are three primes such that $p_1 p_2 + 4 = k_1^2$ and $p_1 p_3 + 4 = k_2^2$, $p_2 \neq p_3$. Find the three primes.
	
%13
	\item Prove that, $gcd \left(\dfrac{a^p -1}{a - 1}, a-1\right) = p$ or 1 where $p$ is a prime number.
	
%14
	\item Prove that, $80 | n^5 - n$, where $n$ is an odd natural number.
	
%15
	\item If $x + j | y + j$  $\forall x, y, j \in \mathbb{N}$, prove that, $x = y$.
	
%16
	\item Let $(i_1, i_2, i_3, \ldots , i_n)$ be a permutation of $(1, 2, \ldots, n)$ where $n$ is odd. Prove that,\\ $(1-i_1)(2-i_2) \ldots (n-i_n)$ is even.
	
%17
	\item $gcd(n, 2) = 1$ and $gcd(n, 5) = 1$. Prove that, there exists a number consisted of 1 only divisible by $n$.
\end{enumerate}








\section{Combinatorics}

\begin{enumerate}

%01
	\item \begin{enumerate}[(i)]
		\item Prove that, $\mycomb[n]{r} =  \mycomb[n]{n-r}$.
		
		\item If $0 < r < s \leq n$ and $\myperm[n]{r} = \myperm[n]{s}$, then the value of $(r+s)$ is
		\begin{tasks}(4)
			\task 1
			\task 2
			\task $2n-1$
			\task $2n-2$
		\end{tasks}
	
	\end{enumerate}
	
%02
	\item If $\mycomb[n]{10} = \mycomb[n]{15}$, find $\mycomb[27]{n}$.
	
%03
	\item \begin{enumerate}[(a)]
		\item If $\mycomb[n]{7} = \mycomb[n]{4}$, find $n$.
		\item If the number of radical axes formed out of a given number of circles be same as the number of radical centers, then find the number of given  circles.
	\end{enumerate}
	
%04
	\item \begin{enumerate}[(a)]
		\item If $\mycomb[15]{3r} = \mycomb[15]{r+3}$, find $r$.
		\item If $p = \myperm[n+2]{n+2}$, $q = \myperm[n]{11}$, $r = \myperm[n-11]{n-11}$ and if $p = 182qr$, then show that the value of $n$ is $12$.	
	\end{enumerate}
	
%05
	\item \begin{enumerate}[(a)]
		\item If $\binom{n}{12} = \binom{n}{8}$, find $\binom{n}{17}$ and $\binom{22}{n}$.
		
		\item If $\binom{k^2-k}{2} = \binom{k^2-k}{4}$, then $k = $
		\begin{tasks}(4)
			\task 2
			\task 3
			\task 4
			\task none of these
		\end{tasks}
	
	\end{enumerate}

%06
	\item There are 3 children with 3 corresponding mothers. In how many ways these 6 can enter a hall such that no mother enteres before her child ?
	
%07
	\item $S = \{1, 2, \cdots, n \}$. $P = \{ p(1), p(2), \cdots p(n) \}$ is a permutation of $S$. Find the no. of permutations in which
		\begin{enumerate}[(i)]
		\item If $p(1) < i < j$, then $i$ appears before $j$ in $P$.
		\item If $i < j < p(1)$, then $j$ appears before $i$ in $P$.		
		\end{enumerate}
		
%08
	\item There are 11 persons among which 7 are students and 4 are teachers.
		\begin{enumerate}[(i)]
			\item In how many ways you can make a committee having at least 2 teachers ?
			\item In how many ways you can make a committee of 5 having at least 2 teachers ?		
		\end{enumerate}
		
%09
	\item How many 4 letter words can be formed from the letters of the word  ASSASSINATION ? \\ Repeatation of words is allowed.
	
%10
	\item $S$ is the set of all natural numbers formed by the digits $1, 3, 5, 7$ without any repeatation. Find the sum of the numbers in $S$.
	
%11
	\item Find the sum of the numbers formed by the digits $0, 1, 2, 3, 4, 5, 6, 7$ which are less than 10000.
	
%12
	\item If $1, 2, 3, \ldots$ upto $3333$ is written at random, how many 0s will occur ?
	
%13
	\item $(x_1 + x_2 + x_3)(x_4 + x_5 + x_6 + x_7) = 91$ where $x_1, \ldots, x_7$ are non-negative integers. How many solutions are there to this equation ?
	
%14
	\item How many triangles can be formed with the veritces of a $n$-sided polygon such that none of the sides of the triangles is a side of the polygon ?
	
%15
	\item Let $X = \{1, 2, \ldots, n\}$. Show that the number of subsets of $X$ having $r$ elements, which contain no consecutive integers is $\displaystyle \binom{n-r+1}{r}$.
	
%16
	\item Find the number of arrangements of the letters of the word PESSIMISTIC such that no two S's are together, no two I's are together and no two S and I are adjacent.
	
%17
	\item Find the number of quadruples $(w, x, y, z)$ of non-negative integers which satisfy the inequality $w+x+y+z \leq 1992$.
	
%18
	\item There are 5 ways to express 4 as a sum of 2 non-negative integers in which the order counts : $4 = 4 + 0 = 3 + 1 = 2 + 2 = 1 + 3 = 0 + 4$. What is the number of ways to express $r$ as a sum of $n$ non-negative integers in which the order counts ?
	
%19
	\item There are 6 ways to express 5 as a sum of 3 positive integers in which the order counts : \\ $5 = 3 + 1 + 1 = 2 + 2 + 1 = 2+1+2 = 1+3+1=1+2+2=1+1+3$. Given positive integers $r$ and $n$ with $r \geq n$, what is the number of ways to express $r$ as a sum of $n$ positive integers in which the order counts ?
	
%20
	\item Prove that, 
	\begin{enumerate}[(i)]
		\item $\displaystyle \binom{n}{0} + \binom{n}{1} + \binom{n}{2} + \cdots + \binom{n}{n-1} + \binom{n}{n} = 2^n $.
		\item $\displaystyle \sum \limits_{r \rightarrow odd} \binom{n}{r} = \sum \limits_{r \rightarrow even} \binom{n}{r} = 2^{n-1}$.
		\item $\displaystyle \sum \limits_{r = 0}^{n} = 1\binom{n}{1} + 2\binom{n}{2} + \cdots + n\binom{n}{n} = n 2^{n-1}$.
		\item $\displaystyle \sum \limits_{i = 0}^{r} \binom{m}{i} \binom{n}{r-i} = \binom{m+n}{r}$. \textbf{[Vander Monde's Identity]}
	
	\end{enumerate}
	
%21
	\item How many words can be formed with the letters $A$, $B$, $C$, $D$, $E$, $F$, $G$ such that $A$ \& $B$ are not adjacent, $B$ \& $C$ are not adjacent and $C$ \& $D$ are not adjacent ? 
	
%22
	\item Prove that, the number of triples $(A, B, C)$ where $A$, $B$, $C$ are subsets of $\{1, 2, \ldots, n\}$ such that $A \cap B \cap C = \phi$, $A \cap B \neq \phi$, $B \cap C \neq \phi$ is $7^n - 2 6^n + 5^n$.
	
%23
	\item Suppose you have 20 red, 17 blue, 17 green, 10 brown and 10 yellow coloured balls. At least how many balls you should pick up in order to be sure that you have 15 balls of some colour ?
	
%24
	\item If $S_n = \sum \limits_{r = 0}^{n} \dfrac{1}{\mycomb[n]{r}}$ and $t_n = \sum \limits_{r = 0}^{n} \dfrac{r}{\mycomb[n]{r}}$, show that, $\dfrac{S_n}{t_n} = \dfrac{n}{2}$.
\end{enumerate}

\end{document}
